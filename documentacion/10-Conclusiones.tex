\newpage
\chapter{APÉNDICES}
\newpage

\section{PROBLEMAS ENCONTRADOS DURANTE EL DESARROLLO}



\newpage
\section{CONCLUSIONES}



\newpage
\section{AMPLIACIONES} 
En esta sección se proponen una serie de ampliaciones del sistema para realizar en un futuro.
\paragraph*{Internacionalización de la base de datos}
Actualmente, la aplicación web permite escoger entre dos idiomas, español e inglés, pero no está completamente internacionalizada. Solo se encuentran traducidos los elementos que se definen directamente en el HTML (títulos, menú de navegación, texto presentación). Los datos de los periodos y los componentes (descripciones, características, curiosidades, ...) se encuentran solo en español, ya que se leen directamente de la base de datos, en la que se han introducido los datos en este idioma. Por tanto, una posible ampliación sería terminar la internacionalización de la aplicación traduciendo también estos campos leídos de la base de datos.

\paragraph*{Más tipos de componentes}
En la aplicación desarrollada solo se puede añadir y visualizar componentes de tipo genérico o CPUs. Una ampliación a realizar en un futuro será añadir nuevos componentes como, por ejemplo, GPUs. El proyecto actual se ha desarrollado teniendo en cuenta esta posible ampliación, utilizando la interfaz \textit{MyComponent} y dividiendo los componentes de Angular de forma que se puedan añadir nuevas clases realizando los mínimos cambios posibles.

\paragraph*{Generación automática de carteles}
Una ampliación que sería interesante para el proyecto del museo es la generación automática de carteles desde la parte de administración. Esto ayudaría a crear los carteles para el museo físico, ya que pulsando un único botón se aprovecharían los datos ya presentes en la aplicación para generar un cartel informativo del periodo seleccionado y sus respectivas CPUs.

%\newpage
%\section{REFERENCIAS BIBLIOGRÁFICAS}

%\nocite{*} %El comando bibliography enseña solo las referencias que se hayan usado en el texto. Este comando permite "no citar" todas y así que aparezcan.
%\bibliographystyle{ieeetr} 
%\bibliography{references}
