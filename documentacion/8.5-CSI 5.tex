\subsection{Prueba de Usabilidad}
Una vez desarrollada la aplicación, podemos pasar la guía de usabilidad desarrollada por Yusef Hassan Montero\cite{Usabilidad}.
\begin{longtable}[H]{p{31em}|p{5em}}
\caption{Checklist de pruebas de Usabilidad}
\\ \hline  \\[-1em]
\cellcolor[rgb]{.949,  .949,  .949} \textbf{Criterios} & \cellcolor[rgb]{.949,  .949,  .949}\textbf{¿Cumplido?} \\ \hline  \\[-1em]
\multicolumn{2}{p{36em}}{\cellcolor[rgb]{ .851,  .886,  .953} \textbf{Generales}} \\ \hline \\[-1em]
\textbf{¿Cuáles son los objetivos del sitio web? ¿Son concretos y bien definidos? ¿Los contenidos y servicios que ofrece se corresponden con esos objetivos?} & Sí \\ \hline \\[-1em]
\textbf{¿Tiene una URL correcta, clara y fácil de recordar? ¿Y las URL de sus páginas internas? ¿Son claras y permanentes?} & Sí \\ \hline \\[-1em]
\textbf{¿Muestra de forma precisa y completa qué contenidos o servicios ofrece realmente el sitio web?} El diseño de la página de inicio debe ser diferente al resto de páginas y cumplir la función de 'escaparate' del sitio. & Sí \\ \hline \\[-1em]
\textbf{¿La estructura general del sitio web está orientada al usuario?} Los sitios web deben estructurarse pensando en el usuario, sus objetivos y necesidades. La estructura interna de la empresa u organización, cómo funciona o se organiza no interesan al usuario. & Sí \\ \hline \\[-1em]
\textbf{¿El look \& feel general se corresponde con los objetivos, características, contenidos y servicios del sitio web?} Ciertas combinaciones de colores ofrecen imágenes más o menos formales, serias o profesionales. & Sí \\ \hline \\[-1em]
\textbf{¿Es coherente el diseño general del sitio web?} Se debe mantener una coherencia y uniformidad en las estructuras y colores de todas las páginas. Esto sirve para que el usuario no se desoriente en su navegación. & Sí \\ \hline \\[-1em]
\textbf{¿Es reconocible el diseño general del sitio web?} Cuánto más se parezca el sitio web al resto de sitios web, más fácil será de usar. & Sí \\ \hline \\[-1em]
\textbf{¿El sitio web se actualiza periódicamente? ¿Indica cuándo se actualiza?} Las fechas que se muestren en la página deben corresponderse con actualizaciones, noticias, eventos...no con la fecha del sistema del usuario. & - \\ \hline \\[-1em]
\multicolumn{2}{p{36em}}{\cellcolor[rgb]{ .851,  .886,  .953} \textbf{Identidad e información}} \\ \hline \\[-1em]
\textbf{¿Se muestra claramente la identidad de la empresa-sitio a través de todas las páginas?} & Sí \\ \hline \\[-1em]
El Logotipo, \textbf{¿es significativo, identificable y suficientemente visible?} & Sí \\ \hline \\[-1em]
El eslogan o tagline, \textbf{¿expresa realmente qué es la empresa y qué servicios ofrece?} & - \\ \hline \\[-1em]
\textbf{¿Se ofrece algún enlace con información sobre la empresa, sitio web, 'webmaster’...? } & Sí \\ \hline \\[-1em]
\textbf{¿Se proporciona mecanismos para ponerse en contacto con la empresa?} (email, teléfono, dirección postal, fax...) & - \\ \hline \\[-1em]
\textbf{¿Se proporciona información sobre la protección de datos de carácter personal de los clientes o los derechos de autor de los contenidos del sitio web?} & - \\ \hline \\[-1em]
En artículos, noticias, informes... \textbf{¿Se muestra claramente información sobre el autor, fuentes y fechas de creación y revisión del documento? } & - \\ \hline \\[-1em]
\multicolumn{2}{p{36em}}{\cellcolor[rgb]{ .851,  .886,  .953} \textbf{Lenguaje y redacción}} \\ \hline \\[-1em]
\textbf{¿El sitio web habla el mismo lenguaje que sus usuarios?} Se debe evitar usar un lenguaje corporativista. Así mismo, hay que prestarle especial atención al idioma, y ofrecer versiones del sitio en diferentes idiomas cuando sea necesario. & Sí \\ \hline \\[-1em]
\textbf{¿Emplea un lenguaje claro y conciso?} & Sí \\ \hline \\[-1em]
\textbf{¿Es amigable, familiar y cercano?} Es decir, lo contrario a utilizar un lenguaje constantemente imperativo, mensajes crípticos, o tratar con "desprecio" al usuario. & Sí \\ \hline \\[-1em]
\textbf{¿1 párrafo = 1 idea?} Cada párrafo es un objeto informativo. Trasmita ideas, mensajes...Se deben evitar párrafos vacíos o varios mensajes en un mismo párrafo. & Sí \\ \hline \\[-1em]
\multicolumn{2}{p{36em}}{\cellcolor[rgb]{ .851,  .886,  .953} \textbf{Rotulado}} \\ \hline \\[-1em]
Los rótulos, \textbf{¿son significativos?} Ejemplo: evitar rótulos del tipo "haga clic aquí". & Sí \\ \hline \\[-1em]
\textbf{¿Usa rótulos estándar?} Siempre que exista un "estándar" comúnmente aceptado para el caso concreto, como "Mapa del Sitio" o "Acerca de...". & Sí \\ \hline \\[-1em]
\textbf{¿Usa un único sistema de organización, bien definido y claro?} No se deben mezclar diferentes. Los sistemas de organización son: alfabético, geográfico, cronológico, temático, orientado a tareas, orientado al público y orientado a metáforas. & Sí \\ \hline \\[-1em]
\textbf{¿Utiliza un sistema de rotulado controlado y preciso?} Por ejemplo, si un enlace tiene el rótulo "Quiénes somos", no puede dirigir a una página cuyo encabezamiento sea "Acerca de" & Sí \\ \hline \\[-1em]
El título de las páginas, \textbf{¿Es correcto? ¿Ha sido planificado?} Relacionado con la capacidad para poder buscar y encontrar el sitio web. & Sí \\ \hline \\[-1em]
\multicolumn{2}{p{36em}}{\cellcolor[rgb]{ .851,  .886,  .953} \textbf{Estructura y navegación}} \\ \hline \\[-1em]
La estructura de organización y navegación, \textbf{¿Es la más adecuada?} Hay varios tipos de estructuras: jerárquicas, hipertextual, facetada... & Sí \\ \hline \\[-1em]
En el caso de estructura jerárquica, \textbf{¿Mantiene un equilibrio entre Profundidad y Anchura?} & - \\ \hline \\[-1em]
En el caso de ser puramente hipertextual, \textbf{¿Están todos los clústeres de nodos comunicados?} Aquí se mide la distancia entre nodos. & Sí \\ \hline \\[-1em]
\textbf{¿Los enlaces son fácilmente reconocibles como tales? ¿Su caracterización indica su estado (visitados, activos...)?} Los enlaces no sólo deben reconocerse como tales, sino que su caracterización debe indicar su estado, y ser reconocidos como una unidad  & No \\ \hline \\[-1em]
En menús de navegación, \textbf{¿Se ha controlado el número de elementos y de términos por elemento para no producir sobrecarga memorística?} No se deben superar los 7±2 elementos, ni los 2 o, como mucho, 3 términos por elemento. & Sí \\ \hline \\[-1em]
\textbf{¿Es predecible la respuesta del sistema antes de hacer clic sobre el enlace?} Relacionado con el nivel de significación del rótulo del enlace, aunque también con: el uso de globos de texto, información contextual, la barra de estado del navegador... & Sí \\ \hline \\[-1em]
\textbf{¿Se ha controlado que no haya enlaces que no llevan a ningún sitio?} Enlaces que no llevan a ningún sitio: Los enlaces rotos, y los que enlazan con la misma página que se está visualizando (por ejemplo, enlaces a la "home" desde la misma página de inicio) & Sí \\ \hline \\[-1em]
\textbf{¿Existen elementos de navegación que orienten al usuario acerca de dónde está y cómo deshacer su navegación?} ...como breadcrumbs, enlaces a la página de inicio...recuerde que el logo también es recomendable que enlace con la página de inicio. & Sí \\ \hline \\[-1em]
Las imágenes enlace, \textbf{¿se reconocen como clicables? ¿Incluyen un atributo 'title' describiendo la página de destino?} En este sentido, también hay que cuidar que no haya imágenes que parezcan enlaces y en realidad no lo sean. & No \\ \hline \\[-1em]
\textbf{¿Se ha evitado la redundancia de enlaces?} & Sí \\ \hline \\[-1em]
\textbf{¿Se ha controlado que no haya páginas "huérfanas"?} Páginas huérfanas: que, aun siendo enlazadas desde otras páginas, éstas no enlacen con ninguna.  & Sí \\ \hline \\[-1em]
\multicolumn{2}{p{36em}}{\cellcolor[rgb]{ .851,  .886,  .953} \textbf{Layout de la página}} \\ \hline \\[-1em]
\textbf{¿Se aprovechan las zonas de alta jerarquía informativa de la página para contenidos de mayor relevancia?} (como por ejemplo la zona central)  & Sí \\ \hline \\[-1em]
\textbf{¿Se ha evitado la sobrecarga informativa?} Esto se consigue haciendo un uso correcto de colores, efectos tipográficos y agrupaciones para discriminar información. Los grupos diferentes de objetos informativos de una página deben ser 7±2.  & Sí \\ \hline \\[-1em]
\textbf{¿Es una interfaz limpia, sin ruido visual?} & Sí \\ \hline \\[-1em]
\textbf{¿Existen zonas en "blanco" entre los objetos informativos de la página para poder descansar la vista?} & Sí \\ \hline \\[-1em]
\textbf{¿Se hace un uso correcto del espacio visual de la página?} Es decir, que no se desaproveche demasiado espacio con elementos de decoración, o grandes zonas en "blanco", y que no se adjudique demasiado espacio a elementos de menor importancia.  & Sí \\ \hline \\[-1em]
\textbf{¿Se utiliza correctamente la jerarquía visual para expresar las relaciones del tipo \"parte de\" entre los elementos de la página?} (La jerarquía visual se utiliza para orientar al usuario) & Sí \\ \hline \\[-1em]
\textbf{¿Se ha controlado la longitud de página?} Se debe evitar en la medida de lo posible el scrolling. Si la página es muy extensa, se debe fraccionar.  & No \\ \hline \\[-1em]
\multicolumn{2}{p{36em}}{\cellcolor[rgb]{ .851,  .886,  .953} \textbf{Elementos multimedia}} \\ \hline \\[-1em]
\textbf{¿Las fotografías están bien recortadas? ¿Son comprensibles? ¿Se ha cuidado su resolución? } & Sí \\ \hline \\[-1em]
\textbf{¿Las metáforas visuales son reconocibles y comprensibles por cualquier usuario?} (prestar especial atención a usuarios de otros países y culturas)  & - \\ \hline \\[-1em]
\textbf{¿El uso de imágenes o animaciones proporciona algún tipo de valor añadido? } & Sí \\ \hline \\[-1em]
\textbf{¿Se ha evitado el uso de animaciones cíclicas?} & Sí \\ \hline \\[-1em]
\multicolumn{2}{p{36em}}{\cellcolor[rgb]{ .851,  .886,  .953} \textbf{Accesibilidad}} \\ \hline \\[-1em]
\textbf{¿El tamaño de fuente se ha definido de forma relativa, o por lo menos, la fuente es lo suficientemente grande como para no dificultar la legibilidad del texto? } & Sí \\ \hline \\[-1em]
\textbf{¿El tipo de fuente, efectos tipográficos, ancho de línea y alineación empleadas facilitan la lectura?} & Sí \\ \hline \\[-1em]
\textbf{¿Existe un alto contraste entre el color de fuente y el fondo?} & Sí \\ \hline \\[-1em]
\textbf{¿Incluyen las imágenes atributos 'alt' que describan su contenido? } & Sí \\ \hline \\[-1em]
\textbf{¿Es compatible el sitio web con los diferentes navegadores? ¿Se visualiza correctamente con diferentes resoluciones de pantalla?} Se debe prestar atención a: JScript, CSS, tablas, fuentes...  & Sí \\ \hline \\[-1em]
\textbf{¿Puede el usuario disfrutar de todos los contenidos del sitio web sin necesidad de tener que descargar e instalar plugins adicionales?} & Sí \\ \hline \\[-1em]
\textbf{¿Se ha controlado el peso de la página?} Se deben optimizar las imágenes, controlar el tamaño del código JScript...  & No \\ \hline \\[-1em]
\textbf{¿Se puede imprimir la página sin problemas?} Leer en pantalla es molesto, por lo que muchos usuarios preferirán imprimir las páginas para leerlas. Se debe asegurar que se puede imprimir la página (no salen partes cortadas), y que el resultado es legible.  & Sí \\ \hline \\[-1em]
\multicolumn{2}{p{36em}}{\cellcolor[rgb]{ .851,  .886,  .953} \textbf{Control y retroalimentación}} \\ \hline \\[-1em]
\textbf{¿Tiene el usuario todo el control sobre el interfaz?} Se debe evitar el uso de ventanas pop-up, ventanas que se abren a pantalla completa, banners intrusivos...  & Sí \\ \hline \\[-1em]
\textbf{¿Se informa constantemente al usuario acerca de lo que está pasando?} Si el usuario tiene que esperar hasta que se termine una operación, se debe mostrar un mensaje indicándoselo y que debe esperar, con el tiempo de espera estimado o una barra de progreso.  & - \\ \hline \\[-1em]
\textbf{¿Se informa al usuario de lo que ha pasado?} Por ejemplo, cuando un usuario valora un artículo o responde a una encuesta, se le debe informar de que su voto ha sido procesado correctamente. & Sí \\ \hline \\[-1em]
Cuando se produce un error, \textbf{¿se informa de forma clara y no alarmista al usuario de lo ocurrido y de cómo solucionar el problema?} Siempre es mejor intentar evitar que se produzcan errores a tener que informar al usuario del error.  & Sí \\ \hline \\[-1em]
\textbf{¿Posee el usuario libertad para actuar?} NO restringir la libertad del usuario: Uso de animaciones que no pueden ser "saltadas", páginas en las que desaparecen los botones de navegación, no impida al usuario poder usar el botón derecho de su ratón... & Sí \\ \hline \\[-1em]
\textbf{¿Se ha controlado el tiempo de respuesta?} Esto tiene que ver con el peso de cada página (accesibilidad) y tiene relación con el tiempo que tarda el servidor en finalizar una tarea y responder. El tiempo máximo que esperará un usuario son 10 segundos  & Sí \\ \hline \\[-1em]
\multicolumn{2}{p{36em}}{\cellcolor[rgb]{ .851,  .886,  .953} \textbf{Aclaraciones}} \\ \hline \\[-1em]
\textbf{¿Se ha evaluado adecuadamente la orientación del usuario?} (Donde estoy, como volver, que he visitado, que va a pasar) & Sí \\ \hline \\[-1em]
\textbf{¿Se ha usado correctamente la publicidad?} & - \\ \hline
\end{longtable}


A continuación, se mostrarán las respuestas de los usuarios a los cuestionarios diseñados anteriormente (sección \ref{sec:cuestionario-usabilidad}). Las pruebas han sido realizadas con tres usuarios para el museo y con el tutor del trabajo para la administración.
\paragraph*{Usuario 1}
\begin{table}[H]
\centering
\caption{Cuestionario de carácter general (Usuario 1)}
\begin{tabular}{p{36em}}
\toprule
\rowcolor[rgb]{ .949,  .949,  .949} \textbf{¿Usa un ordenador frecuentemente?} \\ \midrule
\vspace{-4mm}
\begin{enumerate}
\item \colorbox{blue!30}{Todos los días}
\item Varias veces a la semana
\item Ocasionalmente
\item Nunca
\end{enumerate} \\ \midrule
\rowcolor[rgb]{ .949,  .949,  .949} \textbf{¿Qué tipo de actividades realiza con el ordenador?} \\ \midrule
\vspace{-4mm}
\begin{enumerate}
\item Es parte de mi trabajo o profesión
\item \colorbox{blue!30}{Lo uso básicamente para ocio}
\item Solo empleo aplicaciones estilo Office
\item Únicamente leo el correo y navego ocasionalmente
\end{enumerate} \\ \midrule
\rowcolor[rgb]{ .949,  .949,  .949} \textbf{¿Qué busca Vd. principalmente en una aplicación web?} \\ \midrule
\vspace{-4mm}
\begin{enumerate}
\item \colorbox{blue!30}{Que sea fácil de navegar}
\item Que sea intuitiva
\item Que sea rápida
\item Que tenga todas las funciones necesarias
\end{enumerate} \\ \bottomrule
\end{tabular}
\end{table}

\begin{table}[H]
\centering
\caption{Cuestionario sobre la aplicación (Usuario 1)}
\begin{tabular}{p{15em}|p{4em}|p{7.5em}|p{7.5em}|p{3em}}
\toprule
\rowcolor[rgb]{.949,  .949,  .949} \textbf{Facilidad de uso} & \textbf{Siempre} & \textbf{Frecuentemente} & \textbf{Ocasionalmente} & \textbf{Nunca} \\ \midrule
\textit{¿Sabe dónde está dentro de la aplicación?} & X& & & \\ \midrule
\textit{¿Necesita ayuda para utilizar la aplicación?} & & & &X \\ \midrule
\textit{¿Le resulta sencillo el uso de la aplicación?} &X & & & \\ \midrule
\textit{¿Identifica fácilmente la información que se le presenta?} & X& & & \\ \midrule
\rowcolor[rgb]{.949,  .949,  .949} \textbf{Funcionalidad} & \textbf{Siempre} & \textbf{Frecuentemente} & \textbf{Ocasionalmente} & \textbf{Nunca} \\ \midrule
\textit{¿Funciona cada tarea como Vd. espera?} & X& & & \\ \midrule
\textit{¿El tiempo de respuesta de la aplicación es muy grande?} & & & &X \\ \midrule
\rowcolor[rgb]{ .851,  .886,  .953} \multicolumn{5}{p{36em}}{\textbf{Calidad del interfaz}} \\ \midrule
\rowcolor[rgb]{.949,  .949,  .949} \textbf{Aspectos gráficos} & \textbf{Muy adecuado} & \textbf{Adecuado} & \textbf{Poco adecuado} & \textbf{Nada adecuado} \\ \midrule
\textit{El tipo y el tamaño de letra es} & X & & & \\ \midrule
\textit{Los iconos e imágenes usados son} & X & & & \\ \midrule
\textit{Los colores empleados son} & X & & & \\ \midrule
\rowcolor[rgb]{.949,  .949,  .949}\multicolumn{2}{p{19em}|}{\textbf{Diseño de la interfaz}} & \textbf{Sí} & \textbf{A veces} & \textbf{No} \\ \midrule
\multicolumn{2}{p{19em}|}{\textit{¿Le resulta fácil de usar?}} &X & & \\ \midrule
\multicolumn{2}{p{19em}|}{\textit{¿El diseño de las pantallas es claro y atractivo?}} & & & X\\ \midrule
\multicolumn{2}{p{19em}|}{\textit{¿Es coherente el diseño general del sitio web?}} & & &X \\ \midrule
\multicolumn{2}{p{19em}|}{\textit{¿Cree que el programa está bien estructurado?}} & X& & \\ \midrule
\rowcolor[rgb]{ .851,  .886,  .953}\multicolumn{5}{p{36em}}{\textbf{Observaciones}} \\ \midrule
\multicolumn{5}{p{36em}}{Hacer más visible la división de los periodos en la línea temporal.} \\ \bottomrule
\end{tabular}
\end{table}

\begin{table}[H]
\centering
\caption{Actividades guiadas (Usuario 1)}
\begin{tabular}{p{12em}p{24em}}
\toprule
\rowcolor[rgb]{ .949,  .949,  .949}\multicolumn{2}{p{36em}}{\textbf{\textit{Navegar por la línea temporal}}} \\ \midrule
\textbf{Tiempo empleado:} & 1 minuto \\ \midrule
\textbf{Problemas encontrados:} & No se identificó bien el criterio de división de los componentes en periodos. \\ \midrule
\textbf{Observaciones:} & Se propuso hacer una división de periodos por etapas (años 70, 80,...) \\ \midrule
\rowcolor[rgb]{ .949,  .949,  .949}\multicolumn{2}{p{36em}}{\textbf{\textit{Búsqueda por años}}} \\ \midrule
\textbf{Tiempo empleado:} & 1 minuto \\ \midrule
\textbf{Problemas encontrados:} & Ninguno \\ \midrule
\textbf{Observaciones:} &  \\ \midrule
\rowcolor[rgb]{ .949,  .949,  .949}\multicolumn{2}{p{36em}}{\textbf{\textit{Búsqueda por nombre}}} \\ \midrule
\textbf{Tiempo empleado:} & 1 minuto \\ \midrule
\textbf{Problemas encontrados:} & Ninguno \\ \midrule
\textbf{Observaciones:} &  \\ \midrule
\rowcolor[rgb]{ .949,  .949,  .949}\multicolumn{2}{p{36em}}{\textbf{\textit{Ver detalles de un periodo}}} \\ \midrule
\textbf{Tiempo empleado:} & 1 minuto \\ \midrule
\textbf{Problemas encontrados:} & Ninguno \\ \midrule
\textbf{Observaciones:} &  \\ \midrule
\rowcolor[rgb]{ .949,  .949,  .949}\multicolumn{2}{p{36em}}{\textbf{\textit{Ver detalles de un componente}}} \\ \midrule
\textbf{Tiempo empleado:} & 1 minuto \\ \midrule
\textbf{Problemas encontrados:} & Ninguno \\ \midrule
\textbf{Observaciones:} &  \\ \bottomrule
\end{tabular}
\end{table}


\paragraph*{Usuario 2}
\begin{table}[H]
\centering
\caption{Cuestionario de carácter general (Usuario 2)}
\begin{tabular}{p{36em}}
\toprule
\rowcolor[rgb]{ .949,  .949,  .949} \textbf{¿Usa un ordenador frecuentemente?} \\ \midrule
\vspace{-4mm}
\begin{enumerate}
\item  \colorbox{blue!30}{Todos los días}
\item Varias veces a la semana
\item Ocasionalmente
\item Nunca
\end{enumerate} \\ \midrule
\rowcolor[rgb]{ .949,  .949,  .949} \textbf{¿Qué tipo de actividades realiza con el ordenador?} \\ \midrule
\vspace{-4mm}
\begin{enumerate}
\item  \colorbox{blue!30}{Es parte de mi trabajo o profesión}
\item Lo uso básicamente para ocio
\item Solo empleo aplicaciones estilo Office
\item Únicamente leo el correo y navego ocasionalmente
\end{enumerate} \\ \midrule
\rowcolor[rgb]{ .949,  .949,  .949} \textbf{¿Qué busca Vd. principalmente en una aplicación web?} \\ \midrule
\vspace{-4mm}
\begin{enumerate}
\item  \colorbox{blue!30}{Que sea fácil de navegar}
\item Que sea intuitiva
\item Que sea rápida
\item Que tenga todas las funciones necesarias
\end{enumerate} \\ \bottomrule
\end{tabular}
\end{table}

\begin{table}[H]
\centering
\caption{Cuestionario sobre la aplicación (Usuario 2)}
\begin{tabular}{p{15em}|p{4em}|p{7.5em}|p{7.5em}|p{3em}}
\toprule
\rowcolor[rgb]{.949,  .949,  .949} \textbf{Facilidad de uso} & \textbf{Siempre} & \textbf{Frecuentemente} & \textbf{Ocasionalmente} & \textbf{Nunca} \\ \midrule
\textit{¿Sabe dónde está dentro de la aplicación?} & X& & & \\ \midrule
\textit{¿Necesita ayuda para utilizar la aplicación?} & & & & X\\ \midrule
\textit{¿Le resulta sencillo el uso de la aplicación?} & &X & & \\ \midrule
\textit{¿Identifica fácilmente la información que se le presenta?} & &X & & \\ \midrule
\rowcolor[rgb]{.949,  .949,  .949} \textbf{Funcionalidad} & \textbf{Siempre} & \textbf{Frecuentemente} & \textbf{Ocasionalmente} & \textbf{Nunca} \\ \midrule
\textit{¿Funciona cada tarea como Vd. espera?} & &X & & \\ \midrule
\textit{¿El tiempo de respuesta de la aplicación es muy grande?} & & & &X \\ \midrule
\rowcolor[rgb]{ .851,  .886,  .953} \multicolumn{5}{p{36em}}{\textbf{Calidad del interfaz}} \\ \midrule
\rowcolor[rgb]{.949,  .949,  .949} \textbf{Aspectos gráficos} & \textbf{Muy adecuado} & \textbf{Adecuado} & \textbf{Poco adecuado} & \textbf{Nada adecuado} \\ \midrule
\textit{El tipo y el tamaño de letra es} & & X& & \\ \midrule
\textit{Los iconos e imágenes usados son} &X & & & \\ \midrule
\textit{Los colores empleados son} & &X & & \\ \midrule
\rowcolor[rgb]{.949,  .949,  .949}\multicolumn{2}{p{19em}|}{\textbf{Diseño de la interfaz}} & \textbf{Sí} & \textbf{A veces} & \textbf{No} \\ \midrule
\multicolumn{2}{p{19em}|}{\textit{¿Le resulta fácil de usar?}} &X & & \\ \midrule
\multicolumn{2}{p{19em}|}{\textit{¿El diseño de las pantallas es claro y atractivo?}} &X & & \\ \midrule
\multicolumn{2}{p{19em}|}{\textit{¿Es coherente el diseño general del sitio web?}} & &X & \\ \midrule
\multicolumn{2}{p{19em}|}{\textit{¿Cree que el programa está bien estructurado?}} &X & & \\ \midrule
\rowcolor[rgb]{ .851,  .886,  .953}\multicolumn{5}{p{36em}}{\textbf{Observaciones}} \\ \midrule
\multicolumn{5}{p{36em}}{En la vista de detalles del componente, el menu izquierdo donde se muestra el periodo siguiente tenía la apariencia de que al seleccionarlo se desplegaría un menú con sus componentes en lugar de navegar a los detalles del periodo.} \\ \bottomrule
\end{tabular}
\end{table}

\begin{table}[H]
\centering
\caption{Actividades guiadas (Usuario 2)}
\begin{tabular}{p{12em}p{24em}}
\toprule
\rowcolor[rgb]{ .949,  .949,  .949}\multicolumn{2}{p{36em}}{\textbf{\textit{Navegar por la línea temporal}}} \\ \midrule
\textbf{Tiempo empleado:} & 1 minuto \\ \midrule
\textbf{Problemas encontrados:} & No hay distinción en la letra de los años del periodo y la de los nombres de los componentes. \\ \midrule
\textbf{Observaciones:} & Hacer la letra diferente para apreciar mejor que los nombres de los componentes son enlaces.\\ \midrule
\rowcolor[rgb]{ .949,  .949,  .949}\multicolumn{2}{p{36em}}{\textbf{\textit{Búsqueda por años}}} \\ \midrule
\textbf{Tiempo empleado:} & 1 minuto \\ \midrule
\textbf{Problemas encontrados:} & Ninguno \\ \midrule
\textbf{Observaciones:} &  \\ \midrule
\rowcolor[rgb]{ .949,  .949,  .949}\multicolumn{2}{p{36em}}{\textbf{\textit{Búsqueda por nombre}}} \\ \midrule
\textbf{Tiempo empleado:} & 1 minuto \\ \midrule
\textbf{Problemas encontrados:} & Se esperaba una búsqueda que se actualizara automáticamente al escribir y no solo al pulsar \textit{Enter} \\ \midrule 
\textbf{Observaciones:} & Cambiar el método de búsqueda para actualizar los resultados con cada entrada del usuario. \\ \midrule
\rowcolor[rgb]{ .949,  .949,  .949}\multicolumn{2}{p{36em}}{\textbf{\textit{Ver detalles de un periodo}}} \\ \midrule
\textbf{Tiempo empleado:} & 1 minuto \\ \midrule
\textbf{Problemas encontrados:} & Ninguno \\ \midrule
\textbf{Observaciones:} &  \\ \midrule
\rowcolor[rgb]{ .949,  .949,  .949}\multicolumn{2}{p{36em}}{\textbf{\textit{Ver detalles de un componente}}} \\ \midrule
\textbf{Tiempo empleado:} & 1 minuto \\ \midrule
\textbf{Problemas encontrados:} & Se esperaba un desplegable de los componentes de un periodo distinto al actual en el menú lateral izquierdo. \\ \midrule
\textbf{Observaciones:} & Cambiar el icono que acompaña al nombre de los periodos para que se entienda mejor que es un enlace y no un desplegable. \\ \bottomrule
\end{tabular}
\end{table}


\paragraph*{Usuario 3}
\begin{table}[H]
\centering
\caption{Cuestionario de carácter general (Usuario 3)}
\begin{tabular}{p{36em}}
\toprule
\rowcolor[rgb]{ .949,  .949,  .949} \textbf{¿Usa un ordenador frecuentemente?} \\ \midrule
\vspace{-4mm}
\begin{enumerate}
\item Todos los días
\item \colorbox{blue!30}{Varias veces a la semana}
\item Ocasionalmente
\item Nunca
\end{enumerate} \\ \midrule
\rowcolor[rgb]{ .949,  .949,  .949} \textbf{¿Qué tipo de actividades realiza con el ordenador?} \\ \midrule
\vspace{-4mm}
\begin{enumerate}
\item Es parte de mi trabajo o profesión
\item Lo uso básicamente para ocio
\item \colorbox{blue!30}{Solo empleo aplicaciones estilo Office}
\item Únicamente leo el correo y navego ocasionalmente
\end{enumerate} \\ \midrule
\rowcolor[rgb]{ .949,  .949,  .949} \textbf{¿Qué busca Vd. principalmente en una aplicación web?} \\ \midrule
\vspace{-4mm}
\begin{enumerate}
\item Que sea fácil de navegar
\item \colorbox{blue!30}{Que sea intuitiva}
\item Que sea rápida
\item Que tenga todas las funciones necesarias
\end{enumerate} \\ \bottomrule
\end{tabular}
\end{table}

\begin{table}[H]
\centering
\caption{Cuestionario sobre la aplicación (Usuario 3)}
\begin{tabular}{p{15em}|p{4em}|p{7.5em}|p{7.5em}|p{3em}}
\toprule
\rowcolor[rgb]{.949,  .949,  .949} \textbf{Facilidad de uso} & \textbf{Siempre} & \textbf{Frecuentemente} & \textbf{Ocasionalmente} & \textbf{Nunca} \\ \midrule
\textit{¿Sabe dónde está dentro de la aplicación?} & X& & & \\ \midrule
\textit{¿Necesita ayuda para utilizar la aplicación?} & & &X & \\ \midrule
\textit{¿Le resulta sencillo el uso de la aplicación?} & X& & & \\ \midrule
\textit{¿Identifica fácilmente la información que se le presenta?} &X & & & \\ \midrule
\rowcolor[rgb]{.949,  .949,  .949} \textbf{Funcionalidad} & \textbf{Siempre} & \textbf{Frecuentemente} & \textbf{Ocasionalmente} & \textbf{Nunca} \\ \midrule
\textit{¿Funciona cada tarea como Vd. espera?} & &X & & \\ \midrule
\textit{¿El tiempo de respuesta de la aplicación es muy grande?} & & & &X \\ \midrule
\rowcolor[rgb]{ .851,  .886,  .953} \multicolumn{5}{p{36em}}{\textbf{Calidad del interfaz}} \\ \midrule
\rowcolor[rgb]{.949,  .949,  .949} \textbf{Aspectos gráficos} & \textbf{Muy adecuado} & \textbf{Adecuado} & \textbf{Poco adecuado} & \textbf{Nada adecuado} \\ \midrule
\textit{El tipo y el tamaño de letra es} & X& & & \\ \midrule
\textit{Los iconos e imágenes usados son} & X& & & \\ \midrule
\textit{Los colores empleados son} & X& & & \\ \midrule
\rowcolor[rgb]{.949,  .949,  .949}\multicolumn{2}{p{19em}|}{\textbf{Diseño de la interfaz}} & \textbf{Sí} & \textbf{A veces} & \textbf{No} \\ \midrule
\multicolumn{2}{p{19em}|}{\textit{¿Le resulta fácil de usar?}} & X& & \\ \midrule
\multicolumn{2}{p{19em}|}{\textit{¿El diseño de las pantallas es claro y atractivo?}} & X& & \\ \midrule
\multicolumn{2}{p{19em}|}{\textit{¿Es coherente el diseño general del sitio web?}} & X& & \\ \midrule
\multicolumn{2}{p{19em}|}{\textit{¿Cree que el programa está bien estructurado?}} & X& & \\ \midrule
\rowcolor[rgb]{ .851,  .886,  .953}\multicolumn{5}{p{36em}}{\textbf{Observaciones}} \\ \midrule
\multicolumn{5}{p{36em}}{Que las imágenes que pueden abrirse como galería se resalten más.} \\ \bottomrule
\end{tabular}
\end{table}

\begin{table}[H]
\centering
\caption{Actividades guiadas (Usuario 3)}
\begin{tabular}{p{12em}p{24em}}
\toprule
\rowcolor[rgb]{ .949,  .949,  .949}\multicolumn{2}{p{36em}}{\textbf{\textit{Navegar por la línea temporal}}} \\ \midrule
\textbf{Tiempo empleado: } & 1 minuto \\ \midrule
\textbf{Problemas encontrados: } & Ninguno \\ \midrule
\textbf{Observaciones: } &  \\ \midrule
\rowcolor[rgb]{ .949,  .949,  .949}\multicolumn{2}{p{36em}}{\textbf{\textit{Búsqueda por años}}} \\ \midrule
\textbf{Tiempo empleado: } & 1 minuto \\ \midrule
\textbf{Problemas encontrados: } & Ninguno \\ \midrule
\textbf{Observaciones: } &  \\ \midrule
\rowcolor[rgb]{ .949,  .949,  .949}\multicolumn{2}{p{36em}}{\textbf{\textit{Búsqueda por nombre}}} \\ \midrule
\textbf{Tiempo empleado: } & 1 minuto \\ \midrule
\textbf{Problemas encontrados: } & Ninguno \\ \midrule
\textbf{Observaciones: } &  \\ \midrule
\rowcolor[rgb]{ .949,  .949,  .949}\multicolumn{2}{p{36em}}{\textbf{\textit{Ver detalles de un periodo}}} \\ \midrule
\textbf{Tiempo empleado: } & 1 minuto \\ \midrule
\textbf{Problemas encontrados: } & Ninguno \\ \midrule
\textbf{Observaciones: } &  \\ \midrule
\rowcolor[rgb]{ .949,  .949,  .949}\multicolumn{2}{p{36em}}{\textbf{\textit{Ver detalles de un componente}}} \\ \midrule
\textbf{Tiempo empleado:} & 1 minuto \\ \midrule
\textbf{Problemas encontrados: } &  No se identificaron correctamente las imágenes como clickables para abrir la galería.\\ \midrule
\textbf{Observaciones:} & Remarcar más cada imagen. \\ \bottomrule
\end{tabular}
\end{table}


\paragraph*{Usuario 4 - administrador}
\begin{table}[H]
\centering
\caption{Cuestionario de carácter general (Usuario 5 - administrador)}
\begin{tabular}{p{36em}}
\toprule
\rowcolor[rgb]{ .949,  .949,  .949} \textbf{¿Usa un ordenador frecuentemente?} \\ \midrule
\vspace{-4mm}
\begin{enumerate}
\item \colorbox{blue!30}{Todos los días}
\item Varias veces a la semana
\item Ocasionalmente
\item Nunca
\end{enumerate} \\ \midrule
\rowcolor[rgb]{ .949,  .949,  .949} \textbf{¿Qué tipo de actividades realiza con el ordenador?} \\ \midrule
\vspace{-4mm}
\begin{enumerate}
\item \colorbox{blue!30}{Es parte de mi trabajo o profesión}
\item Lo uso básicamente para ocio
\item Solo empleo aplicaciones estilo Office
\item Únicamente leo el correo y navego ocasionalmente
\end{enumerate} \\ \midrule
\rowcolor[rgb]{ .949,  .949,  .949} \textbf{¿Qué busca Vd. principalmente en una aplicación web?} \\ \midrule
\vspace{-4mm}
\begin{enumerate}
\item Que sea fácil de navegar
\item Que sea intuitiva
\item Que sea rápida
\item \colorbox{blue!30}{Que tenga todas las funciones necesarias}
\end{enumerate} \\ \bottomrule
\end{tabular}
\end{table}

\begin{table}[H]
\centering
\caption{Cuestionario sobre la aplicación (Usuario 4 - administrador)}
\begin{tabular}{p{15em}|p{4em}|p{7.5em}|p{7.5em}|p{3em}}
\toprule
\rowcolor[rgb]{.949,  .949,  .949} \textbf{Facilidad de uso} & \textbf{Siempre} & \textbf{Frecuentemente} & \textbf{Ocasionalmente} & \textbf{Nunca} \\ \midrule
\textit{¿Sabe dónde está dentro de la aplicación?} &X & & & \\ \midrule
\textit{¿Necesita ayuda para utilizar la aplicación?} & &X & & \\ \midrule
\textit{¿Le resulta sencillo el uso de la aplicación?} & X& & & \\ \midrule
\textit{¿Identifica fácilmente la información que se le presenta?} & X& & & \\ \midrule
\rowcolor[rgb]{.949,  .949,  .949} \textbf{Funcionalidad} & \textbf{Siempre} & \textbf{Frecuentemente} & \textbf{Ocasionalmente} & \textbf{Nunca} \\ \midrule
\textit{¿Funciona cada tarea como Vd. espera?} & X& & & \\ \midrule
\textit{¿El tiempo de respuesta de la aplicación es muy grande?} & & & &X \\ \midrule
\rowcolor[rgb]{ .851,  .886,  .953} \multicolumn{5}{p{36em}}{\textbf{Calidad del interfaz}} \\ \midrule
\rowcolor[rgb]{.949,  .949,  .949} \textbf{Aspectos gráficos} & \textbf{Muy adecuado} & \textbf{Adecuado} & \textbf{Poco adecuado} & \textbf{Nada adecuado} \\ \midrule
\textit{El tipo y el tamaño de letra es} &X & & & \\ \midrule
\textit{Los iconos e imágenes usados son} &X & & & \\ \midrule
\textit{Los colores empleados son} &X & & & \\ \midrule
\rowcolor[rgb]{.949,  .949,  .949}\multicolumn{2}{p{19em}|}{\textbf{Diseño de la interfaz}} & \textbf{Sí} & \textbf{A veces} & \textbf{No} \\ \midrule
\multicolumn{2}{p{19em}|}{\textit{¿Le resulta fácil de usar?}} & X& & \\ \midrule
\multicolumn{2}{p{19em}|}{\textit{¿El diseño de las pantallas es claro y atractivo?}} &X & & \\ \midrule
\multicolumn{2}{p{19em}|}{\textit{¿Es coherente el diseño general del sitio web?}} &X & & \\ \midrule
\multicolumn{2}{p{19em}|}{\textit{¿Cree que el programa está bien estructurado?}} & X& & \\ \midrule
\rowcolor[rgb]{ .851,  .886,  .953}\multicolumn{5}{p{36em}}{\textbf{Observaciones}} \\ \midrule
\multicolumn{5}{p{36em}}{} \\ \bottomrule
\end{tabular}
\end{table}

\begin{longtable}{p{12em}p{24em}}
%\centering
\caption{Actividades guiadas (Usuario 4 - administrador)}\\
%\begin{tabular}{p{12em}p{24em}}
\hline \\[-1em] %\toprule
\rowcolor[rgb]{ .949,  .949,  .949}\multicolumn{2}{p{36em}}{\textbf{\textit{Añadir un periodo}}} \\ \hline  \\[-1em]%\midrule
\textbf{Tiempo empleado:} & 30 segundos \\ \hline  \\[-1em]%\midrule
\textbf{Problemas encontrados:} & Ninguno \\ \hline  \\[-1em]%\midrule
\textbf{Observaciones:} &  \\ \hline  \\[-1em]%\midrule
\rowcolor[rgb]{ .949,  .949,  .949}\multicolumn{2}{p{36em}}{\textbf{\textit{Añadir un componente}}} \\ \hline \\[-1em]%\midrule
\textbf{Tiempo empleado:} & 2 minutos \\ \hline \\[-1em]%\midrule
\textbf{Problemas encontrados:} & - Algunos valores introducidos no se pueden reconocer como inválidos. \par - El error que se muestra al guardar de forma incorrecta no especifica por qué ocurre.\\ \hline \\[-1em]%\midrule
\textbf{Observaciones:} & - Añadir información a los campos del formulario para especificar valores límite si existen.\par - Crear diferentes mensajes de error para dar más información. \\ \hline \\[-1em]%\midrule
\rowcolor[rgb]{ .949,  .949,  .949}\multicolumn{2}{p{36em}}{\textbf{\textit{Ver detalles de un periodo}}} \\\hline \\[-1em]% \midrule
\textbf{Tiempo empleado:} & Instantáneo \\ \hline \\[-1em]%\midrule
\textbf{Problemas encontrados:} & Ninguno \\ \hline \\[-1em]%\midrule
\textbf{Observaciones:} &  \\ \hline \\[-1em]%\midrule
\rowcolor[rgb]{ .949,  .949,  .949}\multicolumn{2}{p{36em}}{\textbf{\textit{Ver detalles de un componente}}} \\ \hline \\[-1em]%\midrule
\textbf{Tiempo empleado:} & Instantáneo \\ \hline \\[-1em]%\midrule
\textbf{Problemas encontrados:} & Ninguno \\ \hline \\[-1em]%\midrule
\textbf{Observaciones:} &  \\ \hline \\[-1em]%\midrule
\rowcolor[rgb]{ .949,  .949,  .949}\multicolumn{2}{p{36em}}{\textbf{\textit{Editar un periodo}}} \\ \hline \\[-1em]%\midrule
\textbf{Tiempo empleado:} & 30 segundos \\ \hline \\[-1em]%\midrule
\textbf{Problemas encontrados:} & Ninguno \\ \hline \\[-1em]%\midrule
\textbf{Observaciones:} &  \\ \hline \\[-1em]%\midrule
\rowcolor[rgb]{ .949,  .949,  .949}\multicolumn{2}{p{36em}}{\textbf{\textit{Editar un componente}}} \\ \hline \\[-1em]%\midrule
\textbf{Tiempo empleado:} & 2 minutos \\ \hline \\[-1em]%\midrule
\textbf{Problemas encontrados:} & Ninguno \\ \hline \\[-1em]%\midrule
\textbf{Observaciones:} &  \\ \hline \\[-1em]%\midrule
\rowcolor[rgb]{ .949,  .949,  .949}\multicolumn{2}{p{36em}}{\textbf{\textit{Eliminar un periodo}}} \\ \hline \\[-1em]%\midrule
\textbf{Tiempo empleado:} & Instantáneo \\ \hline \\[-1em]%\midrule
\textbf{Problemas encontrados:} & Ninguno \\ \hline \\[-1em]%\midrule
\textbf{Observaciones:} &  \\ \hline \\[-1em]%\midrule
\rowcolor[rgb]{ .949,  .949,  .949}\multicolumn{2}{p{36em}}{\textbf{\textit{Eliminar un componente}}} \\ \hline \\[-1em]%\midrule
\textbf{Tiempo empleado:} & Instantáneo \\ \hline \\[-1em]%\midrule
\textbf{Problemas encontrados:} & Ninguno \\ \hline \\[-1em]%\midrule
\textbf{Observaciones:} &  \\ \hline%\bottomrule
%\end{tabular}
\end{longtable}


Una vez terminadas estas pruebas, se realizaron los siguientes cambios para mejorar la aplicación web en base a los resultados:
\begin{itemize}
\item Se han modificado los enlaces para identificar aquellos con estado activo.
\item Se ha añadido el atributo \textit{title} y un borde sombreado a las imágenes enlace para distinguirlas de las imágenes que no lo son.
\item Se ha enfatizado el borde de las cajas que contienen cada periodo en la línea temporal del museo.
\item Se ha cambiado la apariencia de los enlaces a los componentes en la vista general del museo.
\item Se ha modificado el menú lateral en la vista del componente para dejar claro que son enlaces y no elementos desplegables.
\item Se ha cambiado la búsqueda por nombre para que los resultados se actualicen según se introduce el texto y que no sea necesario pulsar \textit{Enter}.
\item En los formularios de añadir y editar componente se ha añadido información sobre los valores límite existentes en los años de inicio y fin.
\item Se han añadido mensajes de error en la validación de componentes (campos sin completar, años incorrectos, valores numéricos en negativo).
\end{itemize}