\newpage
\hypersetup{pageanchor=true}

\newpage
\thispagestyle{empty}
\chapter{¿Qué es este trabajo?}
\section{Resumen}
Este proyecto consiste en el desarrollo del sitio web para el Museo de la Escuela de Ingeniería Informática de Oviedo, en el que se exponen antiguos componentes hardware, principalmente CPUs.
\par El usuario podrá navegar por los diferentes periodos históricos en los que se agrupan los componentes, conocer efemérides tecnológicas de la época y otras curiosidades. De cada pieza podrá ver características, sistemas famosos que la utilizaban, e imágenes tanto del componente como de dichos sistemas famosos.
\par Además, el administrador podrá añadir, editar y eliminar los periodos y componentes que se mostrarán en la web del museo.
\newpage
\section{Palabras Clave}
Museo, informática, sitio web, componentes, hardware, CPU, Oviedo, Escuela de Ingeniería Informática.
\pagestyle{fancy}
\newpage
\section{Abstract}
The aim of this project is to develop the Computer Museum's website for the Computer Science School, to exhibit old hardware, mainly CPUs.
\par The user will be able to visit the different historical periods  in which components are grouped, to know technological ephemerides of that time and other curiosities. For each piece, the user will also be able to see its characteristics, famous systems that used it and images of the piece and the famous systems.
\par In addition, the administrator will be able to add, update and delete the periods and components to be displayed in the museum's website.
\pagestyle{fancy}
\newpage
\section{Keywords}
Museum, Computer Science, website, components, hardware, CPU, Oviedo, School of Computer Science.