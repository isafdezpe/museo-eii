\newpage
\chapter{ESTUDIO DE VIABILIDAD DEL SISTEMA}
	\vspace{2cm}	
	\begin{center}
	{\Large \textbf{FASE DE DESARROLLO} \par}
	\end{center}
	\vspace{5cm}
	
	\begin{center}
	\Huge \textbf{EVS}\par
	\end{center}\newpage
\section{EVS 4, 5, 6: ESTUDIO Y VALORACIÓN DE ALTERNATIVAS DE SOLUCIÓN. SELECCIÓN DE ALTERNATIVA FINAL}

\subsection{Evaluación de alternativas de desarrollo} 
\subsubsection{JavaScript y Node.js}
JavaScript es uno de los lenguajes más populares actualmente. Está basado en el estándar ECMAScript. Se trata un lenguaje interpretado, se compila en tiempo de ejecución. Es orientado a objetos, débilmente tipado y dinámico\cite{JavaScript}. 
\par Node.js es un entorno de ejecución de JavaScript orientado a eventos asíncronos, en el que no hace falta ultilizar hilos. Utiliza un modelo de entrada y salida sin bloqueo, lo que asegura un rendimiento más eficiente de la aplicación y evita que se produzca una gran sobrecarga del lado del servidor. Por ello, es muy apropiado para desarrollar sistemas escalables\cite{NodeJS}. Además, puede ser utilizado tanto en el lado del cliente como en el servidor, por lo que no se necesitaría una tecnología adicional para el back-end.
\par Esta fue la primera opción barajada, ya que había utilizado anteriormente estas tecnologías y podría aprovechar este proyecto para profundizar en su aprendizaje.
\begin{figure}[H]
	\centering
	\begin{subfigure}{0.3\textwidth}
	\centering
	\includegraphics[scale=2.1]{javascript}
	\end{subfigure}
	\begin{subfigure}{0.3\textwidth}
	\centering
	\includegraphics[scale=0.5]{nodejs}
	\end{subfigure}
	\caption{Logos de JavaScript y Node.js}
\end{figure}

\subsubsection{Angular, TypeScript y PHP}
La otra opción considerada fue Angular y TypeScript, debido a su popularidad. No había trabajado con ellas antes, y esta sería una buena oportunidad para conocerlas.
\par Angular es un framework desarrollado en TypeScript y utilizado habitualmente para crear aplicaciones de una sola página. Se basa en la utilización de componentes web reutilizables para crear aplicaciones web fácilmente escalables. Angular extiende la sintaxis de HTML y actualiza automáticamente el árbol DOM cuando el estado de un componente cambia. Cuenta con gran cantidad de librerías y es uno de los frameworks más utilizados en la industria actual\cite{Angular}.
\par TypeScript es un lenguaje de programación que extiende JavaScript añadiendo la definición de tipos estáticos. Al compilarlo se transforma en código JavaScript siguiendo todos los estándares, y puede ejecutarse en cualquier lugar que ejecute JavaScript\cite{TypeScript}.
\par En este caso, Angular y TypeScript son ambas tecnologías de front-end, por tanto necesitamos una tercera tecnología para el back-end de esta aplicación. Para ello consideré como opción PHP, lenguaje que se ejecuta en el servidor y envía el resultado generado al cliente, y que es otra de las tecnologías más reconocidas y usadas en el desarrollo web actualmente y desde su creación.
\begin{figure}[H]
	\centering
	\begin{subfigure}{0.2\textwidth}
	\centering
	\includegraphics[scale=0.30]{angular}
	\end{subfigure}
	\begin{subfigure}{0.3\textwidth}
	\centering
	\includegraphics[scale=0.12]{typescript}
	\end{subfigure}
	\begin{subfigure}{0.3\textwidth}
	\centering
	\includegraphics[scale=0.05]{php}
	\end{subfigure}
	\caption {Logos de Angular, TypeScript y PHP}
\end{figure}
\noindent Ambas opciones son de código abierto, lo que me parece un punto positivo ya que, gracias a la colaboración de la comunidad, se consigue una alta calidad en el software. 
\par Finalmente, me decidí por Angular, TypeScript y PHP, principalmente por la razón de profundizar en el aprendizaje de estas tecnologías tan importantes actualmente en el desarrollo de aplicaciones web.

\subsection{Evaluación de alternativas de gestor de bases de datos} 
\subsubsection{MySQL}
MySQL es un SGBD relacional de código abierto con un modelo cliente-servidor. Ha sido la única opción considerada al tratarse de la base de datos relacional que es comúnmente utilizada con Angular, y no se ha encontrado ninguna necesidad o ventaja de usar un sistema no relacional.
\begin{figure}[H]
	\centering
	\includegraphics[scale=0.1]{mysql}
	\caption{Logo de MySQL}
\end{figure}