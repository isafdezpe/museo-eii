\newpage
\chapter{ESTUDIO DE VIABILIDAD DEL SISTEMA}
	\vspace{2cm}	
	\begin{center}
	{\Large \textbf{FASE DE DESARROLLO} \par}
	\end{center}
	\vspace{5cm}
	
	\begin{center}
	\Huge \textbf{EVS}\par
	\end{center}\newpage
\section{EVS 4, 5, 6: ESTUDIO Y VALORACIÓN DE ALTERNATIVAS DE SOLUCIÓN. SELECCIÓN DE ALTERNATIVA FINAL}

\subsection{Evaluación de alternativas de desarrollo} 
\subsubsection{Node.js y JavaScript}
Node.js es un entorno de ejecución de JavaScript orientado a eventos asíncronos, en el que no hace falta ultilizar hilos. Utiliza un modelo de entrada y salida sin bloqueo, lo que asegura un rendimiento más eficiente de la aplicación y evita que se produzca una gran sobrecarga del lado del servidor. Por ello, es muy apropiado para desarrollar sistemas escalables\cite{NodeJS}.\\
\par JavaScript es uno de los lenguajes más populares actualmente. Está basado en el estándar ECMAScript. Se trata un lenguaje interpretado, se compila en tiempo de ejecución. Es orientado a objetos, débilmente tipado y dinámico\cite{JavaScript}.\\
\par Esta fue la primera opción barajada, ya que había utilizado anteriormente estas tecnologías y podría aprovechar este proyecto para profundizar en su aprendizaje.
\begin{figure}[H]
	\begin{subfigure}{0.5\textwidth}
	\centering
	\includegraphics[scale=0.5]{nodejs}
	\end{subfigure}
	\begin{subfigure}{0.5\textwidth}
	\centering
	\includegraphics[scale=2.1]{javascript}
	\end{subfigure}
	\caption{Logos de Node.js y JavaScript}
\end{figure}

\subsubsection{Angular y TypeScript}
La otra opción considerada fue Angular con TypeScript, debido a su popularidad. No había trabajado con ellas antes, y esta sería una buena oportunidad para conocerlas.\\
\par Angular es un framework desarrollado en TypeScript y utilizado habitualmente para crear aplicaciones de una sola página. Se basa en la utilización de componentes web reutilizables para crear aplicaciones web fácilmente escalables. Angular extiende la sintaxis de HTML y actualiza automáticamente el árbol DOM cuando el estado de un componente cambia. Cuenta con gran cantidad de librerías y es uno de los frameworks más utilizados en la industria actual\cite{Angular}.\\
\par TypeScript es un lenguaje de programación que extiende JavaScript añadiendo la definición de tipos estáticos. Al compilarlo se transforma en código JavaScript siguiendo todos los estándares, y puede ejecutarse en cualquier lugar que ejecute JavaScript\cite{TypeScript}.
\begin{figure}[H]
	\begin{subfigure}{0.5\textwidth}
	\centering
	\includegraphics[scale=0.40]{angular}
	\end{subfigure}
	\begin{subfigure}{0.5\textwidth}
	\centering
	\includegraphics[scale=0.15]{typescript}
	\end{subfigure}
	\caption {Logos de Angular y TypeScript}
\end{figure}
Ambas opciones son de código abierto, lo que me parece un punto positivo ya que, gracias a la colaboración de la comunidad, se consigue una alta calidad en el software. \\
\par Finalmente, me decidí por Angular y TypeScript, principalmente por la razón de aprender estas dos tecnologías tan importantes actualmente en el desarrollo de aplicaciones web.
