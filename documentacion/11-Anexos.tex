\newpage
\chapter{ANEXOS}
%\addcontentsline{toc}{chapter}{ANEXOS}
\newpage
%\phantomsection
\section{PLAN DE GESTIÓN DE RIESGOS}
%\addcontentsline{toc}{section}{PLAN DE GESTIÓN DE RIESGOS}

\newpage
\section{CONTENIDO ENTREGADO EN LOS ANEXOS}\label{sec:contenido_anexos}
%\addcontentsline{toc}{section}{CONTENIDO ENTREGADO EN LOS ANEXOS}

\subsection{Contenidos} 


Además de este documento, se hace entrega de una carpeta comprimida ``.zip'' en la que ahora se describirán sus contenidos. Se estructurará también la organización del código fuente.
\begin{itemize}
	\item \textbf{Planificación\_TFG.mpp} \(\rightarrow\) Archivo de Microsoft Project que contiene la planificación del proyecto entera.
	\item \textbf{Presupuesto\_TFG.xlsx} \(\rightarrow\) Archivo Microsoft Excel que contiene los cálculos del presupuesto del proyecto.
	\item \textbf{Documentación\_Compodoc} \(\rightarrow\) Carpeta que contiene la documentación de los proyectos (museo y administración) generada con Compodoc.
	\begin{itemize}
		\item \textbf{Documentación\_Museo} \(\rightarrow\) Contiene la documentación del proyecto del museo (museo-eii).
		\item \textbf{Documentación\_Admin} \(\rightarrow\) Contiene la documentación del proyecto de la administración del museo (museo-eii-admin).
	\end{itemize}
 	Cada una de estas carpetas contiene los archivos HTML con la documentación generada. Abriendo el archivo \textit{index.html} de cada una se puede ver la documentación al completo de su respectivo proyecto.
	\item \textbf{Diagramas} \(\rightarrow\) Carpeta que contiene todos los diagramas utilizados en este documento.
	\begin{itemize}
		\item \textit{Diagrama\_arquitectura\_tecnologica.png}
		\item \textit{Diagrama\_casos\_uso\_museo.png}
		\item \textit{Diagrama\_casos\_uso\_admin.png}
		\item \textit{Diagrama\_clases\_museo-Analisis.png}
		\item \textit{Diagrama\_clases\_admin-Analisis.png}
		\item \textit{Diagrama\_navegabilidad\_museo.png}
		\item \textit{Diagrama\_navegabilidad\_admin.png}
		\item \textit{Diagrama\_clases\_museo-Diseño.png}
		\item \textit{Diagrama\_clases\_admin-Diseño.png}
		\item \textit{Diagrama\_paquetes.png}
		%\item \textit{Diagrama\_componentes.png}
		\item \textit{Diagrama\_despliegue.png}
		\item \textit{Diagrama\_E-R.png}
	\end{itemize}
	\item \textbf{TFG\_codigo.zip} \(\rightarrow\) Carpeta comprimida con todo el código fuente.
\end{itemize}

\textcolor[rgb]{0.65,0.16,0}{Ejemplo de como especificar los contenidos entregados}

Ahora se mostrará el contenido de dicha carpeta comprimida que contiene todo el código fuente de la aplicación la cual esta dividida a su vez en dos carpetas:

\paragraph*{AuthServerGuardMe}
Contiene el código que se aloja en \textit{Heroku} para darle funcionalidad al servidor. La clase principal es la llamada \texttt{mainAuthServer.js}.

\paragraph*{GuardMe}
Contiene el código fuente de la aplicación y se compone de las siguientes carpetas:
\begin{itemize}
	\item \textbf{assets} -> Carpeta que contiene los elementos gráficos usados en la aplicación. Se subdivide en una carpeta llamada \textit{images} que contiene todas las imagenes utilizadas para la construcción de la aplicación.
	\item \textbf{components} -> Carpeta que contiene el código para todos los componentes creados.
	\item \textbf{constants} -> Carpeta que contiene el código
	\item \textbf{docs} -> Carpeta que contiene los archivos html generados por JSDoc.
	\item \textbf{files} -> Carpeta en la que se encuentras los futuros archivos de Términos y Condiciones y Política de Privacidad entre otros.
	\item \textbf{modules\_LICENSES} -> Carpeta que contiene una por una todas las licencias de las librerías utilizadas en el desarrollo.
	\item \textbf{navigation} -> Carpeta que contiene las clases relativas a la navegación de la aplicación.
	\item \textbf{objects} -> Carpeta que contiene los objetos utilizados en el desarrollo que en este caso ha sido solo Fire.js.
	\item \textbf{screens} -> Carpeta que contiene todas las pantallas, agrupadas a su vez en subcarpetas que identifican la pantalla sobre la que están relacionadas.
	\item \textbf{styles} -> Carpeta que contiene todos los estilos de las pantallas, agrupadas a su vez en subcarpetas que siguen la misma estructura que \textit{screens}.
	\item \textbf{App.js} -> Clase principal y encargada de que comience la aplicación entera.
	\item \textbf{LICENSE} -> Licencia sobre el código fuente.
	\item \textbf{README.md} -> Archivo con la descripción del proyecto para la documentación y el repositorio de GitHub.
	\item \textbf{package.json} -> Archivo que contiene las librerías utilizadas en el proyecto.
	\item \textbf{app.json} -> Archivo que contiene la configuración de la aplicación.
	\item \textbf{configJSDoc.json} -> Archivo de configuración para la creación de documentación por parte de JSDoc.
	\item \textbf{Otros archivos} -> Los demás archivos no son relevantes ya que muchos se generan por defecto y los demás son configuraciones propias de expo.
\end{itemize}