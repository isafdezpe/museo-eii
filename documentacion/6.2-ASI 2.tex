\subsection{Obtención de los Requisitos del Sistema} 

\subsubsection{Requisitos de interfaces externas}

\paragraph*{Interfaces de usuario}
	
	\newlist{myEnumIU}{enumerate}{4}
	\setlist[myEnumIU,1]{label=\textbf{RIU-\arabic*.}}
	\setlist[myEnumIU,2]{label*=\textbf{\arabic*.}}
	\setlist[myEnumIU,3]{label*=\textbf{\arabic*.}}
	\setlist[myEnumIU,4]{label*=\textbf{\arabic*.}}

	\begin{myEnumIU}
		\item El sistema será accesible desde cualquier dispositivo que cuente con conexión a internet y un navegador web.
		\item El sistema estará disponible en diferentes idiomas.
		\begin{myEnumIU}
			\item Español
			\item Inglés
		\end{myEnumIU}
		\item El sistema deberá ser accesible para todos los usuarios a través de los navegadores más comunes.
		\begin{myEnumIU}
			\item Google Chrome
			\item Mozilla Firefox
			\item Microsoft Edge
		\end{myEnumIU}
		\item El usuario podrá utilizar todas las funcionalidades desarrolladas de la aplicación sin inconvenientes.
		\item El usuario no necesitará de conocimientos tecnológicos avanzados.
	\end{myEnumIU}

\paragraph*{Interfaces hardware}
	
	\newlist{myEnumIH}{enumerate}{4}
	\setlist[myEnumIH,1]{label=\textbf{RIH-\arabic*.}}
	\setlist[myEnumIH,2]{label*=\textbf{\arabic*.}}
	\setlist[myEnumIH,3]{label*=\textbf{\arabic*.}}
	\setlist[myEnumIH,4]{label*=\textbf{\arabic*.}}

	\begin{myEnumIH}
		\item El sistema dispondrá de una base de datos para almacenar la información necesaria.
	\end{myEnumIH}

%\paragraph*{Interfaces software}

\paragraph*{Interfaces de comunicaciones}

	\newlist{myEnumIC}{enumerate}{4}
	\setlist[myEnumIC,1]{label=\textbf{RIC-\arabic*.}}
	\setlist[myEnumIC,2]{label*=\textbf{\arabic*.}}
	\setlist[myEnumIC,3]{label*=\textbf{\arabic*.}}
	\setlist[myEnumIC,4]{label*=\textbf{\arabic*.}}

	\begin{myEnumIC}
		\item El sistema contendrá enlaces a diferentes sitios web.
		\item El sistema mostrará por defecto enlaces a los siguientes sitios web.
		\begin{myEnumIC}
			\item Twitter oficial de la Escuela de Ingeniería Informática
			\item Página web de la Escuela de Ingeniería Informática
			\item Página web de la Universidad de Oviedo
		\end{myEnumIC}
	\end{myEnumIC}



\subsubsection{Requisitos funcionales}

\newlist{myEnumerate}{enumerate}{9}
\setlist[myEnumerate,1]{label=\textbf{RF-\arabic*.}}
\setlist[myEnumerate,2]{label*=\textbf{\arabic*.}}
\setlist[myEnumerate,3]{label*=\textbf{\arabic*.}}
\setlist[myEnumerate,4]{label*=\textbf{\arabic*.}}
\setlist[myEnumerate,5]{label*=\textbf{\arabic*.}}
\setlist[myEnumerate,6]{label*=\textbf{\arabic*.}}
\setlist[myEnumerate,7]{label*=\textbf{\arabic*.}}
\setlist[myEnumerate,8]{label*=\textbf{\arabic*.}}
\setlist[myEnumerate,9]{label*=\textbf{\arabic*.}}

\begin{myEnumerate}
\item
\end{myEnumerate}


\subsubsection{Requisitos de rendimiento}
\subsubsection{Requisitos lógicos de BD}
\subsubsection{Requisitos de desarrollo}
\subsubsection{Restricciones de diseño}
\subsubsection{Atributos del sistema}


\subsection{Identificación de Actores del Sistema} 
\subsubsection{Usuario administrador}
Actor que interactúa con el sistema. Es responsable de gestionar el sistema y su mantenimiento. Es el único actor con acceso a la base de datos del sistema y capacidad de modificarla. Debe tener amplios conocimientos sobre el sistema.
\subsubsection{Usuario estándar}
Actor que interactúa con el sistema. Tiene acceso de lectura a toda la aplicación web, exceptuando la parte dedicada al mantenimiento. Solo debe tener un conocimiento básico para navegar por internet.

\subsection{Especificación de Casos de Uso}

\textcolor[rgb]{0.65,0.16,0}{Ejemplo de tabla para especificación de casos de uso}

\begin{table}[htbp]
  \centering
  \caption{Especificación Caso de Uso 1}
    \begin{tabular}{p{20.855em}r}
\cmidrule{1-1}    \rowcolor[rgb]{ .949,  .949,  .949} \multicolumn{1}{p{20.855em}}{\textbf{Nombre del caso de uso}} & \multicolumn{1}{r}{\cellcolor[rgb]{ 1,  1,  1}} \\
\cmidrule{1-1}    \multicolumn{1}{p{20.855em}}{Registro} & \multicolumn{1}{r}{} \\
    \midrule
    \rowcolor[rgb]{ .949,  .949,  .949} \multicolumn{2}{p{31.64em}}{\textbf{Descripción}} \\
    \midrule
    \multicolumn{2}{p{31.64em}}{Un usuario no registrado debe poder registrarse en el sistema mediante su cuenta de Google, lo que hará que automáticamente se inicie sesión en la aplicación.} \\
    \bottomrule
    \end{tabular}%
  \label{espec_caso_uso_1}%
  \vspace{-4mm}
\end{table}%