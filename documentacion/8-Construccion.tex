\newpage
\chapter{CONSTRUCCIÓN DEL SISTEMA DE INFORMACIÓN}
	\vspace{2cm}	
	\begin{center}
	{\Large \textbf{FASE DE DESARROLLO} \par}
	\end{center}
	\vspace{5cm}
	
	\begin{center}
	\Huge \textbf{CSI}\par
	\end{center}

\newpage


\section{CSI 1: PREPARACIÓN DEL ENTORNO DE GENERACIÓN Y CONSTRUCCIÓN}

\subsection{Estándares y normas seguidos}


\subsection{Lenguajes de programación}


\subsection{Herramientas y programas usados para el desarrollo}


\newpage
\section{CSI 2: GENERACIÓN DEL CÓDIGO DE LOS COMPONENTES Y PROCEDIMIENTOS}

\textcolor[rgb]{0.65,0.16,0}{Ejemplos de tablas descripción de clases}

\begin{table}[H]
\vspace{-4mm}
  \centering
  \caption{Descripción de diseño de LoginScreen}
    \begin{tabular}{p{8.645em}rr}
    \toprule
    \rowcolor[rgb]{ .851,  .886,  .953} \multicolumn{3}{p{31.285em}}{\textbf{LoginScreen}} \\
    \midrule
    \rowcolor[rgb]{ .949,  .949,  .949} \multicolumn{3}{p{31.285em}}{\textbf{Descripción}} \\
    \midrule
    \multicolumn{3}{p{31.285em}}{Es la encargada de las acciones y la renderización de la pantalla de inicio de sesión.} \\
    \midrule
    \rowcolor[rgb]{ .906,  .902,  .902} \multicolumn{3}{p{31.285em}}{\textbf{Atributos propuestos}} \\
    \midrule
    \multicolumn{3}{p{31.285em}}{-} \\
    \midrule
    \rowcolor[rgb]{ .906,  .902,  .902} \multicolumn{3}{p{31.285em}}{\textbf{Métodos propuestos}} \\
    \midrule
    \textbf{signInWithGoogle} & \multicolumn{2}{p{22.64em}}{Hace una llamada al objeto Fire para el inicio de sesión con Firebase authentication mediante una cuenta de Google.} \\
    \midrule
    \textbf{render} & \multicolumn{2}{r}{} \\
    \bottomrule
    \end{tabular}%
\end{table}%


\begin{table}[htbp]
  \centering
  \caption{Descripción de diseño de HomeScreen}
    \begin{tabular}{p{10em}rr}
    \toprule
    \rowcolor[rgb]{ .851,  .886,  .953} \multicolumn{3}{p{31.285em}}{\textbf{HomeScreen}} \\
    \midrule
    \rowcolor[rgb]{ .949,  .949,  .949} \multicolumn{3}{p{31.285em}}{\textbf{Descripción}} \\
    \midrule
    \multicolumn{3}{p{31.285em}}{Es la encargada de las acciones y la renderización de la pantalla de emergencia.} \\
    \midrule
    \rowcolor[rgb]{ .906,  .902,  .902} \multicolumn{3}{p{31.285em}}{\textbf{Atributos propuestos}} \\
    \midrule
    \multicolumn{3}{p{31.285em}}{-} \\
    \midrule
    \rowcolor[rgb]{ .906,  .902,  .902} \multicolumn{3}{p{31.285em}}{\textbf{Métodos propuestos}} \\
    \midrule
    \textbf{componentWillMount} & \multicolumn{2}{r}{} \\
    \midrule
    \textbf{emergencyCalling} & \multicolumn{2}{p{21.285em}}{Es el método encargado de redirigir la aplicación hacia el marcador con el 112 marcado.} \\
    \midrule
    \textbf{warnProtectors} & \multicolumn{2}{p{21.285em}}{[Falta implementar] Es el encargado de generar un mensaje de aviso a los protectores creando notificaciones push.} \\
    \midrule
    \textbf{render} & \multicolumn{2}{r}{} \\
    \bottomrule
    \end{tabular}%
\end{table}%

\newpage
\section{CSI 3: EJECUCIÓN DE LAS PRUEBAS UNITARIAS}


\newpage
\section{CSI 4: EJECUCIÓN DE LAS PRUEBAS DE INTEGRACIÓN}


\newpage
\section{CSI 5: EJECUCIÓN DE LAS PRUEBAS DEL SISTEMA}

\subsection{Prueba de Usabilidad}

\subsection{Pruebas de Accesibilidad} 
 
\subsubsection{Revisión Preliminar} 

\subsubsection{Evaluación de Conformidad} 

\subsubsection{Checklist del WCAG 2.1} 

\subsubsection{Accesibilidad con Dispositivos Móviles} 


\newpage
\section{CSI 6: ELABORACIÓN DE LOS MANUALES DE USUARIO}

\subsection{Manual de Instalación} 

\subsection{Manual de Ejecución} 

\subsection{Manual de Usuario} 

\subsection{Manual del Programador}


\newpage
\section{CSI 8: CONSTRUCCIÓN DE LOS COMPONENTES Y PROCEDIMIENTOS DE MIGRACIÓN Y CARGA INICIAL DE DATOS}


