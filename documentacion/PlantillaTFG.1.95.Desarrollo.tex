\documentclass[11pt]{report}
%\usepackage[margin=0.75in]{geometry}
\usepackage{geometry}
\usepackage[utf8]{inputenc}
\usepackage[greek, spanish, es-tabla]{babel}
\usepackage[bottom]{footmisc}
\usepackage[tt]{titlepic}
\usepackage{url}
\usepackage{xurl}
\usepackage{graphicx}
\usepackage{makeidx}
\usepackage{enumerate}
\usepackage{fancyhdr, ragged2e}
\usepackage{eurosym}
\usepackage{float}
\usepackage{titlesec}
\usepackage[bookmarks,hidelinks]{hyperref}
\usepackage{nameref}
\usepackage{enumitem}
\usepackage{booktabs}
\usepackage[table]{xcolor}
\usepackage{multirow}
\usepackage{amsmath}
\usepackage{listings}
\usepackage{subcaption}

	%\addtolength{\oddsidemargin}{-.8in}
	%\addtolength{\evensidemargin}{-.8in}
	%\addtolength{\textwidth}{1.45in}

	\addtolength{\topmargin}{-.1in}
	\addtolength{\textheight}{1.25in}

\graphicspath{{figures/}}
\pagestyle{fancy}
\renewcommand{\chaptermark}[1]{\markboth{\scriptsize\MakeUppercase{#1}}{}}
\renewcommand{\sectionmark}[1]{\markright{\tiny\MakeUppercase{#1}}{}}
\fancyfoot{}
%\fancyfoot[RO, LE] {\thepage}
\fancyfoot[R] {\thepage}
%\fancyfoot[LO, RE] {\scriptsize Escuela de Ingeniería Informática - Universidad de Oviedo. Mª Isabel Fernández Pérez}
\fancyfoot[L] {\scriptsize Escuela de Ingeniería Informática - Universidad de Oviedo. Mª Isabel Fernández Pérez}

\renewcommand{\footrulewidth}{0.4pt}
%Evita warnings de cabecera
\setlength{\headheight}{15pt}
\setcounter{secnumdepth}{4}

%Esto es para poder ponerle un formato bien hecho a los \paragraph{}
\titleformat{\paragraph}
{\normalfont\normalsize\bfseries}{\theparagraph}{1em}{}
\titlespacing*{\paragraph}
{0pt}{3.25ex plus 1ex minus .2ex}{1.5ex plus .2ex}

%Y esto para para ponerselo a los \subparagraph{}
\titleformat{\subparagraph}
{\normalfont\normalsize\bfseries}{\theparagraph}{1em}{}
\titlespacing*{\subparagraph}
{0pt}{1ex plus 1ex minus .2ex}{1.5ex plus .2ex}


\makeindex

\begin{document}
\selectlanguage{spanish} 

\hypersetup{pageanchor=false}
\begin{titlepage}
	\centering
	\includegraphics[width=0.2\textwidth]{EscudoUniovi}
	\hspace{3 cm}
	\includegraphics[width=0.3\textwidth]{EscudoEscuela}
	\par\vspace{1cm}
	
	\vspace{1.5cm}
	{\huge\bfseries Creación del sitio web para el Museo de la Informática de la Escuela de Ingeniería Informática de Oviedo\par}
	\vspace{2cm}
	{\large \textbf{GRADO EN INGENIERÍA INFORMÁTICA DEL SOFTWARE} \par}
	\vspace{1cm}
	{\scshape\Large Trabajo Fin de Grado\par}
	%\vfill
   	
  \vspace{2cm}
	%\vfill
	\textbf{AUTOR}\par
	 Mª Isabel Fernández Pérez \\
	%\vfill
	\vspace{1.5cm}
	%{\Large\itshape Mª Isabel Fernández Pérez\par}
	%\vfill
	\textbf{TUTOR}\par
	José Manuel Redondo López
	\vfill
	
	{\large Julio 2021 \par}
\end{titlepage}


\newpage
Copyright (C) 2020 \textbf{ELENA ALLEGUE GONZÁLEZ, JOSÉ MANUEL REDONDO LÓPEZ} \\
Teaching Innovation Project: PINN-19-A-029 (University of Oviedo)\\
This work has been published in \cite{RedondoPlantillasRG19} \cite{RedondoUCO20}\\
\\
Esta versión de la plantilla para Trabajos de Fin de Grado ha sido posible gracias a la donación de la ex-alumna Elena Allegue González de su documentación de Trabajo de Fin de Grado, que ha servido como base para elaborar esta versión. Aquí podréis encontrar todos los títulos y subtítulos de las secciones, pero las explicaciones se mantendrán en la versión \textit{Word} de la plantilla (se proporciona una versión PDF de la misma para facilitar el acceso a las mismas). No obstante, del trabajo de Elena se han conservado ejemplos de como hacer elementos clave como imágenes, tablas, etc.

Desarrollar una versión \textit{Latex} de la plantilla desde cero es una trabajo bastante largo, pero gracias al trabajo de Elena se ha podido equiparar esta versión con las de \textit{Word} mucho más rápidamente.

%\newpage
\thispagestyle{empty}
\chapter*{Agradecimientos}
Para empezar doy las gracias por haber llegado hasta aquí, ya que hubo momentos a lo largo de la carrera en los que pensé que no lo conseguiría.\\
\par Quiero agradecer a mis padres por el esfuerzo que han hecho para que pudiera estudiar fuera de casa estos cinco años, y por haber confíado siempre en mí, apoyándome aunque las circunstancias no fueran las mejores. También a mis abuelos por su gran ayuda.\\
\par Por supuesto, también quiero agradecer a mis amigos, por acompañarme todo este tiempo y hacerlo todo más llevadero. Especialmente le doy las gracias a Javi, que además de ser un gran amigo desde que empezamos la carrera, siempre estuvo ahí para ayudarme con cualquier cosa, incluso cuando apenas nos conocíamos. También a Cris, ya que conviviendo con ella y con Javi, esta época pandémica ha sido un poco mejor.\\
\par Por último quiero agradecer a Redondo, que ya con su forma de dar las clases de TPP me animó a no abandonar la carrera en segundo curso, y ha sido un tutor estupendo para un proyecto que me interesó desde el primer momento y que me ha gustado mucho desarrollar.

\pagestyle{fancy}
\renewcommand{\chaptermark}[1]{\markboth{\scriptsize\MakeUppercase{#1}}{}}
\renewcommand{\sectionmark}[1]{\markright{\tiny\MakeUppercase{#1}}{}}

%\fancyhead{}
%\lhead{\parbox[t]{0.5\textwidth}{\RaggedRight\rightmark\strut}}
%\rhead{\parbox[t]{0.5\textwidth}{\RaggedLeft\leftmark\strut}}
%\setlength{\headheight}{5\baselineskip}

\fancyfoot{}
%\fancyfoot[RO, LE] {\thepage}
\fancyfoot[R] {\thepage}
%\fancyfoot[LO, RE] {\scriptsize Escuela de Ingeniería Informática - Universidad de Oviedo. Mª Isabel Fernández Pérez}
\fancyfoot[L] {\scriptsize Escuela de Ingeniería Informática - Universidad de Oviedo.  Mª Isabel Fernández Pérez}
\renewcommand{\footrulewidth}{0.4pt}
\pagenumbering{arabic}
%\newpage


\thispagestyle{empty}


\setcounter{tocdepth}{2}
\setcounter{secnumdepth}{4}
\pagestyle{empty}
{
  \renewcommand{\thispagestyle}[1]{}
  \tableofcontents
}
\clearpage


\newpage
{
  \renewcommand{\thispagestyle}[1]{}
  \listoffigures
}
\clearpage

\newpage
{
  \renewcommand{\thispagestyle}[1]{}
  \listoftables
}
\clearpage


\newpage
\hypersetup{pageanchor=true}

\newpage
\thispagestyle{empty}
\chapter{¿Qué es este trabajo?}
\section{Resumen}
Este proyecto consiste en el desarrollo del sitio web para el Museo de la Escuela de Ingeniería Informática de Oviedo, en el que se exponen antiguos componentes hardware, principalmente CPUs.
\par El usuario podrá navegar por los diferentes periodos históricos en los que se agrupan los componentes, conocer efemérides tecnológicas de la época y otras curiosidades. De cada pieza podrá ver características, sistemas famosos que la utilizaban, e imágenes tanto del componente como de dichos sistemas famosos.
\par Además, el administrador podrá añadir, editar y eliminar los periodos y componentes que se mostrarán en la web del museo.
\newpage
\section{Palabras Clave}
Museo, informática, sitio web, componentes, hardware, CPU, Oviedo, Escuela de Ingeniería Informática.
\pagestyle{fancy}
\newpage
\section{Abstract}
The aim of this project is to develop the Computer Museum's website for the Computer Science School, to exhibit old hardware, mainly CPUs.
\par The user will be able to visit the different historical periods  in which components are grouped, to know technological ephemerides of that time and other curiosities. For each piece, the user will also be able to see its characteristics, famous systems that used it and images of the piece and the famous systems.
\par In addition, the administrator will be able to add, update and delete the periods and components to be displayed in the museum's website.
\pagestyle{fancy}
\newpage
\section{Keywords}
Museum, Computer Science, website, components, hardware, CPU, Oviedo, School of Computer Science.
\pagestyle{fancy}
%\newpage
\chapter{PSI: PLANIFICACIÓN DEL SISTEMA DE INFORMACIÓN}

	\vspace{2cm}	
	\begin{center}
	{\Large \textbf{FASE DE PLANIFICACIÓN} \par}
	\end{center}
	\vspace{5cm}
	
	\begin{center}
	\Huge \textbf{PSI}\par
	\end{center}

\newpage

\section{PSI 1: INICIO DEL PLAN DE SISTEMAS DE INFORMACIÓN}

\subsection{PSI 1.1: Análisis de la Necesidad del PSI} 
El tutor de este trabajo de fin de grado, José Manuel Redondo, ha propuesto el desarrollo de una aplicación web para el Museo de la Informática de Asturias, que contenga toda la información disponible sobre los componentes del museo y la muestre de forma ordenada para que las personas interesadas puedan acceder a ella fácilmente. El sistema será gestionado directamente por el tutor del trabajo.\\
\par El sistema debe identificar cada componente, mostrar la información disponible del mismo, e indicar la localización física de cada uno para ofrecer la posibilidad de visitar presencialmente el lugar de exposición de este. El software permitirá añadir la información de las nuevas piezas que puedan ser incluidas en la exposición en un futuro gracias a donaciones o compras. Los componentes serán ordenados según su tipo y la época a la que pertenecen. Además, el sistema automatizará la creación de los carteles informativos que acompañan a los componentes en la exposición física del Museo. 

\newpage
\subsection{PSI 1.2: Identificación del Alcance del PSI}
Actualmente las piezas del museo se encuentran expuestas en la Escuela de Ingeniería Informática, acompañadas de los carteles informativos correspondientes. Los objetivos de este proyecto son los siguientes:
\begin{itemize}
	\item Recopilar los datos disponibles de las piezas que se encuentran actualmente en el Museo e introducirlos en una base de datos.	
	\item Mostrar una linea temporal con los diferentes periodos a los que pertenecen los componentes del Museo. 
	\item Permitir acceder a cada periodo para ver los componentes del mismo.
	\item Organizar las diferentes piezas en función de su tipo y de la familia de la que forman parte.
	\item Presentar la información disponible de cada pieza, así como imágenes de la misma y otras curiosidades.
	\item Automatizar la creación de los carteles, utilizando plantillas predefinidas que se rellenarían con la información y las fotografías disponibles de la familia de piezas pertinente, que hasta el momento se han realizado de forma manual con Microsoft Publisher.  De este modo, se facilitará la exposición de nuevas piezas, ya que el esfuerzo de crear cada cartel informativo se reducirá de forma considerable. Los carteles están organizados por familia de piezas, más concretamente por familias de CPU.
\end{itemize}
En definitiva, estos objetivos se pueden resumir en:
\begin{itemize}
	\item Permitir a los usuarios visitar el Museo de la Informática de forma online, ofreciendo la misma información que se encuentra disponible en la exposición física.
	\item Facilitar el proceso de exposición de nuevas piezas gracias a la creación automática de la cartelería.
\end{itemize}

\newpage
\subsection{PSI 1.3: Determinación de Responsables}
\begin{itemize}
	\item \textbf{El proyectante} se encargará del desarrollo del software descrito y de realizar la carga de los datos disponibles a la base de datos correspondientes.
	\item\textbf{El tutor del proyecto} se encargará de la supervisión de las fases del proyecto y de su validación.
	\item \textbf{Una serie de usuarios escogidos aleatoriamente} realizará pruebas del software para comprobar su correcto funcionamiento.
\end{itemize}

\newpage
\section{PSI 2: DEFINICIÓN Y ORGANIZACIÓN DEL PSI}
 

\subsection{PSI 2.1: Especificación del Ámbito y Alcance} 


\subsection{PSI 2.2: Organización del PSI}



\newpage
\section{PSI 3: ESTUDIO DE LA INFORMACIÓN RELEVANTE}
 
\subsection{PSI 3.1: Selección y Análisis de Antecedentes} 
%\newpage
\chapter{PSI 7: DEFINICIÓN DE LA ARQUITECTURA TECNOLÓGICA}
	\vspace{2cm}	
	\begin{center}
	{\Large \textbf{FASE DE PLANIFICACIÓN} \par}
	\end{center}
	\vspace{5cm}
	
	\begin{center}
	\Huge \textbf{PSI}\par
	\end{center}

\newpage

\section{PSI 7.1: Identificación de las Necesidades de Infraestructura Tecnológica} 

\newpage
\section{PSI 7.2: Selección de la Arquitectura Tecnológica} 


\newpage
\chapter{ESTUDIO DE VIABILIDAD DEL SISTEMA}
	\vspace{2cm}	
	\begin{center}
	{\Large \textbf{FASE DE DESARROLLO} \par}
	\end{center}
	\vspace{5cm}
	
	\begin{center}
	\Huge \textbf{EVS}\par
	\end{center}\newpage
\section{EVS 4, 5, 6: ESTUDIO Y VALORACIÓN DE ALTERNATIVAS DE SOLUCIÓN. SELECCIÓN DE ALTERNATIVA FINAL}

\subsection{Evaluación de alternativas de desarrollo} 
\subsubsection{Node.js y JavaScript}
Node.js es un entorno de ejecución de JavaScript orientado a eventos asíncronos, en el que no hace falta ultilizar hilos. Utiliza un modelo de entrada y salida sin bloqueo, lo que asegura un rendimiento más eficiente de la aplicación y evita que se produzca una gran sobrecarga del lado del servidor. Por ello, es muy apropiado para desarrollar sistemas escalables\cite{NodeJS}.\\
\par JavaScript es uno de los lenguajes más populares actualmente. Está basado en el estándar ECMAScript. Se trata un lenguaje interpretado, se compila en tiempo de ejecución. Es orientado a objetos, débilmente tipado y dinámico\cite{JavaScript}.\\
\par Esta fue la primera opción barajada, ya que había utilizado anteriormente estas tecnologías y podría aprovechar este proyecto para profundizar en su aprendizaje.
\begin{figure}[H]
	\begin{subfigure}{0.5\textwidth}
	\centering
	\includegraphics[scale=0.5]{nodejs}
	\end{subfigure}
	\begin{subfigure}{0.5\textwidth}
	\centering
	\includegraphics[scale=2.1]{javascript}
	\end{subfigure}
	\caption{Logos de Node.js y JavaScript}
\end{figure}

\subsubsection{Angular y TypeScript}
La otra opción considerada fue Angular con TypeScript, debido a su popularidad. No había trabajado con ellas antes, y esta sería una buena oportunidad para conocerlas.\\
\par Angular es un framework desarrollado en TypeScript y utilizado habitualmente para crear aplicaciones de una sola página. Se basa en la utilización de componentes web reutilizables para crear aplicaciones web fácilmente escalables. Angular extiende la sintaxis de HTML y actualiza automáticamente el árbol DOM cuando el estado de un componente cambia. Cuenta con gran cantidad de librerías y es uno de los frameworks más utilizados en la industria actual\cite{Angular}.\\
\par TypeScript es un lenguaje de programación que extiende JavaScript añadiendo la definición de tipos estáticos. Al compilarlo se transforma en código JavaScript siguiendo todos los estándares, y puede ejecutarse en cualquier lugar que ejecute JavaScript\cite{TypeScript}.
\begin{figure}[H]
	\begin{subfigure}{0.5\textwidth}
	\centering
	\includegraphics[scale=0.40]{angular}
	\end{subfigure}
	\begin{subfigure}{0.5\textwidth}
	\centering
	\includegraphics[scale=0.15]{typescript}
	\end{subfigure}
	\caption {Logos de Angular y TypeScript}
\end{figure}
Ambas opciones son de código abierto, lo que me parece un punto positivo ya que, gracias a la colaboración de la comunidad, se consigue una alta calidad en el software. \\
\par Finalmente, me decidí por Angular y TypeScript, principalmente por la razón de aprender estas dos tecnologías tan importantes actualmente en el desarrollo de aplicaciones web.

\newpage
\chapter{PLANIFICACIÓN Y GESTIÓN DEL TFG}
\newpage

\newpage
\section{PLANIFICACIÓN DEL PROYECTO}

\subsection{Identificación de Interesados}
\begin{itemize}
\item \textbf{Usuarios de la aplicación web del museo}: Personas interesadas en la información ofrecida de las piezas expuestas en el museo que consultarán la página web.
\item \textbf{Tutor de este trabajo y administrador del museo}: Persona que ha propuesto el proyecto de digitalización del museo, y que lo administrará una vez terminado el desarrollo.
\item \textbf{Autora del trabajo y desarrolladora}: Encargada del desarrollo de la aplicación y de la documentación asociada.
\end{itemize}


\subsection{%OBS y 
PBS}
\begin{figure}[H]
\centering
\centerline{\includegraphics[scale=0.25]{PBS}}
\caption{Product Breakdown Structure}
\end{figure}


\subsection{Planificación Inicial. WBS}


\subsection{Riesgos}

\subsubsection{Plan de Gestión de Riesgos} 

\subsubsection{Identificación de Riesgos}

\subsubsection{Registro de Riesgos} 



\subsection{Presupuesto Inicial}

\subsubsection{Presupuesto de Costes}

\subsubsection{Presupuesto de Cliente} 


\newpage
\section{EJECUCIÓN DEL PROYECTO}

\subsection{Plan Seguimiento de Planificación}

\subsection{Bitácora de Incidencias del Proyecto}

\subsection{Riesgos}


\newpage
\section{CIERRE DEL PROYECTO}

\subsection{Planificación Final}

\subsection{Informe Final de Riesgos}

\subsection{Presupuesto Final de Costes}

\subsection{Informe de Lecciones Aprendidas}

\newpage
\chapter{ANÁLISIS DEL SISTEMA DE INFORMACIÓN}
	\vspace{2cm}	
	\begin{center}
	{\Large \textbf{FASE DE DESARROLLO} \par}
	\end{center}
	\vspace{5cm}
	
	\begin{center}
	\Huge \textbf{ASI}\par
	\end{center}

%\newpage
%\section{ASI 1: DEFINICIÓN DEL SISTEMA}

%\subsection{Determinación del Alcance del Sistema}



\newpage
\section{ASI 2: ESTABLECIMIENTO DE REQUISITOS}
\subsection{Obtención de los Requisitos del Sistema} 

\subsubsection{Requisitos de interfaces externas}

\paragraph*{Interfaces de usuario}
\newlist{myEnumIU}{enumerate}{4}
\setlist[myEnumIU,1]{label=\textbf{RIE-IU-\arabic*.}}
\setlist[myEnumIU,2]{label*=\textbf{\arabic*.}}
\setlist[myEnumIU,3]{label*=\textbf{\arabic*.}}
\setlist[myEnumIU,4]{label*=\textbf{\arabic*.}}
\begin{myEnumIU}
	\item El sistema será accesible desde cualquier dispositivo que cuente con conexión a internet y un navegador web.
	\item El sistema estará disponible en diferentes idiomas.
	\begin{myEnumIU}
		\item Español
		\item Inglés
	\end{myEnumIU}
	\item El sistema deberá ser accesible para todos los usuarios a través de los navegadores más comunes.
	\begin{myEnumIU}
		\item Google Chrome
		\item Mozilla Firefox
		\item Microsoft Edge
	\end{myEnumIU}
	\item El usuario podrá utilizar todas las funcionalidades desarrolladas de la aplicación sin inconvenientes.
	\item El usuario no necesitará de conocimientos tecnológicos avanzados.
\end{myEnumIU}

\paragraph*{Interfaces hardware}
\newlist{myEnumIH}{enumerate}{4}
\setlist[myEnumIH,1]{label=\textbf{RIE-IH-\arabic*.}}
\setlist[myEnumIH,2]{label*=\textbf{\arabic*.}}
\setlist[myEnumIH,3]{label*=\textbf{\arabic*.}}
\setlist[myEnumIH,4]{label*=\textbf{\arabic*.}}
\begin{myEnumIH}
	\item El sistema dispondrá de una base de datos para almacenar la información necesaria.
\end{myEnumIH}

%\paragraph*{Interfaces software}

\paragraph*{Interfaces de comunicaciones}
\newlist{myEnumIC}{enumerate}{4}
\setlist[myEnumIC,1]{label=\textbf{RIE-IC-\arabic*.}}
\setlist[myEnumIC,2]{label*=\textbf{\arabic*.}}
\setlist[myEnumIC,3]{label*=\textbf{\arabic*.}}
\setlist[myEnumIC,4]{label*=\textbf{\arabic*.}}
\begin{myEnumIC}
	\item El sistema contendrá enlaces a diferentes sitios web.
	\item El sistema mostrará por defecto enlaces a los siguientes sitios web.
	\begin{myEnumIC}
		\item Twitter oficial de la Escuela de Ingeniería Informática
		\item Página web de la Escuela de Ingeniería Informática
		\item Página web de la Universidad de Oviedo
	\end{myEnumIC}
\end{myEnumIC}


\subsubsection{Requisitos funcionales}
\newlist{myEnumRF}{enumerate}{2}
\setlist[myEnumRF,1]{label=\textbf{RF-\arabic*.}}
\setlist[myEnumRF,2]{label*=\textbf{\arabic*.}}
\begin{myEnumRF}
	\item El sistema estará constituido por dos aplicaciones web diferentes.
	\begin{myEnumRF}
		\item El museo.
		\item La administración del museo.
	\end{myEnumRF}
\end{myEnumRF}

\paragraph*{Museo}
\newlist{myEnumRFM}{enumerate}{9}
\setlist[myEnumRFM,1]{label=\textbf{RF-MU-\arabic*.}}
\setlist[myEnumRFM,2]{label*=\textbf{\arabic*.}}
\setlist[myEnumRFM,3]{label*=\textbf{\arabic*.}}
\setlist[myEnumRFM,4]{label*=\textbf{\arabic*.}}
\setlist[myEnumRFM,5]{label*=\textbf{\arabic*.}}
\setlist[myEnumRFM,6]{label*=\textbf{\arabic*.}}
\setlist[myEnumRFM,7]{label*=\textbf{\arabic*.}}
\setlist[myEnumRFM,8]{label*=\textbf{\arabic*.}}
\setlist[myEnumRFM,9]{label*=\textbf{\arabic*.}}
\begin{myEnumRFM}
	\item El sistema mostrará los periodos existentes.
	\begin{myEnumRFM}
		\item Los periodos estarán ordenados por año de inicio.
		\item Se presentarán en una línea temporal incluyendo los siguientes datos.
		\begin{myEnumRFM}
			\item Nombre.
			\item Año de inicio.
			\item Año de fin.
			\item Nombres de los componentes pertenecientes al periodo.
		\end{myEnumRFM}
	\end{myEnumRFM}
	\item El sistema permitirá realizar búsquedas.
	\begin{myEnumRFM}
		\item Se podrá buscar por los siguientes campos.
		\begin{myEnumRFM}
			\item Por nombre.
			\item Por un intervalo de años.
		\end{myEnumRFM}
		\item Al realizar la búsqueda se mostrará la línea temporal filtrada con los resultados.
	\end{myEnumRFM}
	\item\label{it:detalles_periodo} El sistema mostrará los detalles de un periodo.
	\begin{myEnumRFM}
		\item Nombre.
		\item Características.
		\item Listado de datos curiosos (Sabías qué...).
		\item Eventos ocurridos durante el periodo.
		\item Sistemas famosos que contienen componentes de este periodo.
		\item Listado de los componentes pertenecientes al periodo.
	\end{myEnumRFM}
	\item\label{it:detalles_comp} El sistema mostrará los detalles de un componente.
	\begin{myEnumRFM}
		\item Nombre.
		\item Descripción.
		\item Imágenes.
		\item Año de inicio.
		\item Año de fin.
		\item Precio.
		\item Tipo de dispositivos en los que se encuentra.
		\begin{myEnumRFM}
			\item Portátiles.
			\item De escritorio.
		\end{myEnumRFM}
		\item Detalles específicos del tipo de componente.
		\begin{myEnumRFM}
			\item Detalles de CPU.
			\begin{myEnumRFM}
				\item Memoria ROM (obligatorio).
				\item Memoria RAM (obligatorio).
				\item Velocidad de reloj (obligatorio).
				\item Potencia (obligatorio).
				\item Tamaño de palabra (obligatorio).
				\item Nanómetros de los transistores (obligatorio).
				\item Passmark (obligatorio).
				\item Número de transistores (obligatorio).
			\end{myEnumRFM}
		\end{myEnumRFM}
	\end{myEnumRFM}
\end{myEnumRFM}

\paragraph*{Administración del museo}
\newlist{myEnumRFA}{enumerate}{9}
\setlist[myEnumRFA,1]{label=\textbf{RF-ADM-\arabic*.}}
\setlist[myEnumRFA,2]{label*=\textbf{\arabic*.}}
\setlist[myEnumRFA,3]{label*=\textbf{\arabic*.}}
\setlist[myEnumRFA,4]{label*=\textbf{\arabic*.}}
\setlist[myEnumRFA,5]{label*=\textbf{\arabic*.}}
\setlist[myEnumRFA,6]{label*=\textbf{\arabic*.}}
\setlist[myEnumRFA,7]{label*=\textbf{\arabic*.}}
\setlist[myEnumRFA,8]{label*=\textbf{\arabic*.}}
\setlist[myEnumRFA,9]{label*=\textbf{\arabic*.}}
\begin{myEnumRFA}
	\item El sistema permitirá al usuario iniciar sesión mediante un formulario.
	\begin{myEnumRFA}
		\item El formulario se solicitan los siguientes campos.
		\begin{myEnumRFA}
			\item Correo electrónico (obligatorio).
			\item Contraseña (obligatorio).
		\end{myEnumRFA}
		\item Si los campos no son correctos, se mostrará de nuevo el inicio de sesión.
		\item Si los campos son válidos, se accederá al sistema como administrador.
	\end{myEnumRFA}
%	\item El sistema permitirá al usuario modificar la contraseña.
	\item El sistema mostrará un listado de los periodos existentes.
	\item El sistema mostrará los detalles de un periodo.
	\begin{myEnumRFA}
		\item Especificados en \ref{it:detalles_periodo}
	\end{myEnumRFA}
	\item El sistema mostrará los detalles de un componente.
	\begin{myEnumRFA}
		\item Especificados en \ref{it:detalles_comp}
	\end{myEnumRFA}
	\item El sistema permitirá añadir un periodo.
	\begin{myEnumRFA}
		\item\label{it:campos_periodo} Se introducirán los datos mediante un formulario con los siguientes campos.
		\begin{myEnumRFA}
			\item Nombre (obligatorio).
			\item Características (obligatorio).
			\item Listado de datos curiosos (Sabías qué...) (obligatorio).
			\item Eventos ocurridos durante el periodo (obligatorio).
		\end{myEnumRFA}
	\end{myEnumRFA}
	\item El sistema permitirá añadir un componente a un periodo existente.
	\begin{myEnumRFA}
		\item\label{it:campos_comp} Se introducirán los datos mediante un formulario con los siguientes campos.
		\begin{myEnumRFA}
			\item Nombre (obligatorio).
			\item Tipo de componente (obligatorio).
			\begin{myEnumRFA}
				\item CPU.
			\end{myEnumRFA}
			\item Descripción (obligatorio).
			\item Familia de componente (obligatorio).
			\item Imágenes.
			\item Año de inicio (obligatorio).
			\item Año de fin (obligatorio).
			\item Precio (obligatorio).
			\item Tipo de dispositivos en los que se encuentra.
			\begin{myEnumRFA}
				\item Portátiles.
				\item De escritorio.
			\end{myEnumRFA}
			\item Sistema famoso que lo contiene.
			\begin{myEnumRFA}
				\item Nombre del sistema.
				\item Imagen del sistema.
			\end{myEnumRFA}
			\item Detalles específicos del tipo de componente.
			\begin{myEnumRFA}
				\item Detalles de CPU.
				\begin{myEnumRFA}
					\item Memoria ROM (obligatorio).
					\item Memoria RAM (obligatorio).
					\item Velocidad de reloj (obligatorio).
					\item Potencia (obligatorio).
					\item Tamaño de palabra (obligatorio).
					\item Nanómetros de los transistores (obligatorio).
					\item Passmark (obligatorio).
					\item Número de transistores (obligatorio).
				\end{myEnumRFA}
			\end{myEnumRFA}
		\end{myEnumRFA}
	\end{myEnumRFA}
	\item El sistema permitirá editar periodos.
	\begin{myEnumRFA}
		\item Se introducirán los datos mediante un formulario.
		\begin{myEnumRFA}
			\item Campos especificados en \ref{it:campos_periodo}
		\end{myEnumRFA}
	\end{myEnumRFA}
	\item El sistema permitirá editar componentes.
	\begin{myEnumRFA}
		\item Se introducirán los datos mediante un formulario.
		\begin{myEnumRFA}
			\item Campos especificados en \ref{it:campos_comp}
		\end{myEnumRFA}
	\end{myEnumRFA}
	\item El sistema permitirá eliminar un periodo.
	\begin{myEnumRFA}
		\item Se pedirá confirmación antes de eliminarlo.
		\item Se eliminarán los componentes pertenecientes a dicho periodo.
	\end{myEnumRFA}
	\item El sistema permitirá eliminar un componente.
	\begin{myEnumRFA}
		\item Se pedirá confirmación antes de eliminarlo.
	\end{myEnumRFA}
\end{myEnumRFA}


%\subsubsection{Requisitos de rendimiento}
%\subsubsection{Requisitos lógicos de BD}
%\subsubsection{Requisitos de desarrollo}
%\subsubsection{Restricciones de diseño}
\subsubsection{Atributos del sistema}

\paragraph*{Seguridad}
\newlist{myEnumSEG}{enumerate}{2}
\setlist[myEnumSEG,1]{label=\textbf{RNF-SEG-\arabic*.}}
\setlist[myEnumSEG,2]{label*=\textbf{\arabic*.}}
\begin{myEnumSEG}
	\item La parte de administración del sistema se asegurará de que el usuario se identifica para acceder a ella.
	\begin{myEnumSEG}
		\item El usuario se identificará mediante un email y una contraseña.
	\end{myEnumSEG}
	\item El sistema cifrará la contraseña para almacenarla en la base de datos.
\end{myEnumSEG}


\subsection{Identificación de Actores del Sistema} 
\subsubsection{Usuario estándar}
Actor que interactúa con el sistema. Tiene acceso de lectura a la página web del museo. Solo debe tener un conocimiento básico para navegar por internet.
\subsubsection{Usuario administrador}
Actor que interactúa con el sistema. Debe iniciar sesión en la parte de administración del sistema, es el único actor con acceso a esta. Es responsable de gestionar el sistema y su mantenimiento. Debe tener amplios conocimientos sobre el sistema.

\subsection{Especificación de Casos de Uso}

\begin{figure}[H]
\centering
\includegraphics[scale=0.7]{casos-uso-museo}
\caption{Diagrama de casos de uso del museo}
\end{figure}

\begin{table}[htbp]
  \centering
  \caption{Especificación Caso de Uso 1}
    \begin{tabular}{p{20.855em}r}
\cmidrule{1-1}    \rowcolor[rgb]{ .949,  .949,  .949} \multicolumn{1}{p{20.855em}}{\textbf{Nombre del caso de uso}} & \multicolumn{1}{r}{\cellcolor[rgb]{ 1,  1,  1}} \\
\cmidrule{1-1}    \multicolumn{1}{p{20.855em}}{Consultar periodos} & \multicolumn{1}{r}{} \\
    \midrule
    \rowcolor[rgb]{ .949,  .949,  .949} \multicolumn{2}{p{31.64em}}{\textbf{Descripción}} \\
    \midrule
    \multicolumn{2}{p{31.64em}}{Un usuario estándar puede visualizar los periodos existentes en el museo.} \\
    \bottomrule
    \end{tabular}%
  \label{espec_caso_uso_1}%
  \vspace{-4mm}
\end{table}%

\begin{table}[htbp]
  \centering
  \caption{Especificación Caso de Uso 2}
    \begin{tabular}{p{20.855em}r}
\cmidrule{1-1}    \rowcolor[rgb]{ .949,  .949,  .949} \multicolumn{1}{p{20.855em}}{\textbf{Nombre del caso de uso}} & \multicolumn{1}{r}{\cellcolor[rgb]{ 1,  1,  1}} \\
\cmidrule{1-1}    \multicolumn{1}{p{20.855em}}{Consultar componentes} & \multicolumn{1}{r}{} \\
    \midrule
    \rowcolor[rgb]{ .949,  .949,  .949} \multicolumn{2}{p{31.64em}}{\textbf{Descripción}} \\
    \midrule
    \multicolumn{2}{p{31.64em}}{Un usuario estándar puede visualizar los componentes pertenecientes a cada periodo del museo.} \\
    \bottomrule
    \end{tabular}%
  \label{espec_caso_uso_2}%
  \vspace{-4mm}
\end{table}%


\begin{figure}[H]
\centering
\vspace{15mm}
\includegraphics[scale=0.7]{casos-uso-admin}
\caption{Diagrama de casos de uso de la administración del museo}
\end{figure}

\begin{table}[htbp]
  \centering
  \caption{Especificación Caso de Uso 3}
    \begin{tabular}{p{20.855em}r}
\cmidrule{1-1}    \rowcolor[rgb]{ .949,  .949,  .949} \multicolumn{1}{p{20.855em}}{\textbf{Nombre del caso de uso}} & \multicolumn{1}{r}{\cellcolor[rgb]{ 1,  1,  1}} \\
\cmidrule{1-1}    \multicolumn{1}{p{20.855em}}{Iniciar sesión} & \multicolumn{1}{r}{} \\
    \midrule
    \rowcolor[rgb]{ .949,  .949,  .949} \multicolumn{2}{p{31.64em}}{\textbf{Descripción}} \\
    \midrule
    \multicolumn{2}{p{31.64em}}{Un usuario puede iniciar sesión en la aplicación para acceder a esta como administrador.} \\
    \bottomrule
    \end{tabular}%
  \label{espec_caso_uso_3}%
  \vspace{-2mm}
\end{table}%

\begin{table}[htbp]
  \centering
  \caption{Especificación Caso de Uso 4}
    \begin{tabular}{p{20.855em}r}
\cmidrule{1-1}    \rowcolor[rgb]{ .949,  .949,  .949} \multicolumn{1}{p{20.855em}}{\textbf{Nombre del caso de uso}} & \multicolumn{1}{r}{\cellcolor[rgb]{ 1,  1,  1}} \\
\cmidrule{1-1}    \multicolumn{1}{p{20.855em}}{Consultar periodos} & \multicolumn{1}{r}{} \\
    \midrule
    \rowcolor[rgb]{ .949,  .949,  .949} \multicolumn{2}{p{31.64em}}{\textbf{Descripción}} \\
    \midrule
    \multicolumn{2}{p{31.64em}}{El administrador puede consultar los periodos existentes en el sistema.} \\
    \bottomrule
    \end{tabular}%
  \label{espec_caso_uso_4}%
  \vspace{-2mm}
\end{table}%

\begin{table}[htbp]
  \centering
  \caption{Especificación Caso de Uso 5}
    \begin{tabular}{p{20.855em}r}
\cmidrule{1-1}    \rowcolor[rgb]{ .949,  .949,  .949} \multicolumn{1}{p{20.855em}}{\textbf{Nombre del caso de uso}} & \multicolumn{1}{r}{\cellcolor[rgb]{ 1,  1,  1}} \\
\cmidrule{1-1}    \multicolumn{1}{p{20.855em}}{Añadir periodo} & \multicolumn{1}{r}{} \\
    \midrule
    \rowcolor[rgb]{ .949,  .949,  .949} \multicolumn{2}{p{31.64em}}{\textbf{Descripción}} \\
    \midrule
    \multicolumn{2}{p{31.64em}}{El administrador puede añadir periodos a la base de datos del sistema.} \\
    \bottomrule
    \end{tabular}%
  \label{espec_caso_uso_5}%
  \vspace{-2mm}
\end{table}%

\begin{table}[htbp]
  \centering
  \caption{Especificación Caso de Uso 6}
    \begin{tabular}{p{20.855em}r}
\cmidrule{1-1}    \rowcolor[rgb]{ .949,  .949,  .949} \multicolumn{1}{p{20.855em}}{\textbf{Nombre del caso de uso}} & \multicolumn{1}{r}{\cellcolor[rgb]{ 1,  1,  1}} \\
\cmidrule{1-1}    \multicolumn{1}{p{20.855em}}{Modificar periodo} & \multicolumn{1}{r}{} \\
    \midrule
    \rowcolor[rgb]{ .949,  .949,  .949} \multicolumn{2}{p{31.64em}}{\textbf{Descripción}} \\
    \midrule
    \multicolumn{2}{p{31.64em}}{El administrador puede modificar los periodos existentes en la base de datos del sistema.} \\
    \bottomrule
    \end{tabular}%
  \label{espec_caso_uso_6}%
  \vspace{-2mm}
\end{table}%

\begin{table}[htbp]
  \centering
  \caption{Especificación Caso de Uso 7}
    \begin{tabular}{p{20.855em}r}
\cmidrule{1-1}    \rowcolor[rgb]{ .949,  .949,  .949} \multicolumn{1}{p{20.855em}}{\textbf{Nombre del caso de uso}} & \multicolumn{1}{r}{\cellcolor[rgb]{ 1,  1,  1}} \\
\cmidrule{1-1}    \multicolumn{1}{p{20.855em}}{Eliminar periodo} & \multicolumn{1}{r}{} \\
    \midrule
    \rowcolor[rgb]{ .949,  .949,  .949} \multicolumn{2}{p{31.64em}}{\textbf{Descripción}} \\
    \midrule
    \multicolumn{2}{p{31.64em}}{El administrador puede eliminar los periodos existentes en la base de datos del sistema.} \\
    \bottomrule
    \end{tabular}%
  \label{espec_caso_uso_7}%
  \vspace{-2mm}
\end{table}%

\begin{table}[htbp]
  \centering
  \caption{Especificación Caso de Uso 8}
    \begin{tabular}{p{20.855em}r}
\cmidrule{1-1}    \rowcolor[rgb]{ .949,  .949,  .949} \multicolumn{1}{p{20.855em}}{\textbf{Nombre del caso de uso}} & \multicolumn{1}{r}{\cellcolor[rgb]{ 1,  1,  1}} \\
\cmidrule{1-1}    \multicolumn{1}{p{20.855em}}{Consultar componentes} & \multicolumn{1}{r}{} \\
    \midrule
    \rowcolor[rgb]{ .949,  .949,  .949} \multicolumn{2}{p{31.64em}}{\textbf{Descripción}} \\
    \midrule
    \multicolumn{2}{p{31.64em}}{El administrador puede consultar los componentes existentes en el sistema.} \\
    \bottomrule
    \end{tabular}%
  \label{espec_caso_uso_8}%
  \vspace{-2mm}
\end{table}%

\begin{table}[htbp]
  \centering
  \caption{Especificación Caso de Uso 9}
    \begin{tabular}{p{20.855em}r}
\cmidrule{1-1}    \rowcolor[rgb]{ .949,  .949,  .949} \multicolumn{1}{p{20.855em}}{\textbf{Nombre del caso de uso}} & \multicolumn{1}{r}{\cellcolor[rgb]{ 1,  1,  1}} \\
\cmidrule{1-1}    \multicolumn{1}{p{20.855em}}{Añadir componente} & \multicolumn{1}{r}{} \\
    \midrule
    \rowcolor[rgb]{ .949,  .949,  .949} \multicolumn{2}{p{31.64em}}{\textbf{Descripción}} \\
    \midrule
    \multicolumn{2}{p{31.64em}}{El administrador puede añadir componentes a los periodos existentes en la base de datos del sistema.} \\
    \bottomrule
    \end{tabular}%
  \label{espec_caso_uso_9}%
  \vspace{-2mm}
\end{table}%

\begin{table}[htbp]
  \centering
  \caption{Especificación Caso de Uso 10}
    \begin{tabular}{p{20.855em}r}
\cmidrule{1-1}    \rowcolor[rgb]{ .949,  .949,  .949} \multicolumn{1}{p{20.855em}}{\textbf{Nombre del caso de uso}} & \multicolumn{1}{r}{\cellcolor[rgb]{ 1,  1,  1}} \\
\cmidrule{1-1}    \multicolumn{1}{p{20.855em}}{Modificar componente} & \multicolumn{1}{r}{} \\
    \midrule
    \rowcolor[rgb]{ .949,  .949,  .949} \multicolumn{2}{p{31.64em}}{\textbf{Descripción}} \\
    \midrule
    \multicolumn{2}{p{31.64em}}{El administrador puede modificar los componentes existentes en la base de datos del sistema.} \\
    \bottomrule
    \end{tabular}%
  \label{espec_caso_uso_10}%
  \vspace{-2mm}
\end{table}%

\begin{table}[htbp]
  \centering
  \caption{Especificación Caso de Uso 11}
    \begin{tabular}{p{20.855em}r}
\cmidrule{1-1}    \rowcolor[rgb]{ .949,  .949,  .949} \multicolumn{1}{p{20.855em}}{\textbf{Nombre del caso de uso}} & \multicolumn{1}{r}{\cellcolor[rgb]{ 1,  1,  1}} \\
\cmidrule{1-1}    \multicolumn{1}{p{20.855em}}{Eliminar componente} & \multicolumn{1}{r}{} \\
    \midrule
    \rowcolor[rgb]{ .949,  .949,  .949} \multicolumn{2}{p{31.64em}}{\textbf{Descripción}} \\
    \midrule
    \multicolumn{2}{p{31.64em}}{El administrador puede eliminar los componentes existentes en la base de datos del sistema.} \\
    \bottomrule
    \end{tabular}%
  \label{espec_caso_uso_11}%
  \vspace{-2mm}
\end{table}%



%\newpage
%\section{ASI 3: IDENTIFICACIÓN DE SUBSISTEMAS DE ANÁLISIS}
%
%\subsection{Descripción de los Subsistemas} 
%%A continuación se describen los subsistemas identificados en el análisis
%%\subsubsection{Vistas}
%%
%%\subsubsection{Modelos}
%%
%%\subsubsection{Bases de datos}
%%
%%\subsubsection{Servidor}
%
%
%\subsection{Descripción de los Interfaces entre Subsistemas}



\newpage
\section{ASI 4: ANÁLISIS DE LOS CASOS DE USO}
\subsection{Caso de Uso 1} 

\begin{table}[H]
  \centering
  \vspace{-5mm}
  \caption{Análisis del Caso de Uso 1}
    \begin{tabular}{p{7.5em}p{24.145em}}
    \toprule
    \rowcolor[rgb]{ .871,  .918,  .965} \multicolumn{2}{p{31.645em}}{\textbf{Consultar periodos}} \\
    \midrule
    \rowcolor[rgb]{ .906,  .902,  .902} \textbf{Precondiciones} & \cellcolor[rgb]{ 1,  1,  1}- \\
    \midrule
    \rowcolor[rgb]{ .906,  .902,  .902} \textbf{Postcondiciones} & \cellcolor[rgb]{ 1,  1,  1}- \\
    \midrule
    \rowcolor[rgb]{ .906,  .902,  .902} \textbf{Actores} & \cellcolor[rgb]{ 1,  1,  1}Usuario estándar \\
    \midrule
    \rowcolor[rgb]{ .906,  .902,  .902} \textbf{Descripción} & \cellcolor[rgb]{ 1,  1,  1}El usuario accederá a la vista principal del museo y podrá visualizar los periodos existentes. Podrá acceder a los periodos. Podrá acceder a los componentes de los periodos. Podrá realizar una búsqueda.\\
    \midrule
    \rowcolor[rgb]{ .906,  .902,  .902} \textbf{Escenarios          Secundarios} & \cellcolor[rgb]{ 1,  1,  1}  \\
    \bottomrule
    \end{tabular}%
\end{table}%
 
\subsection{Caso de Uso 2}
\begin{table}[H]
  \centering
  \vspace{-5mm}
  \caption{Análisis del Caso de Uso 2}
    \begin{tabular}{p{7.5em}p{24.145em}}
    \toprule
    \rowcolor[rgb]{ .871,  .918,  .965} \multicolumn{2}{p{31.645em}}{\textbf{Consultar componentes}} \\
    \midrule
    \rowcolor[rgb]{ .906,  .902,  .902} \textbf{Precondiciones} & \cellcolor[rgb]{ 1,  1,  1}- \\
    \midrule
    \rowcolor[rgb]{ .906,  .902,  .902} \textbf{Postcondiciones} & \cellcolor[rgb]{ 1,  1,  1}- \\
    \midrule
    \rowcolor[rgb]{ .906,  .902,  .902} \textbf{Actores} & \cellcolor[rgb]{ 1,  1,  1}Usuario estándar \\
    \midrule
    \rowcolor[rgb]{ .906,  .902,  .902} \textbf{Descripción} & \cellcolor[rgb]{ 1,  1,  1}El usuario accederá a la vista de un periodo y podrá visualizar los componentes pertenecientes al mismo. Podrá acceder a otros periodos. Podrá acceder a los otros componentes de ese periodo. \\
    \midrule
    \rowcolor[rgb]{ .906,  .902,  .902} \textbf{Escenarios          Secundarios} & \cellcolor[rgb]{ 1,  1,  1}  \\
    \bottomrule
    \end{tabular}%
\end{table}%
 
\subsection{Caso de Uso 3}
\begin{table}[H]
  \centering
  \vspace{-5mm}
  \caption{Análisis del Caso de Uso 3}
    \begin{tabular}{p{7.5em}p{24.145em}}
    \toprule
    \rowcolor[rgb]{ .871,  .918,  .965} \multicolumn{2}{p{31.645em}}{\textbf{Iniciar sesión}} \\
    \midrule
    \rowcolor[rgb]{ .906,  .902,  .902} \textbf{Precondiciones} & \cellcolor[rgb]{ 1,  1,  1}El usuario no debe haber iniciado sesión. \\
    \midrule
    \rowcolor[rgb]{ .906,  .902,  .902} \textbf{Postcondiciones} & \cellcolor[rgb]{ 1,  1,  1}- \\
    \midrule
    \rowcolor[rgb]{ .906,  .902,  .902} \textbf{Actores} & \cellcolor[rgb]{ 1,  1,  1}Usuario \\
    \midrule
    \rowcolor[rgb]{ .906,  .902,  .902} \textbf{Descripción} & \cellcolor[rgb]{ 1,  1,  1}El usuario accederá a la página principal de la aplicación de administración e introducirá su email y contraseña para iniciar sesión en el sistema. \\
    \midrule
    \rowcolor[rgb]{ .906,  .902,  .902} \textbf{Escenarios          Secundarios} & \cellcolor[rgb]{ 1,  1,  1} Los datos introducidos no se corresponden con los datos de un usuario con permiso de administrador. Se muestra un error y de nuevo se solicita iniciar sesión. \\
    \bottomrule
    \end{tabular}%
\end{table}%
 
\subsection{Caso de Uso 4}
\begin{table}[H]
  \centering
  \vspace{-5mm}
  \caption{Análisis del Caso de Uso 4}
    \begin{tabular}{p{7.5em}p{24.145em}}
    \toprule
    \rowcolor[rgb]{ .871,  .918,  .965} \multicolumn{2}{p{31.645em}}{\textbf{Consultar periodos}} \\
    \midrule
    \rowcolor[rgb]{ .906,  .902,  .902} \textbf{Precondiciones} & \cellcolor[rgb]{ 1,  1,  1}El usuario debe haber iniciado sesión. \\
    \midrule
    \rowcolor[rgb]{ .906,  .902,  .902} \textbf{Postcondiciones} & \cellcolor[rgb]{ 1,  1,  1}- \\
    \midrule
    \rowcolor[rgb]{ .906,  .902,  .902} \textbf{Actores} & \cellcolor[rgb]{ 1,  1,  1}Usuario administrador \\
    \midrule
    \rowcolor[rgb]{ .906,  .902,  .902} \textbf{Descripción} & \cellcolor[rgb]{ 1,  1,  1}El usuario accederá al listado de periodos. Podrá acceder a cada uno de ellos. \\
    \midrule
    \rowcolor[rgb]{ .906,  .902,  .902} \textbf{Escenarios          Secundarios} & \cellcolor[rgb]{ 1,  1,  1}Aún no existe ningún periodo en el sistema. Se muestra una tabla vacía. \\
    \bottomrule
    \end{tabular}%
\end{table}
 
\subsection{Caso de Uso 5}
\begin{table}[H]
  \centering
  \vspace{-5mm}
  \caption{Análisis del Caso de Uso 5}
    \begin{tabular}{p{7.5em}p{24.145em}}
    \toprule
    \rowcolor[rgb]{ .871,  .918,  .965} \multicolumn{2}{p{31.645em}}{\textbf{Añadir periodo}} \\
    \midrule
    \rowcolor[rgb]{ .906,  .902,  .902} \textbf{Precondiciones} & \cellcolor[rgb]{ 1,  1,  1}El usuario debe haber iniciado sesión. \\
    \midrule
    \rowcolor[rgb]{ .906,  .902,  .902} \textbf{Postcondiciones} & \cellcolor[rgb]{ 1,  1,  1}El periodo añadido se guardará en la base de datos. \\
    \midrule
    \rowcolor[rgb]{ .906,  .902,  .902} \textbf{Actores} & \cellcolor[rgb]{ 1,  1,  1}Usuario administrador \\
    \midrule
    \rowcolor[rgb]{ .906,  .902,  .902} \textbf{Descripción} & \cellcolor[rgb]{ 1,  1,  1}El usuario accede al formulario para añadir un periodo. Rellena los campos necesarios. Pulsa el botón de guardar. \\
    \midrule
    \rowcolor[rgb]{ .906,  .902,  .902} \textbf{Escenarios          Secundarios} & \cellcolor[rgb]{ 1,  1,  1}- Se pulsa el botón cancelar. El formulario se restablece.\par - Se intenta acceder a otra página de la aplicación sin haber guardado los cambios. Se avisa de la situación y se pide una confirmación para continuar.\par - No se puede añadir el periodo. Se mostrará un error avisando de la situación. \\
    \bottomrule
    \end{tabular}%
\end{table}%
 
\subsection{Caso de Uso 6}
\begin{table}[H]
  \centering
  \vspace{-5mm}
  \caption{Análisis del Caso de Uso 6}
    \begin{tabular}{p{7.5em}p{24.145em}}
    \toprule
    \rowcolor[rgb]{ .871,  .918,  .965} \multicolumn{2}{p{31.645em}}{\textbf{Modificar periodo}} \\
    \midrule
    \rowcolor[rgb]{ .906,  .902,  .902} \textbf{Precondiciones} & \cellcolor[rgb]{ 1,  1,  1}El usuario debe haber iniciado sesión. Debe existir al menos un periodo. \\
    \midrule
    \rowcolor[rgb]{ .906,  .902,  .902} \textbf{Postcondiciones} & \cellcolor[rgb]{ 1,  1,  1}Los cambios realizados al periodo se guardarán en la base de datos. \\
    \midrule
    \rowcolor[rgb]{ .906,  .902,  .902} \textbf{Actores} & \cellcolor[rgb]{ 1,  1,  1}Usuario administrador \\
    \midrule
    \rowcolor[rgb]{ .906,  .902,  .902} \textbf{Descripción} & \cellcolor[rgb]{ 1,  1,  1}El usuario accede al periodo deseado y selecciona la opción de editar. Se mostrará el formulario correspondiente. Se realizan los cambios en el formulario. Pulsa el botón de guardar. \\
    \midrule
    \rowcolor[rgb]{ .906,  .902,  .902} \textbf{Escenarios          Secundarios} & \cellcolor[rgb]{ 1,  1,  1}- Se pulsa el botón cancelar. El formulario se restablece.\par - Se intenta acceder a otra página de la aplicación sin haber guardado los cambios. Se avisa de la situación y se pide una confirmación para continuar.\par - No se puede modificar el periodo. Se mostrará un error avisando de la situación. \\
    \bottomrule
    \end{tabular}%
\end{table}%
 
\subsection{Caso de Uso 7}
\begin{table}[H]
  \centering
  \vspace{-5mm}
  \caption{Análisis del Caso de Uso 7}
    \begin{tabular}{p{7.5em}p{24.145em}}
    \toprule
    \rowcolor[rgb]{ .871,  .918,  .965} \multicolumn{2}{p{31.645em}}{\textbf{Eliminar periodo}} \\
    \midrule
    \rowcolor[rgb]{ .906,  .902,  .902} \textbf{Precondiciones} & \cellcolor[rgb]{ 1,  1,  1}El usuario debe haber iniciado sesión. Debe existir al menos un periodo. \\
    \midrule
    \rowcolor[rgb]{ .906,  .902,  .902} \textbf{Postcondiciones} & \cellcolor[rgb]{ 1,  1,  1}El periodo eliminado y los componentes que pertenecen al mismo se borrarán de la base de datos y dejarán de mostrarse en la aplicación. \\
    \midrule
    \rowcolor[rgb]{ .906,  .902,  .902} \textbf{Actores} & \cellcolor[rgb]{ 1,  1,  1}Usuario administrador \\
    \midrule
    \rowcolor[rgb]{ .906,  .902,  .902} \textbf{Descripción} & \cellcolor[rgb]{ 1,  1,  1}El usuario accede al periodo deseado y selecciona la opción de eliminar. Se pide confirmación para eliminarlo. Se acepta esta confirmación. \\
    \midrule
    \rowcolor[rgb]{ .906,  .902,  .902} \textbf{Escenarios          Secundarios} & \cellcolor[rgb]{ 1,  1,  1}- No se acepta la confirmación para eliminarlo. El periodo y sus componentes permanecen en la base de datos.\par - No se puede eliminar el periodo. Se mostrará un error avisando de la situación. \\
    \bottomrule
    \end{tabular}%
\end{table}%
 
\subsection{Caso de Uso 8}
\begin{table}[H]
  \centering
  \vspace{-5mm}
  \caption{Análisis del Caso de Uso 8}
    \begin{tabular}{p{7.5em}p{24.145em}}
    \toprule
    \rowcolor[rgb]{ .871,  .918,  .965} \multicolumn{2}{p{31.645em}}{\textbf{Consultar componentes}} \\
    \midrule
    \rowcolor[rgb]{ .906,  .902,  .902} \textbf{Precondiciones} & \cellcolor[rgb]{ 1,  1,  1}El usuario debe haber iniciado sesión. Debe existir al menos un periodo. \\
    \midrule
    \rowcolor[rgb]{ .906,  .902,  .902} \textbf{Postcondiciones} & \cellcolor[rgb]{ 1,  1,  1}- \\
    \midrule
    \rowcolor[rgb]{ .906,  .902,  .902} \textbf{Actores} & \cellcolor[rgb]{ 1,  1,  1}Usuario administrador \\
    \midrule
    \rowcolor[rgb]{ .906,  .902,  .902} \textbf{Descripción} & \cellcolor[rgb]{ 1,  1,  1}El usuario accederá a un periodo existente y visualizará los componentes pertenecientes a este. Podrá acceder a cada uno de ellos. \\
    \midrule
    \rowcolor[rgb]{ .906,  .902,  .902} \textbf{Escenarios          Secundarios} & \cellcolor[rgb]{ 1,  1,  1}Aún no existen componentes para el periodo que se consulta. Se mostrará una tabla vacía. \\
    \bottomrule
    \end{tabular}%
\end{table}%
 
\subsection{Caso de Uso 9}
\begin{table}[H]
  \centering
  \vspace{-5mm}
  \caption{Análisis del Caso de Uso 9}
    \begin{tabular}{p{7.5em}p{24.145em}}
    \toprule
    \rowcolor[rgb]{ .871,  .918,  .965} \multicolumn{2}{p{31.645em}}{\textbf{Añadir componente}} \\
    \midrule
    \rowcolor[rgb]{ .906,  .902,  .902} \textbf{Precondiciones} & \cellcolor[rgb]{ 1,  1,  1}El usuario debe haber iniciado sesión. Debe existir al menos un periodo. \\
    \midrule
    \rowcolor[rgb]{ .906,  .902,  .902} \textbf{Postcondiciones} & \cellcolor[rgb]{ 1,  1,  1}El componente añadido se guardará en la base de datos y se asociará al periodo correspondiente. \\
    \midrule
    \rowcolor[rgb]{ .906,  .902,  .902} \textbf{Actores} & \cellcolor[rgb]{ 1,  1,  1}Usuario administrador \\
    \midrule
    \rowcolor[rgb]{ .906,  .902,  .902} \textbf{Descripción} & \cellcolor[rgb]{ 1,  1,  1} \\
    \midrule
    \rowcolor[rgb]{ .906,  .902,  .902} \textbf{Escenarios          Secundarios} & \cellcolor[rgb]{ 1,  1,  1}- Se pulsa el botón cancelar. El formulario se restablece.\par - Se intenta acceder a otra página de la aplicación sin haber guardado los cambios. Se avisa de la situación y se pide una confirmación para continuar.\par - No se puede añadir el componente. Se mostrará un error avisando de la situación.  \\
    \bottomrule
    \end{tabular}%
\end{table}%
 
\subsection{Caso de Uso 10}
\begin{table}[H]
  \centering
  \vspace{-5mm}
  \caption{Análisis del Caso de Uso 10}
    \begin{tabular}{p{7.5em}p{24.145em}}
    \toprule
    \rowcolor[rgb]{ .871,  .918,  .965} \multicolumn{2}{p{31.645em}}{\textbf{Modificar componente}} \\
    \midrule
    \rowcolor[rgb]{ .906,  .902,  .902} \textbf{Precondiciones} & \cellcolor[rgb]{ 1,  1,  1}El usuario debe haber iniciado sesión. Debe existir al menos un componente. \\
    \midrule
    \rowcolor[rgb]{ .906,  .902,  .902} \textbf{Postcondiciones} & \cellcolor[rgb]{ 1,  1,  1}Los cambios realizados en el componente se guardarán en la base de datos. \\
    \midrule
    \rowcolor[rgb]{ .906,  .902,  .902} \textbf{Actores} & \cellcolor[rgb]{ 1,  1,  1}Usuario administrador \\
    \midrule
    \rowcolor[rgb]{ .906,  .902,  .902} \textbf{Descripción} & \cellcolor[rgb]{ 1,  1,  1}El usuario accede al componente deseado y selecciona la opción de editar. Se mostrará el formulario correspondiente. Se realizan los cambios en el formulario. Pulsa el botón de guardar. \\
    \midrule
    \rowcolor[rgb]{ .906,  .902,  .902} \textbf{Escenarios          Secundarios} & \cellcolor[rgb]{ 1,  1,  1}- Se pulsa el botón cancelar. El formulario se restablece.\par - Se intenta acceder a otra página de la aplicación sin haber guardado los cambios. Se avisa de la situación y se pide una confirmación para continuar.\par - No se puede modificar el componente. Se mostrará un error avisando de la situación.  \\
    \bottomrule
    \end{tabular}%
\end{table}
 
\subsection{Caso de Uso 11}
\begin{table}[H]
  \centering
  \vspace{-5mm}
  \caption{Análisis del Caso de Uso 11}
    \begin{tabular}{p{7.5em}p{24.145em}}
    \toprule
    \rowcolor[rgb]{ .871,  .918,  .965} \multicolumn{2}{p{31.645em}}{\textbf{Eliminar componente}} \\
    \midrule
    \rowcolor[rgb]{ .906,  .902,  .902} \textbf{Precondiciones} & \cellcolor[rgb]{ 1,  1,  1}El usuario debe haber iniciado sesión. Debe existir al menos un componente. \\
    \midrule
    \rowcolor[rgb]{ .906,  .902,  .902} \textbf{Postcondiciones} & \cellcolor[rgb]{ 1,  1,  1}El componente eliminado  se borrará de la base de datos y dejará de mostrarse en la aplicación. \\
    \midrule
    \rowcolor[rgb]{ .906,  .902,  .902} \textbf{Actores} & \cellcolor[rgb]{ 1,  1,  1}Usuario administrador \\
    \midrule
    \rowcolor[rgb]{ .906,  .902,  .902} \textbf{Descripción} & \cellcolor[rgb]{ 1,  1,  1}El usuario accede al componente deseado y selecciona la opción de eliminar. Se pide confirmación para eliminarlo. Se acepta esta confirmación.  \\
    \midrule
    \rowcolor[rgb]{ .906,  .902,  .902} \textbf{Escenarios          Secundarios} & \cellcolor[rgb]{ 1,  1,  1}- No se acepta la confirmación para eliminarlo. El componente permanece en la base de datos.\par - No se puede eliminar el componente. Se mostrará un error avisando de la situación.  \\
    \bottomrule
    \end{tabular}%
\end{table}%


\newpage
\section{ASI 5: ANÁLISIS DE CLASES}

\subsection{Diagrama de Clases} 
\subsubsection{Museo}
\begin{figure}[H]
\centering
\centerline{\includegraphics[scale=0.5]{asi-clases-museo}}
\caption{Análisis de clases: diagrama de clases del museo}
\end{figure}
\subsubsection{Administración del museo}
\begin{figure}[H]
\centering
\centerline{\includegraphics[scale=0.5]{asi-clases-admin}}
\caption{Análisis de clases: diagrama de clases de la administración}
\end{figure}

\subsection{Descripción de las Clases}
\textit{Period}, \textit{MyComponent} y \textit{Cpu} son iguales en el proyecto del museo y en el de la administración, por lo tanto se describen una única vez a continuación:

\begin{table}[H]
\vspace{-4mm}
  \centering
  \caption{Descripción de la clase Period}
    \begin{tabular}{p{8.645em}p{5em}p{15.5em}}
    \toprule
    \rowcolor[rgb]{ .851,  .886,  .953} \multicolumn{3}{p{31.285em}}{\textbf{Periodo}} \\ \midrule
    \rowcolor[rgb]{ .949,  .949,  .949} \multicolumn{3}{p{31.285em}}{\textbf{Descripción}} \\ \midrule
    \multicolumn{3}{p{31.285em}}{Clase que modela un periodo.} \\ \midrule
    \rowcolor[rgb]{ .906,  .902,  .902} \multicolumn{3}{p{31.285em}}{\textbf{Atributos propuestos}} \\ \midrule
    \textbf{id} & number & Identificador del periodo\\
    \textbf{name} & string & Nombre del periodo \\ 
    \textbf{trivia} & string[] & Curiosidades del periodo \\
    \textbf{details} & string[] & Detalles del periodo \\
    \textbf{events} & string[] & Eventos ocurridos durante este periodo \\
    \textbf{famous\_systems} & string[] & Sistemas famosos que llevaban componentes pertenecientes al periodo \\ \midrule
    \rowcolor[rgb]{ .906,  .902,  .902} \multicolumn{3}{p{31.285em}}{\textbf{Métodos propuestos}} \\ \midrule
    \multicolumn{3}{p{31.285em}}{-} \\ \bottomrule
    \end{tabular}%
\end{table}%

\begin{table}[H]
\vspace{-4mm}
  \centering
  \caption{Descripción de la interfaz MyComponent }
    \begin{tabular}{p{8.645em}p{5em}p{15.5em}}
    \toprule
    \rowcolor[rgb]{ .851,  .886,  .953} \multicolumn{3}{p{31.285em}}{\textbf{MyComponent}} \\ \midrule
    \rowcolor[rgb]{ .949,  .949,  .949} \multicolumn{3}{p{31.285em}}{\textbf{Descripción}} \\ \midrule
    \multicolumn{3}{p{31.285em}}{Interfaz que modela los atributos genéricos de un componente.} \\ \midrule
    \rowcolor[rgb]{ .906,  .902,  .902} \multicolumn{3}{p{31.285em}}{\textbf{Atributos propuestos}} \\ \midrule
    \textbf{id} & number & Identificador del componente\\
    \textbf{name} & string & Nombre del componente \\ 
    \textbf{description} & string & Descripción del componente \\
    \textbf{family} & string & Familia a la que pertenece \\
    \textbf{type} & string & Tipo de componente (CPU, genérico...) \\
    \textbf{years} & string & Rango de años en los que se utilizó \\
    \textbf{price} & string & Precio de venta del componente \\
    \textbf{devices} & string[] & Tipo de dispositivos en los que se usaba el componente (portátiles o de escritorio) \\
    \textbf{imgs} & string[] & Nombres de las imágenes del componente \\
    \textbf{period\_id} & number & Identificador del periodo al que pertenece el componente \\ \midrule
    \rowcolor[rgb]{ .906,  .902,  .902} \multicolumn{3}{p{31.285em}}{\textbf{Métodos propuestos}} \\ \midrule
    \multicolumn{3}{p{31.285em}}{-} \\ \bottomrule
    \end{tabular}%
\end{table}%

\begin{table}[H]
\vspace{-4mm}
  \centering
  \caption{Descripción de la clase Cpu}
    \begin{tabular}{p{8.645em}p{5em}p{15.5em}}
    \toprule
    \rowcolor[rgb]{ .851,  .886,  .953} \multicolumn{3}{p{31.285em}}{\textbf{Cpu}} \\ \midrule
    \rowcolor[rgb]{ .949,  .949,  .949} \multicolumn{3}{p{31.285em}}{\textbf{Descripción}} \\ \midrule
    \multicolumn{3}{p{31.285em}}{Clase que implementa la interfaz MyComponent. Modela una CPU, con sus atributos específicos correspondientes. } \\ \midrule
    \rowcolor[rgb]{ .906,  .902,  .902} \multicolumn{3}{p{31.285em}}{\textbf{Atributos propuestos}} \\ \midrule
    \textbf{program\_memory} & string & Memoria ROM de la CPU\\
    \textbf{ram\_memory} & string & Memoria RAM de la CPU\\
    \textbf{clockspeed} & string & Velocidad de reloj de la CPU\\ 
    \textbf{power} & string & Potencia de la CPU\\
    \textbf{wordsize} & string & Tamaño de palabra de la CPU\\
    \textbf{passmark} & number & Passmark de la CPU\\
    \textbf{transistors} & number & Número de transistores de la CPU\\ \midrule
    \rowcolor[rgb]{ .906,  .902,  .902} \multicolumn{3}{p{31.285em}}{\textbf{Métodos propuestos}} \\ \midrule
    \multicolumn{3}{p{31.285em}}{-} \\ \bottomrule
    \end{tabular}%
\end{table}%

\subsubsection{Museo}

\begin{table}[H]
\vspace{-4mm}
  \centering
  \caption{Descripción de la clase PeriodService (museo)}
    \begin{tabular}{p{8.645em}p{5em}p{15.5em}}
    \toprule
    \rowcolor[rgb]{ .851,  .886,  .953} \multicolumn{3}{p{31.285em}}{\textbf{PeriodService}} \\ \midrule
    \rowcolor[rgb]{ .949,  .949,  .949} \multicolumn{3}{p{31.285em}}{\textbf{Descripción}} \\ \midrule
    \multicolumn{3}{p{31.285em}}{Servicio que conecta con el back-end de la aplicación para realizar las operaciones relacionadas con los periodos.} \\ \midrule
    \rowcolor[rgb]{ .906,  .902,  .902} \multicolumn{3}{p{31.285em}}{\textbf{Atributos propuestos}} \\ \midrule
    \multicolumn{3}{p{31.285em}}{-} \\ \midrule
    \rowcolor[rgb]{ .906,  .902,  .902} \multicolumn{3}{p{31.285em}}{\textbf{Métodos propuestos}} \\ \midrule
    \textbf{getPeriods} & \multicolumn{2}{p{22.64em}}{Devuelve todos los periodos existentes.} \\ 
    \textbf{getPeriodName} & \multicolumn{2}{p{22.64em}}{Devuelve el nombre del periodo correspondiente al identificador pasado por parámetro.} \\ 
    \textbf{getPeriod} & \multicolumn{2}{p{22.64em}}{Devuelve el periodo cuyo identificador se pasa como parámetro.} \\ \bottomrule
    \end{tabular}%
\end{table}%

\begin{table}[H]
\vspace{-4mm}
  \centering
  \caption{Descripción de la clase CompService (museo)}
    \begin{tabular}{p{10em}p{5em}p{14.5em}}
    \toprule
    \rowcolor[rgb]{ .851,  .886,  .953} \multicolumn{3}{p{31.285em}}{\textbf{CompService}} \\ \midrule
    \rowcolor[rgb]{ .949,  .949,  .949} \multicolumn{3}{p{31.285em}}{\textbf{Descripción}} \\ \midrule
    \multicolumn{3}{p{31.285em}}{Servicio que conecta con el back-end de la aplicación para realizar las operaciones relacionadas con los componentes.} \\ \midrule
    \rowcolor[rgb]{ .906,  .902,  .902} \multicolumn{3}{p{31.285em}}{\textbf{Atributos propuestos}} \\ \midrule
    \multicolumn{3}{p{31.285em}}{-} \\ \midrule
    \rowcolor[rgb]{ .906,  .902,  .902} \multicolumn{3}{p{31.285em}}{\textbf{Métodos propuestos}} \\ \midrule
    \textbf{getCompsFromPeriod} & \multicolumn{2}{p{19.64em}}{Devuelve los componentes pertenecientes al periodo cuyo identificador se pasa como parámetro.} \\ 
    \textbf{getComponent} & \multicolumn{2}{p{19.64em}}{Devuelve el componente cuyo identificador se pasa como parámetro.} \\ \bottomrule
    \end{tabular}%
\end{table}%

\begin{table}[H]
\vspace{-4mm}
  \centering
  \caption{Descripción de la clase TimelineComponent}
    \begin{tabular}{p{8.645em}p{5em}p{15.5em}}
    \toprule
    \rowcolor[rgb]{ .851,  .886,  .953} \multicolumn{3}{p{31.285em}}{\textbf{TimelineComponent}} \\ \midrule
    \rowcolor[rgb]{ .949,  .949,  .949} \multicolumn{3}{p{31.285em}}{\textbf{Descripción}} \\ \midrule
    \multicolumn{3}{p{31.285em}}{Clase asociada a la vista \ref{iu:timeline}.} \\ \midrule
    \rowcolor[rgb]{ .906,  .902,  .902} \multicolumn{3}{p{31.285em}}{\textbf{Atributos propuestos}} \\ \midrule
    \textbf{periods} & Period[] & Listado de todos los periodos existentes. \\ \midrule
    \rowcolor[rgb]{ .906,  .902,  .902} \multicolumn{3}{p{31.285em}}{\textbf{Métodos propuestos}} \\ \midrule
    \textbf{getPeriods} & \multicolumn{2}{p{22.64em}}{Obtiene los periodos y los asigna a \textit{periods}.} \\ 
    \textbf{search} & \multicolumn{2}{p{22.64em}}{Filtra los periodos según el texto introducido en la búsqueda.} \\ \bottomrule
    \end{tabular}%
\end{table}%

\begin{table}[H]
\vspace{-4mm}
  \centering
  \caption{Descripción de la clase PeriodDetailsComponent}
    \begin{tabular}{p{8.645em}p{7em}p{13.5em}}
    \toprule
    \rowcolor[rgb]{ .851,  .886,  .953} \multicolumn{3}{p{31.285em}}{\textbf{PeriodDetailsComponent}} \\ \midrule
    \rowcolor[rgb]{ .949,  .949,  .949} \multicolumn{3}{p{31.285em}}{\textbf{Descripción}} \\ \midrule
    \multicolumn{3}{p{31.285em}}{Clase asociada a la vista \ref{iu:period-details}.} \\ \midrule
    \rowcolor[rgb]{ .906,  .902,  .902} \multicolumn{3}{p{31.285em}}{\textbf{Atributos propuestos}} \\ \midrule
    \textbf{period} & Period & Periodo del que se muestran los detalles. \\ 
    \textbf{components} & MyComponent[] & Componentes pertenecientes al periodo. \\ \midrule
    \rowcolor[rgb]{ .906,  .902,  .902} \multicolumn{3}{p{31.285em}}{\textbf{Métodos propuestos}} \\ \midrule
    \textbf{getPeriod} & \multicolumn{2}{p{22.64em}}{Obtiene el periodo y lo asigna a \textit{period}.} \\ 
    \textbf{getComponents} & \multicolumn{2}{p{22.64em}}{Obtiene los componentes del periodo y los asigna a \textit{components}.} \\ \bottomrule
    \end{tabular}%
\end{table}%

\begin{table}[H]
\vspace{-4mm}
  \centering
  \caption{Descripción de la clase CompDetailsComponent}
    \begin{tabular}{p{8.645em}p{7em}p{13.5em}}
    \toprule
    \rowcolor[rgb]{ .851,  .886,  .953} \multicolumn{3}{p{31.285em}}{\textbf{CompDetailsComponent}} \\ \midrule
    \rowcolor[rgb]{ .949,  .949,  .949} \multicolumn{3}{p{31.285em}}{\textbf{Descripción}} \\ \midrule
    \multicolumn{3}{p{31.285em}}{Clase asociada a la vista \ref{iu:comp-details}.} \\ \midrule
    \rowcolor[rgb]{ .906,  .902,  .902} \multicolumn{3}{p{31.285em}}{\textbf{Atributos propuestos}} \\ \midrule
    \textbf{component} & MyComponent & Componente del que se muestran los detalles. \\ 
    \textbf{periodName} & string & Nombre del periodo al que pertenece el componente. \\ 
    \textbf{components} & MyComponent[] & Otros componentes pertenecientes al periodo. \\ \midrule
    \rowcolor[rgb]{ .906,  .902,  .902} \multicolumn{3}{p{31.285em}}{\textbf{Métodos propuestos}} \\ \midrule
    \multicolumn{1}{p{10.2em}}{\textbf{getComponent}} & \multicolumn{2}{p{19.64em}}{Obtiene el componente y lo asigna a \textit{component}.} \\ 
    \multicolumn{1}{p{10.2em}}{\textbf{getPeriodName}} & \multicolumn{2}{p{19.64em}}{Obtiene el nombre del periodo y lo asigna a \textit{periodName}.} \\ 
    \multicolumn{1}{p{10.2em}}{\textbf{getPeriodComponents}} & \multicolumn{2}{p{19.64em}}{Obtiene los componentes del periodo y los asigna a \textit{components}.} \\ \bottomrule
    \end{tabular}%
\end{table}%


\subsubsection{Administración del museo}

\begin{table}[H]
\vspace{-4mm}
  \centering
  \caption{Descripción de la clase UserService }
    \begin{tabular}{p{8.645em}p{5em}p{15.5em}}
    \toprule
    \rowcolor[rgb]{ .851,  .886,  .953} \multicolumn{3}{p{31.285em}}{\textbf{UserService}} \\ \midrule
    \rowcolor[rgb]{ .949,  .949,  .949} \multicolumn{3}{p{31.285em}}{\textbf{Descripción}} \\ \midrule
    \multicolumn{3}{p{31.285em}}{Servicio que conecta con el back-end de la aplicación para realizar las operaciones relacionadas con el usuario administrador.} \\ \midrule
    \rowcolor[rgb]{ .906,  .902,  .902} \multicolumn{3}{p{31.285em}}{\textbf{Atributos propuestos}} \\ \midrule
    \multicolumn{3}{p{31.285em}}{-} \\ \midrule
    \rowcolor[rgb]{ .906,  .902,  .902} \multicolumn{3}{p{31.285em}}{\textbf{Métodos propuestos}} \\ \midrule
    \textbf{login} & \multicolumn{2}{p{22.64em}}{Comprueba si el usuario y la contraseña introducidos se corresponden con los existentes en la base de datos.} \\ \bottomrule
    \end{tabular}%
\end{table}%

\begin{table}[H]
\vspace{-4mm}
  \centering
  \caption{Descripción de la clase PeriodService (administración)}
    \begin{tabular}{p{8.645em}p{5em}p{15.5em}}
    \toprule
    \rowcolor[rgb]{ .851,  .886,  .953} \multicolumn{3}{p{31.285em}}{\textbf{PeriodService}} \\ \midrule
    \rowcolor[rgb]{ .949,  .949,  .949} \multicolumn{3}{p{31.285em}}{\textbf{Descripción}} \\ \midrule
    \multicolumn{3}{p{31.285em}}{Servicio que conecta con el back-end de la aplicación para realizar las operaciones relacionadas con los periodos.} \\ \midrule
    \rowcolor[rgb]{ .906,  .902,  .902} \multicolumn{3}{p{31.285em}}{\textbf{Atributos propuestos}} \\ \midrule
    \multicolumn{3}{p{31.285em}}{-} \\ \midrule
    \rowcolor[rgb]{ .906,  .902,  .902} \multicolumn{3}{p{31.285em}}{\textbf{Métodos propuestos}} \\ \midrule
    \textbf{getPeriods} & \multicolumn{2}{p{22.64em}}{Devuelve todos los periodos existentes.} \\ 
    \textbf{getPeriod} & \multicolumn{2}{p{22.64em}}{Devuelve el periodo cuyo identificador se pasa como parámetro.} \\ 
    \textbf{addPeriod} & \multicolumn{2}{p{22.64em}}{Añade el periodo pasado como parámetro a la base de datos.} \\ 
    \textbf{updatePeriod} & \multicolumn{2}{p{22.64em}}{Actualiza el periodo pasado como parámetro en la base de datos.} \\ 
    \textbf{getPeriod} & \multicolumn{2}{p{22.64em}}{Elimina de la base de datos el periodo cuyo identificador se pasa como parámetro.} \\ \bottomrule
    \end{tabular}%
\end{table}%

\begin{table}[H]
\vspace{-4mm}
  \centering
  \caption{Descripción de la clase CompService (administración)}
    \begin{tabular}{p{10em}p{5em}p{14.5em}}
    \toprule
    \rowcolor[rgb]{ .851,  .886,  .953} \multicolumn{3}{p{31.285em}}{\textbf{CompService}} \\ \midrule
    \rowcolor[rgb]{ .949,  .949,  .949} \multicolumn{3}{p{31.285em}}{\textbf{Descripción}} \\ \midrule
    \multicolumn{3}{p{31.285em}}{Servicio que conecta con el back-end de la aplicación para realizar las operaciones relacionadas con los componentes.} \\ \midrule
    \rowcolor[rgb]{ .906,  .902,  .902} \multicolumn{3}{p{31.285em}}{\textbf{Atributos propuestos}} \\ \midrule
    \multicolumn{3}{p{31.285em}}{-} \\ \midrule
    \rowcolor[rgb]{ .906,  .902,  .902} \multicolumn{3}{p{31.285em}}{\textbf{Métodos propuestos}} \\ \midrule
    \textbf{getCompsFromPeriod} & \multicolumn{2}{p{19.64em}}{Devuelve los componentes pertenecientes al periodo cuyo identificador se pasa como parámetro.} \\ 
    \textbf{getComponent} & \multicolumn{2}{p{19.64em}}{Devuelve el componente cuyo identificador se pasa como parámetro.} \\ 
    \textbf{addComponent} & \multicolumn{2}{p{19.64em}}{Añade el componente pasado como parámetro a la base de datos.} \\ 
    \textbf{updateComponent} & \multicolumn{2}{p{19.64em}}{Actualiza el componente pasado como parámetro en la base de datos.} \\ 
    \textbf{getComponent} & \multicolumn{2}{p{19.64em}}{Elimina de la base de datos el componente cuyo identificador se pasa como parámetro.} \\ \bottomrule
    \end{tabular}%
\end{table}%

\begin{table}[H]
\vspace{-4mm}
  \centering
  \caption{Descripción de la clase LoginComponent}
    \begin{tabular}{p{8.645em}p{5em}p{15.5em}}
    \toprule
    \rowcolor[rgb]{ .851,  .886,  .953} \multicolumn{3}{p{31.285em}}{\textbf{LoginComponent}} \\ \midrule
    \rowcolor[rgb]{ .949,  .949,  .949} \multicolumn{3}{p{31.285em}}{\textbf{Descripción}} \\ \midrule
    \multicolumn{3}{p{31.285em}}{Clase asociada a la vista \ref{iu:login}.} \\ \midrule
    \rowcolor[rgb]{ .906,  .902,  .902} \multicolumn{3}{p{31.285em}}{\textbf{Atributos propuestos}} \\ \midrule
    \multicolumn{3}{p{31.285em}}{-} \\ \midrule
    \rowcolor[rgb]{ .906,  .902,  .902} \multicolumn{3}{p{31.285em}}{\textbf{Métodos propuestos}} \\ \midrule
    \textbf{login} & \multicolumn{2}{p{22.64em}}{Comprueba los datos introducidos para iniciar sesión.} \\ \bottomrule
    \end{tabular}%
\end{table}%

\begin{table}[H]
\vspace{-4mm}
  \centering
  \caption{Descripción de la clase ListPeriodsComponent}
    \begin{tabular}{p{8.645em}p{5em}p{15.5em}}
    \toprule
    \rowcolor[rgb]{ .851,  .886,  .953} \multicolumn{3}{p{31.285em}}{\textbf{ListPeriodsComponent}} \\ \midrule
    \rowcolor[rgb]{ .949,  .949,  .949} \multicolumn{3}{p{31.285em}}{\textbf{Descripción}} \\ \midrule
    \multicolumn{3}{p{31.285em}}{Clase asociada a la vista \ref{iu:list-periods}.} \\ \midrule
    \rowcolor[rgb]{ .906,  .902,  .902} \multicolumn{3}{p{31.285em}}{\textbf{Atributos propuestos}} \\ \midrule
    \textbf{periods} & Period[] & Listado de todos los periodos existentes. \\ \midrule
    \rowcolor[rgb]{ .906,  .902,  .902} \multicolumn{3}{p{31.285em}}{\textbf{Métodos propuestos}} \\ \midrule
    \textbf{getPeriods} & \multicolumn{2}{p{22.64em}}{Obtiene los periodos y los asigna a \textit{periods}.} \\ 
    \textbf{deletePeriod} & \multicolumn{2}{p{22.64em}}{Elimina el periodo seleccionado.} \\ \bottomrule
    \end{tabular}%
\end{table}%

\begin{table}[H]
\vspace{-4mm}
  \centering
  \caption{Descripción de la clase PeriodComponent}
    \begin{tabular}{p{8.645em}p{7em}p{13.5em}}
    \toprule
    \rowcolor[rgb]{ .851,  .886,  .953} \multicolumn{3}{p{31.285em}}{\textbf{PeriodComponent}} \\ \midrule
    \rowcolor[rgb]{ .949,  .949,  .949} \multicolumn{3}{p{31.285em}}{\textbf{Descripción}} \\ \midrule
    \multicolumn{3}{p{31.285em}}{Clase asociada a la vista \ref{iu:period}.} \\ \midrule
    \rowcolor[rgb]{ .906,  .902,  .902} \multicolumn{3}{p{31.285em}}{\textbf{Atributos propuestos}} \\ \midrule
    \multicolumn{1}{p{8.645em}}{\textbf{period}} & Period & Periodo del que se muestran los detalles. \\ 
    \multicolumn{1}{p{8.645em}}{\textbf{components}} & MyComponent[] & Componentes pertenecientes al periodo. \\ \midrule
    \rowcolor[rgb]{ .906,  .902,  .902} \multicolumn{3}{p{31.285em}}{\textbf{Métodos propuestos}} \\ \midrule
    \multicolumn{1}{p{13.2em}}{\textbf{getPeriod}} & \multicolumn{2}{p{16.64em}}{Obtiene el periodo y lo asigna a \textit{period}.} \\ 
    \multicolumn{1}{p{13.2em}}{\textbf{getComponentsFromPeriod}} & \multicolumn{2}{p{16.64em}}{Obtiene los componentes del periodo y los asigna a \textit{components}.} \\ 
    \multicolumn{1}{p{13.2em}}{\textbf{deleteComponent}} & \multicolumn{2}{p{16.64em}}{Elimina el componente seleccionado.} \\ \bottomrule
    \end{tabular}%
\end{table}%

\begin{table}[H]
\vspace{-4mm}
  \centering
  \caption{Descripción de la clase AddPeriodComponent}
    \begin{tabular}{p{8.645em}p{7em}p{13.5em}}
    \toprule
    \rowcolor[rgb]{ .851,  .886,  .953} \multicolumn{3}{p{31.285em}}{\textbf{AddPeriodComponent}} \\ \midrule
    \rowcolor[rgb]{ .949,  .949,  .949} \multicolumn{3}{p{31.285em}}{\textbf{Descripción}} \\ \midrule
    \multicolumn{3}{p{31.285em}}{Clase asociada a la vista \ref{iu:add-period}.} \\ \midrule
    \rowcolor[rgb]{ .906,  .902,  .902} \multicolumn{3}{p{31.285em}}{\textbf{Atributos propuestos}} \\ \midrule
    \textbf{model} & Period & Periodo asociado al formulario en el que se introducen los datos. \\ \midrule
    \rowcolor[rgb]{ .906,  .902,  .902} \multicolumn{3}{p{31.285em}}{\textbf{Métodos propuestos}} \\ \midrule
    \textbf{submit} & \multicolumn{2}{p{22.64em}}{Añade el periodo con los datos introducidos en el formulario.} \\ \bottomrule
    \end{tabular}%
\end{table}%

\begin{table}[H]
\vspace{-4mm}
  \centering
  \caption{Descripción de la clase EditPeriodComponent}
    \begin{tabular}{p{8.645em}p{7em}p{13.5em}}
    \toprule
    \rowcolor[rgb]{ .851,  .886,  .953} \multicolumn{3}{p{31.285em}}{\textbf{EditPeriodComponent}} \\ \midrule
    \rowcolor[rgb]{ .949,  .949,  .949} \multicolumn{3}{p{31.285em}}{\textbf{Descripción}} \\ \midrule
    \multicolumn{3}{p{31.285em}}{Clase asociada a la vista \ref{iu:add-period}.} \\ \midrule
    \rowcolor[rgb]{ .906,  .902,  .902} \multicolumn{3}{p{31.285em}}{\textbf{Atributos propuestos}} \\ \midrule
    \textbf{period} & Period & Periodo que se va a editar, con los datos iniciales. \\ 
    \textbf{model} & Period & Periodo asociado al formulario en el que se editan los datos. \\ \midrule
    \rowcolor[rgb]{ .906,  .902,  .902} \multicolumn{3}{p{31.285em}}{\textbf{Métodos propuestos}} \\ \midrule
    \textbf{getPeriod} & \multicolumn{2}{p{22.64em}}{Obtiene el periodo y lo asigna a \textit{period}.} \\ 
    \textbf{submit} & \multicolumn{2}{p{22.64em}}{Actualiza el periodo con los datos introducidos en el formulario.} \\  \bottomrule
    \end{tabular}%
\end{table}%

\begin{table}[H]
\vspace{-4mm}
  \centering
  \caption{Descripción de la clase MyCompComponent}
    \begin{tabular}{p{8.645em}p{7em}p{13.5em}}
    \toprule
    \rowcolor[rgb]{ .851,  .886,  .953} \multicolumn{3}{p{31.285em}}{\textbf{MyCompComponent}} \\ \midrule
    \rowcolor[rgb]{ .949,  .949,  .949} \multicolumn{3}{p{31.285em}}{\textbf{Descripción}} \\ \midrule
    \multicolumn{3}{p{31.285em}}{Clase asociada a la vista \ref{iu:my-comp}.} \\ \midrule
    \rowcolor[rgb]{ .906,  .902,  .902} \multicolumn{3}{p{31.285em}}{\textbf{Atributos propuestos}} \\ \midrule
    \textbf{component} & MyComponent & Componente del que se muestran los detalles. \\ \midrule
    \rowcolor[rgb]{ .906,  .902,  .902} \multicolumn{3}{p{31.285em}}{\textbf{Métodos propuestos}} \\ \midrule
    \textbf{getComponent} & \multicolumn{2}{p{19.64em}}{Obtiene el componente y lo asigna a \textit{component}.} \\ \bottomrule
    \end{tabular}%
\end{table}%

\begin{table}[H]
\vspace{-4mm}
  \centering
  \caption{Descripción de la clase AddCompComponent}
    \begin{tabular}{p{8.645em}p{7em}p{13.5em}}
    \toprule
    \rowcolor[rgb]{ .851,  .886,  .953} \multicolumn{3}{p{31.285em}}{\textbf{AddCompComponent}} \\ \midrule
    \rowcolor[rgb]{ .949,  .949,  .949} \multicolumn{3}{p{31.285em}}{\textbf{Descripción}} \\ \midrule
    \multicolumn{3}{p{31.285em}}{Clase asociada a la vista \ref{iu:add-comp}.} \\ \midrule
    \rowcolor[rgb]{ .906,  .902,  .902} \multicolumn{3}{p{31.285em}}{\textbf{Atributos propuestos}} \\ \midrule
    \textbf{model} & MyComponent & Componente asociado al formulario en el que se introducen los datos. \\ 
    \textbf{periods} & Period[] & Listado de periodos existentes. \\ \midrule
    \rowcolor[rgb]{ .906,  .902,  .902} \multicolumn{3}{p{31.285em}}{\textbf{Métodos propuestos}} \\ \midrule
    \textbf{getPeriods} & \multicolumn{2}{p{22.64em}}{Obtiene los periodos y los asigna a \textit{periods}.} \\ 
    \textbf{submit} & \multicolumn{2}{p{22.64em}}{Añade el componente con los datos introducidos en el formulario.} \\ \bottomrule
    \end{tabular}%
\end{table}%

\begin{table}[H]
\vspace{-4mm}
  \centering
  \caption{Descripción de la clase EditCompComponent}
    \begin{tabular}{p{8.645em}p{7em}p{13.5em}}
    \toprule
    \rowcolor[rgb]{ .851,  .886,  .953} \multicolumn{3}{p{31.285em}}{\textbf{EditCompComponent}} \\ \midrule
    \rowcolor[rgb]{ .949,  .949,  .949} \multicolumn{3}{p{31.285em}}{\textbf{Descripción}} \\ \midrule
    \multicolumn{3}{p{31.285em}}{Clase asociada a la vista \ref{iu:add-comp}.} \\ \midrule
    \rowcolor[rgb]{ .906,  .902,  .902} \multicolumn{3}{p{31.285em}}{\textbf{Atributos propuestos}} \\ \midrule
    \textbf{comp} & MyComponent & Componente que se va a editar, con los datos iniciales. \\ 
    \textbf{model} & MyComponent & Componente asociado al formulario en el que se editan los datos. \\ 
    \textbf{periods} & Period[] & Listado de periodos existentes. \\ \midrule
    \rowcolor[rgb]{ .906,  .902,  .902} \multicolumn{3}{p{31.285em}}{\textbf{Métodos propuestos}} \\ \midrule
    \textbf{getComponent} & \multicolumn{2}{p{22.64em}}{Obtiene el componente y lo asigna a \textit{comp}.} \\ 
    \textbf{getPeriods} & \multicolumn{2}{p{22.64em}}{Obtiene los periodos y los asigna a \textit{periods}.} \\ 
    \textbf{submit} & \multicolumn{2}{p{22.64em}}{Actualiza el componente con los datos introducidos en el formulario.} \\  \bottomrule
    \end{tabular}%
\end{table}%



\newpage
\section{ASI 8: DEFINICIÓN DE INTERFACES DE USUARIO}

%\subsection{Descripción de la Interfaz} 

\subsection{Definición del aspecto de la interfaz}

\subsubsection{Museo}
A continuación se presentan los prototipos de interfaces diseñados para la página web del museo. Todas ellas tienen en común la barra de navegación, que contiene el logo de la EII, un enlace a la vista general del museo y un selector de idioma.
\paragraph*{Inicio}
En la página de inicio del museo se muestra un mensaje de bienvenida y un botón que conduce a la vista general del museo.
\begin{figure}[H]
\centering
\includegraphics[scale=0.45]{homeIU}
\caption{Prototipo: Página de inicio}
\end{figure}
\paragraph*{Vista general del museo}\label{iu:timeline}
En la vista general del museo encontramos un menú lateral con filtros de búsqueda, y una sección principal que contiene una línea temporal con los periodos en los que se divide la historia de las CPUs.
\begin{figure}[H]
\centering
\includegraphics[scale=0.45]{museoIU}
\caption{Prototipo: Página de la vista general del museo}
\end{figure}
\paragraph*{Detalles del periodo}\label{iu:period-details}
En esta página hay un menú para volver a la vista general, y se muestran todos los detalles de un periodo (nombre, características, sistemas famosos de dicho periodo, etc.) y los componentes que pertenecen al mismo. 
\begin{figure}[H]
\centering
\includegraphics[scale=0.45]{periodoIU}
\caption{Prototipo: Página de detalles del periodo (museo)}
\end{figure}
\paragraph*{Detalles del componente}\label{iu:comp-details}
En esta página se muestra una galería de fotos del componente, la descripción del mismo, y un listado de características. En el menú de esta página hay un listado de componentes pertenecientes al mismo periodo.
\begin{figure}[H]
\centering
\includegraphics[scale=0.45]{piezaIU}
\caption{Prototipo: Página de detalles del componente (museo)}
\end{figure}

\subsubsection{Administración del museo}
A continuación, se muestran los prototipos inciales para la aplicación de administración del museo. En todas ellas, salvo en la de inicio de sesión, hay un menú lateral de navegación. 
\paragraph*{Iniciar sesión}\label{iu:login}
En esta página el administrador del sistema deberá introducir su usuario y contraseña para acceder al mismo.
\begin{figure}[H]
\centering
\includegraphics[scale=0.55]{loginIU}
\caption{Prototipo: Página de inicio de sesión}
\end{figure}
\paragraph*{Listado de periodos}\label{iu:list-periods}
En esta página se muestra un listado de los periodos existentes, con las opciones de acceder a cada uno, editarlo o eliminarlo.
\begin{figure}[H]
\centering
\includegraphics[scale=0.45]{listadoPeriodosIU}
\caption{Prototipo: Página de listado de periodos}
\end{figure}
\paragraph*{Periodo}\label{iu:period}
Esta página contiene los detalles de un periodo así como un listado de los componentes pertenecientes al mismo, ofreciendo la opción de acceder a ellos, editarlos o eliminarlos.
\begin{figure}[H]
\centering
\includegraphics[scale=0.45]{periodoIU2}
\caption{Prototipo: Página de detalles de un periodo (admimnistración)}
\end{figure}
\paragraph*{Componente}\label{iu:my-comp}
En esta página se muestran los detalles correspondientes al componente.
\begin{figure}[H]
\centering
\includegraphics[scale=0.45]{compIU}
\caption{Prototipo: Página de detalles de un componente (admimnistración)}
\end{figure}
\paragraph*{Añadir/editar periodo}\label{iu:add-period}
Los formularios para añadir o editar un periodo son idénticos, con la única diferencia de que el formulario para editar ya tiene los campos completados con los valores existentes del periodo, por tanto solo se muestra una captura representando ambos.
\begin{figure}[H]
\centering
\includegraphics[scale=0.45]{añadirPeriodoIU}
\caption{Prototipo: Formulario para añadir o editar un periodo}
\end{figure}
\paragraph*{Añadir/editar componente}\label{iu:add-comp}
Con los formularios para añadir o editar un componente ocurre igual que con los del periodo ya mencionados.
\begin{figure}[H]
\centering
\includegraphics[scale=0.45]{añadirCompIU}
\caption{Prototipo: Formulario para añadir o editar un componente}
\end{figure}



%\subsection{Descripción del Comportamiento de la Interfaz} 

\subsection{Diagrama de Navegabilidad}
A continuación se presentan dos diagramas de navegabilidad, correspondientes a las dos aplicaciones web que constituyen el sistema.
\subsubsection{Museo}
\begin{figure}[H]
\centering
\includegraphics[scale=0.7]{nav-museo}
\caption{Diagrama de navegabilidad del museo}
\end{figure}

\subsubsection{Administración del museo}
\begin{figure}[H]
\centering
\centerline{\includegraphics[scale=0.7]{nav-admin}}
\caption{Diagrama de navegabilidad de la administración del museo}
\end{figure}


\newpage
\section{ASI 10: ESPECIFICACIÓN DEL PLAN DE PRUEBAS}

\subsection{Pruebas unitarias}
Se realizarán pruebas unitarias del sistema utilizando Jasmine y Karma, herramientas incluidas en Angular para probar el correcto funcionamiento de los diferentes componentes.
\begin{table}[H]
%\vspace{-4mm}
  \centering
  \caption{Pruebas unitarias: Caso de uso 1}
    \begin{tabular}{p{9em}p{27em}}
    \toprule
    \rowcolor[rgb]{ .851,  .886,  .953} \multicolumn{2}{p{36em}}{\textbf{Caso de uso 1: Consultar periodos (museo)}} \\ \midrule
    \rowcolor[rgb]{ .949,  .949,  .949} \textbf{Prueba} &  \textbf{Resultado esperado}\\ \midrule
    \textbf{Obtener periodos existentes} & El sistema devolverá una lista de todos los periodos existentes. \\ \midrule
    \textbf{Obtener periodo por nombre} & El sistema devolverá una lista de los periodos cuyo nombre contenga el texto introducido. \\ \midrule
    \textbf{Obtener periodo por años} & El sistema devolverá una lista de los periodos cuyos años coincidan con los introducidos.\\ \bottomrule
    \end{tabular}%
\end{table}%
\begin{table}[H]
\vspace{-4mm}
  \centering
  \caption{Pruebas unitarias: Caso de uso 2}
    \begin{tabular}{p{11em}p{25em}}
    \toprule
    \rowcolor[rgb]{ .851,  .886,  .953} \multicolumn{2}{p{36em}}{\textbf{Caso de uso 2: Consultar componentes (museo)}} \\ \midrule
    \rowcolor[rgb]{ .949,  .949,  .949} \textbf{Prueba} &  \textbf{Resultado esperado}\\ \midrule
    \textbf{Obtener componentes de un periodo} & El sistema devolverá una lista de todos los componentes pertenecientes al periodo. \\ \bottomrule
    \end{tabular}%
\end{table}%
\begin{table}[H]
\vspace{-4mm}
  \centering
  \caption{Pruebas unitarias: Caso de uso 3}
    \begin{tabular}{p{9em}p{27em}}
    \toprule
    \rowcolor[rgb]{ .851,  .886,  .953} \multicolumn{2}{p{36em}}{\textbf{Caso de uso 3: Iniciar sesión}} \\ \midrule
    \rowcolor[rgb]{ .949,  .949,  .949} \textbf{Prueba} &  \textbf{Resultado esperado}\\ \midrule
    \textbf{Iniciar sesión con datos válidos } & El sistema permitirá el acceso a la página de administración. \\ \midrule
    \textbf{Iniciar sesión con datos incorrectos} & El sistema no permitirá el acceso y se mostrará un error. \\ \bottomrule
    \end{tabular}%
\end{table}%
\begin{table}[H]
\vspace{-4mm}
  \centering
  \caption{Pruebas unitarias: Caso de uso 4}
    \begin{tabular}{p{9em}p{27em}}
    \toprule
    \rowcolor[rgb]{ .851,  .886,  .953} \multicolumn{2}{p{36em}}{\textbf{Caso de uso 4: Consultar periodos (administración)}} \\ \midrule
    \rowcolor[rgb]{ .949,  .949,  .949} \textbf{Prueba} &  \textbf{Resultado esperado}\\ \midrule
    \textbf{Obtener periodos existentes} & El sistema devolverá una lista de todos los periodos existentes. \\ \bottomrule
    \end{tabular}%
\end{table}%
\begin{table}[H]
\vspace{-4mm}
  \centering
  \caption{Pruebas unitarias: Caso de uso 5}
    \begin{tabular}{p{11em}p{25em}}
    \toprule
    \rowcolor[rgb]{ .851,  .886,  .953} \multicolumn{2}{p{36em}}{\textbf{Caso de uso 5: Añadir periodo}} \\ \midrule
    \rowcolor[rgb]{ .949,  .949,  .949} \textbf{Prueba} &  \textbf{Resultado esperado}\\ \midrule
    \textbf{Añadir nuevo periodo} & El sistema tendrá un periodo más. \\ \midrule
    \textbf{Añadir periodo que ya existe} & El sistema no añadirá el periodo y responderá con un error.  \\ \midrule
    \textbf{Añadir periodo con campos vacíos} & El sistema no añadirá el periodo y responderá con un error. \\ \bottomrule
    \end{tabular}%
\end{table}%
\begin{table}[H]
\vspace{-4mm}
  \centering
  \caption{Pruebas unitarias: Caso de uso 6}
    \begin{tabular}{p{11em}p{25em}}
    \toprule
    \rowcolor[rgb]{ .851,  .886,  .953} \multicolumn{2}{p{36em}}{\textbf{Caso de uso 6: Modificar periodo}} \\ \midrule
    \rowcolor[rgb]{ .949,  .949,  .949} \textbf{Prueba} &  \textbf{Resultado esperado}\\ \midrule
    \textbf{Modificar periodo existente} & El sistema actualizará los datos del periodo. \\ \midrule
    \textbf{Modificar un periodo que no existe} & El sistemá responderá con un error. \\ \midrule
    \textbf{Modificar periodo dejando campos vacíos} & El sistema no actualizará el periodo y responderá con un error. \\ \bottomrule
    \end{tabular}%
\end{table}%
\begin{table}[H]
\vspace{-4mm}
  \centering
  \caption{Pruebas unitarias: Caso de uso 7}
    \begin{tabular}{p{11em}p{25em}}
    \toprule
    \rowcolor[rgb]{ .851,  .886,  .953} \multicolumn{2}{p{36em}}{\textbf{Caso de uso 7: Eliminar periodo}} \\ \midrule
    \rowcolor[rgb]{ .949,  .949,  .949} \textbf{Prueba} &  \textbf{Resultado esperado}\\ \midrule
    \textbf{Eliminar un periodo existente} & El sistema tendrá un periodo menos. \\ \midrule
    \textbf{Eliminar un periodo que no existe} & El sistema responderá con un error. \\ \bottomrule
    \end{tabular}%
\end{table}%
\begin{table}[H]
\vspace{-4mm}
  \centering
  \caption{Pruebas unitarias: Caso de uso 8}
    \begin{tabular}{p{11em}p{25em}}
    \toprule
    \rowcolor[rgb]{ .851,  .886,  .953} \multicolumn{2}{p{36em}}{\textbf{Caso de uso 8: Consultar componentes (administración)}} \\ \midrule
    \rowcolor[rgb]{ .949,  .949,  .949} \textbf{Prueba} &  \textbf{Resultado esperado}\\ \midrule
    \textbf{Obtener componentes de un periodo} & El sistema devolverá una lista de todos los componentes pertenecientes al periodo. \\ \bottomrule
    \end{tabular}%
\end{table}%
\begin{table}[H]
\vspace{-4mm}
  \centering
  \caption{Pruebas unitarias: Caso de uso 9}
    \begin{tabular}{p{13em}p{23em}}
    \toprule
    \rowcolor[rgb]{ .851,  .886,  .953} \multicolumn{2}{p{36em}}{\textbf{Caso de uso 9: Añadir componente}} \\ \midrule
    \rowcolor[rgb]{ .949,  .949,  .949} \textbf{Prueba} &  \textbf{Resultado esperado}\\ \midrule
    \textbf{Añadir nuevo componente} & El sistema tendrá un componente más. \\ \midrule
    \textbf{Añadir componente que ya existe} & El sistema no añadirá el componente y responderá con un error.  \\ \midrule
    \textbf{Añadir componente a un periodo que no existe} & El sistema no añadirá el componente y responderá con un error.  \\ \midrule
    \textbf{Añadir componente con campos obligatorios vacíos} & El sistema no añadirá el componente y responderá con un error. \\ \bottomrule
    \end{tabular}%
\end{table}%
\begin{table}[H]
\vspace{-4mm}
  \centering
  \caption{Pruebas unitarias: Caso de uso 10}
    \begin{tabular}{p{13em}p{23em}}
    \toprule
    \rowcolor[rgb]{ .851,  .886,  .953} \multicolumn{2}{p{36em}}{\textbf{Caso de uso 10: Modificar componente}} \\ \midrule
    \rowcolor[rgb]{ .949,  .949,  .949} \textbf{Prueba} &  \textbf{Resultado esperado}\\ \midrule
    \textbf{Modificar componente existente} & El sistema actualizará los datos del componente. \\ \midrule
    \textbf{Modificar un componente que no existe} & El sistemá responderá con un error. \\ \midrule
    \textbf{Modificar componente dejando campos obligatorios vacíos} & El sistema no actualizará el componente y responderá con un error. \\ \bottomrule
    \end{tabular}%
\end{table}%
\begin{table}[H]
\vspace{-4mm}
  \centering
  \caption{Pruebas unitarias: Caso de uso 11}
    \begin{tabular}{p{13em}p{23em}}
    \toprule
    \rowcolor[rgb]{ .851,  .886,  .953} \multicolumn{2}{p{36em}}{\textbf{Caso de uso 11: Eliminar componente}} \\ \midrule
    \rowcolor[rgb]{ .949,  .949,  .949} \textbf{Prueba} &  \textbf{Resultado esperado}\\ \midrule
    \textbf{Eliminar un componente existente} & El sistema tendrá un componente menos. \\ \midrule
    \textbf{Eliminar un componente que no existe} & El sistema responderá con un error. \\ \bottomrule
    \end{tabular}%
\end{table}%

\subsection{Pruebas del sistema}
Se probará el conjunto completo del sistema, es decir, si la aplicación, el servidor y la base de datos se conectan de forma correcta, sin errores en tiempo de ejecución y obteniendo las respuestas esperadas. Estas pruebas se realizarán de forma manual sobre la aplicación.
\subsection{Pruebas de usabilidad}
Para realizar estas pruebas se hará uso de una serie de cuestionarios, cuyos resultados se obtendrán tras observar como usuarios con diferentes perfiles interactúan con el sistema.



\newpage
\chapter{DISEÑO DEL SISTEMA DE INFORMACIÓN}
	\vspace{2cm}	
	\begin{center}
	{\Large \textbf{FASE DE DESARROLLO} \par}
	\end{center}
	\vspace{5cm}
	
	\begin{center}
	\Huge \textbf{DSI}\par
	\end{center}

\newpage


\section{DSI 3: DISEÑO DE CASOS DE USO REALES}

\subsection{Caso de Uso 1.1} 

\subsubsection{Diagramas de Interacción (Comunicación y Secuencia)} 

\subsubsection{Diagramas de Estados de las Clases} 
 
\subsubsection{Diagramas de Actividades} 


\subsection{Caso de Uso 1.2}


\newpage
\section{DSI 4: DISEÑO DE CLASES}

\subsection{Diagrama de Clases}


\newpage
\section{DSI 5: DISEÑO DE LA ARQUITECTURA DE MÓDULOS DEL SISTEMA}

\subsection{DSI 5.1 Diseño de Módulos del Sistema}

\subsection{DSI 5.2 Diseño de Comunicaciones entre Módulos}

\subsection{DSI 5.3 Revisión de la Interfaz de Usuario}


\newpage
\section{DSI 6: DISEÑO FÍSICO DE DATOS}

\subsection{Descripción del SGBD Usado} 
Se ha creado una base de datos relacional, utilizando MySQL 8 como sistema gestor de bases de datos, debido a su gran popularidad en todo el mundo y, más concretamente, en entornos de desarrollo web.\\
\par {\color{red}explicar algo más}

\subsection{Integración del SGBD en Nuestro Sistema} 

\subsection{Diagrama E--R} 
\begin{figure}[H]
\centering
\includegraphics[scale=0.6]{diagrama_e-r}
\caption{Diagrama Entidad-Relación de la base de datos creada}
\end{figure}

\newpage
\section{DSI 9: DISEÑO DE LA MIGRACIÓN Y CARGA INICIAL DE DATOS}


\newpage
\section{DSI 10: ESPECIFICACIÓN TÉCNICA DEL PLAN DE PRUEBAS}

\subsection{Pruebas Unitarias} 

\subsection{Pruebas de Integración y del Sistema} 

\subsection{Pruebas de Usabilidad y Accesibilidad} 

\subsubsection{Diseño de Cuestionarios} 

\subsubsection{Actividades de las Pruebas de Usabilidad} 


\subsection{Pruebas de Accesibilidad} 

\subsection{Pruebas de Rendimiento} 

\newpage
\chapter{CONSTRUCCIÓN DEL SISTEMA DE INFORMACIÓN}
	\vspace{2cm}	
	\begin{center}
	{\Large \textbf{FASE DE DESARROLLO} \par}
	\end{center}
	\vspace{5cm}
	
	\begin{center}
	\Huge \textbf{CSI}\par
	\end{center}

\newpage


\section[CSI 1: PREPARACIÓN DEL ENTORNO DE GENERACIÓN Y \\ CONSTRUCCIÓN]{CSI 1: PREPARACIÓN DEL ENTORNO DE GENERACIÓN Y CONSTRUCCIÓN}

\subsection{Estándares y normas seguidos}
\subsubsection{Angular Style Guide}
La guía de estilos de Angular\cite{AngularSG} es un conjunto de recomendaciones sobre la sintaxis, estructura y convenciones de código en proyectos de Angular.
\subsubsection{HTML5}
HTML5 es la versión más reciente y la actualmente usada de HTML, y está estandarizada por el W3C (World Wide Web Consortium).
\subsubsection{CSS}
Hojas de estilos estandarizadas por el W3C.
\subsubsection{PHP Code Style Guide}
La guía de estilos de PHP\cite{PhpSG} contiene normas de código y buenas prácticas.

\subsection{Lenguajes de programación}
\subsubsection{TypeScript}
TypeScript es un lenguaje de programación de código abierto desarrollado por Microsoft. Extiende JavaScript añadiendo la definición de tipos estáticos.
\subsubsection{HTML}
HTML (HyperText Markup Language) es un lenguaje de marcado utilizado en la elaboración de páginas web.
\subsubsection{CSS}
CSS (Cascading Style Sheets) es un lenguaje de diseño gráfico que permite modificar la presentación de los elementos definidos en los documentos HTML.
\subsubsection{PHP}
PHP es un lenguaje de programación utilizado en el desarrollo web que es procesado en el lado del servidor.
\subsubsection{SQL}
SQL (Structured Query Language) es un lenguaje de consultas utilizado para leer, insertar, actualizar o eliminar datos de la base de datos relacional utilizada. 
\subsection{Herramientas y programas usados para el desarrollo}
\subsubsection{Visual Studio Code}
Visual Studio Code es un editor de código fuente desarrollado por Microsoft, gratuito y de código abierto. Tiene soporte integrado para TypeScript y Node.js, extensiones para otros lenguajes como PHP. También cuenta con soporte para depuración, control integrado de Git e \textit{IntelliSense}, una función de autocompletado de código\cite{VSCode}.
\begin{figure}[H]
	\centering
	\includegraphics[scale=0.05]{vscode}
	\caption{Logo de Visual Studio Code}
\end{figure}
\subsubsection{XAMPP}
XAMPP es una distribución de Apache gratuita que contiene MariaDB, PHP y Perl\cite{Xampp}. Fue usado inicialmente para trabajar con la base de datos y los PHP necesarios, pero una vez configurado el servidor Ubuntu 20.04 dejó de ser necesario.
\begin{figure}[H]
	\centering
	\includegraphics[scale=0.3]{xampp}
	\caption{Logo de XAMPP}
\end{figure}
\subsubsection{MobaXTerm}
MobaXTerm permite trabajar con herramientas de red remotas, como SSH, utilizando una terminal Unix desde Windows. Se ha usado para configurar el servidor Ubuntu 20.04 que aloja el servidor MySQL con la base de datos del sistema, y el servidor Apache con los PHP que se utilizan para trabajar con esta base de datos.
\begin{figure}[H]
	\centering
	\includegraphics[scale=0.5]{MobaXTerm}
	\caption{Logo de MobaXTerm}
\end{figure}

\subsubsection{Git}
Git es un software de control de versiones gratuito y de código abierto, diseñado para gestionar los cambios de un repositorio\cite{Git}.
\begin{figure}[H]
	\centering
	\includegraphics[scale=0.2]{git}
	\caption{Logo de Git}
\end{figure}

\newpage
\section[CSI 2: GENERACIÓN DEL CÓDIGO DE LOS COMPONENTES Y \\ PROCEDIMIENTOS]{CSI 2: GENERACIÓN DEL CÓDIGO DE LOS COMPONENTES Y PROCEDIMIENTOS}

\textcolor[rgb]{0.65,0.16,0}{Ejemplos de tablas descripción de clases}

\begin{table}[H]
\vspace{-4mm}
  \centering
  \caption{Descripción de diseño de LoginScreen}
    \begin{tabular}{p{8.645em}rr}
    \toprule
    \rowcolor[rgb]{ .851,  .886,  .953} \multicolumn{3}{p{31.285em}}{\textbf{LoginScreen}} \\
    \midrule
    \rowcolor[rgb]{ .949,  .949,  .949} \multicolumn{3}{p{31.285em}}{\textbf{Descripción}} \\
    \midrule
    \multicolumn{3}{p{31.285em}}{Es la encargada de las acciones y la renderización de la pantalla de inicio de sesión.} \\
    \midrule
    \rowcolor[rgb]{ .906,  .902,  .902} \multicolumn{3}{p{31.285em}}{\textbf{Atributos propuestos}} \\
    \midrule
    \multicolumn{3}{p{31.285em}}{-} \\
    \midrule
    \rowcolor[rgb]{ .906,  .902,  .902} \multicolumn{3}{p{31.285em}}{\textbf{Métodos propuestos}} \\
    \midrule
    \textbf{signInWithGoogle} & \multicolumn{2}{p{22.64em}}{Hace una llamada al objeto Fire para el inicio de sesión con Firebase authentication mediante una cuenta de Google.} \\
    \midrule
    \textbf{render} & \multicolumn{2}{r}{} \\
    \bottomrule
    \end{tabular}%
\end{table}%


\begin{table}[htbp]
  \centering
  \caption{Descripción de diseño de HomeScreen}
    \begin{tabular}{p{10em}rr}
    \toprule
    \rowcolor[rgb]{ .851,  .886,  .953} \multicolumn{3}{p{31.285em}}{\textbf{HomeScreen}} \\
    \midrule
    \rowcolor[rgb]{ .949,  .949,  .949} \multicolumn{3}{p{31.285em}}{\textbf{Descripción}} \\
    \midrule
    \multicolumn{3}{p{31.285em}}{Es la encargada de las acciones y la renderización de la pantalla de emergencia.} \\
    \midrule
    \rowcolor[rgb]{ .906,  .902,  .902} \multicolumn{3}{p{31.285em}}{\textbf{Atributos propuestos}} \\
    \midrule
    \multicolumn{3}{p{31.285em}}{-} \\
    \midrule
    \rowcolor[rgb]{ .906,  .902,  .902} \multicolumn{3}{p{31.285em}}{\textbf{Métodos propuestos}} \\
    \midrule
    \textbf{componentWillMount} & \multicolumn{2}{r}{} \\
    \midrule
    \textbf{emergencyCalling} & \multicolumn{2}{p{21.285em}}{Es el método encargado de redirigir la aplicación hacia el marcador con el 112 marcado.} \\
    \midrule
    \textbf{warnProtectors} & \multicolumn{2}{p{21.285em}}{[Falta implementar] Es el encargado de generar un mensaje de aviso a los protectores creando notificaciones push.} \\
    \midrule
    \textbf{render} & \multicolumn{2}{r}{} \\
    \bottomrule
    \end{tabular}%
\end{table}%

\newpage
\section{CSI 3: EJECUCIÓN DE LAS PRUEBAS UNITARIAS}


\newpage
\section{CSI 4: EJECUCIÓN DE LAS PRUEBAS DE INTEGRACIÓN}


\newpage
\section{CSI 5: EJECUCIÓN DE LAS PRUEBAS DEL SISTEMA}

\subsection{Prueba de Usabilidad}

\subsection{Pruebas de Accesibilidad} 
 
\subsubsection{Revisión Preliminar} 

\subsubsection{Evaluación de Conformidad} 

\subsubsection{Checklist del WCAG 2.1} 

\subsubsection{Accesibilidad con Dispositivos Móviles} 


\newpage
\section{CSI 6: ELABORACIÓN DE LOS MANUALES DE USUARIO}

\subsection{Manual de Instalación} 
En este manual se detallarán los pasos necesarios para realizar las instalaciones necesarias para la ejecución del sistema.\par
En primer lugar, es necesario instalar NodeJS (se puede descargar en \url{https://nodejs.org/en/download/}) y reiniciar el sistema, ya que con esta instalación se ha cambiado la configuración de variables del PATH.\par
Para los siguientes pasos, es necesario el uso de la terminal del sistema.\par
Usando npm, el gestor de paquetes de NodeJS, hay que instalar Angular CLI. Para ello hay que ejecutar el comando \textit{npm install -g @angular/cli}.
\begin{figure}[H]
\centering
\includegraphics[scale=1]{npmangularcli}
\caption{Instalación de Angular CLI}
\end{figure}
Por último, desde la carpeta que ubica tanto el proyecto del museo (museo-eii) como el de la administración (museo-eii-admin), se ejecuta el comando \textit{npm install} para instalar los paquetes necesarios.
\begin{figure}[H]
\centering
\includegraphics[scale=1]{npminstallmuseo}
\caption{Instalación de los paquetes del proyecto del museo}
\end{figure}
\begin{figure}[H]
\centering
\includegraphics[scale=1]{npminstalladmin}
\caption{Instalación de los paquetes del proyecto de administración}
\end{figure}


\subsection{Manual de Ejecución} 
Una vez completada la instalación siguiendo los pasos descritos en el apartado anterior, se pueden ejecutar ambas aplicaciones utilizando el comando \textit{ng serve -o}, \textit{npm start} o \textit{npm run ng serve -o}. Esto hará que la aplicación esté disponible en \url{http://localhost:4200}.
\begin{figure}[H]
\centering
\includegraphics[scale=0.65]{ngservemuseo}
\caption{Ejecución de la aplicación del museo}
\end{figure}
\begin{figure}[H]
\centering
\includegraphics[scale=0.65]{ngserveadmin}
\caption{Ejecución de la aplicación de administración}
\end{figure}


\subsection{Manual de Usuario} 
\subsubsection{Museo}
Al acceder a la web observamos la página de inicio. La parte superior de esta página está presente en toda la aplicación web y, por orden de izquierda a derecha, observamos:
\begin{itemize}
	\item El logo de la escuela, que redirige a esta página de inicio.
	\item Museo, que redirige a la vista general del museo.
	\item Acerca de.
	\item Un selector de idioma, que permite cambiar entre inglés y español.
\end{itemize}
La parte inferior, que contiene enlaces a las redes sociales de la escuela, también está presente en toda la aplicación web.\par
En la parte central se encuentra el contenido específico de la página de inicio: una bienvenida a la página web y un botón que nos dirige a la vista general del museo.
\begin{figure}[H]
\centering
\includegraphics[scale=0.25]{homeIUDef}
\caption{Manual de usuario: Inicio}
\end{figure}
En la vista general del museo hay una línea temporal y filtros de búsqueda.\par
En cada elemento de la línea temporal se muestra el nombre del periodo con un enlace al mismo, los años que comprende dicho periodo, y los nombres de los componentes pertenecientes al periodo, también con enlaces a cada uno de ellos.\par
La búsqueda puede realizarse por años o por nombre. Se puede filtrar por años mediante la barra deslizadora, y se mostrarán entonces todos aquellos periodos que, parcialmente o en su totalidad, tengan componentes pertenecientes a esos años. La búsqueda por nombre se realiza tras escribir en el recuadro de búsqueda y pulsar la tecla \textit{Enter}, y el resultado será aquellos periodos cuyo nombre o el nombre de alguno de sus componentes contenga el texto buscado.
\begin{figure}[H]
\centering
\includegraphics[scale=0.25]{museoIUDef}
\caption{Manual de usuario: Vista general del museo}
\end{figure}
Al entrar en un periodo, en la parte superior podemos ver un menú, en la izquierda se mostraría el periodo anterior cronológicamente, y en la izquierda el periodo siguiente (si existen). El contenido principal de la página son los detalles del periodo: nombre, características, una lista de curiosidades (sabías que...), eventos informáticos ocurridos en esos años, los componentes pertenecientes al periodo (mostrando una imagen y el nombre, con un enlace al componente), y una serie de sistemas famosos que llevan esos componentes. 
\begin{figure}[H]
\centering
\includegraphics[scale=0.25]{periodoIUDef}
\caption{Manual de usuario: Detalles del periodo (museo)}
\end{figure}
Por último, al acceder a un componente, podemos ver una galería de fotos que se abrirán en grande al pulsar sobre ellas, una descripción del componente y un listado de características. Además, en el lateral izquierdo hay un menú que permite navegar entre periodos (ver el anterior, el actual y el siguiente) y entre los componentes del periodo actual.
\begin{figure}[H]
\centering
\includegraphics[scale=0.25]{piezaIUDef}
\caption{Manual de usuario: Detalles del componente (museo)}
\end{figure}

\subsubsection{Administración del museo}
Al entrar a la web de administración del museo nos encontramos con el inicio de sesión. Es necesario indicar el correo electrónico y la contraseña para acceder.
\begin{figure}[H]
\centering
\includegraphics[scale=0.45]{loginIUDef}
\caption{Manual de usuario: Inicio de sesión}
\end{figure}
Si el administrador necesita cambiar la contraseña, podrá hacerlo accediendo a la máquina Ubuntu 20.04 donde se encuentra alojado el servidor Apache que contiene los archivos PHP y la base de datos del sistema, ya que habrá un fichero ejecutable que solicitará la nueva contraseña y realizará el cambio. Este fichero solo estará disponible en esta máquina ya que, al haber un solo usuario administrador, es más sencillo hacer el cambio de contraseña de esta forma en lugar de enviar un correo con un enlace temporal para realizarlo, y es seguro ya que el administrador es el único usuario con acceso a la máquina.
\begin{figure}[H]
\centering
\includegraphics[scale=1]{changepassphp}
\caption{Manual de usuario: Cambio de contraseña}
\end{figure}
Lo primero que se muestra una vez iniciada la sesión es un listado de los periodos existentes, mostrando sus nombres con un enlace a cada uno de ellos, y permitiendo editar y eliminar cada periodo. Eliminar un periodo borrará también los componentes asociados al mismo, para ello se mostrará un aviso y se pedirá confirmación. En el lateral izquierdo hay un menú que se incluye en todas las páginas de la aplicación, desde el que se puede acceder a este listado de periodos, y a los formularios para añadir periodos y componentes.
\begin{figure}[H]
\centering
\includegraphics[scale=0.35]{listadoPeriodosIUDef}
\caption{Manual de usuario: Listado de periodos}
\end{figure}
Los detalles de un periodo y del componente muestran los mismos datos explicados anteriormente en el manual de usuario del museo, con la diferencia de en que cada una de estas páginas se muestra una opción para editar el periodo o el componente que estemos visualizando, y en el listado de componentes del periodo también se da la opción de editar o eliminar cada uno de ellos.
\begin{figure}[H]
\centering
\includegraphics[scale=0.35]{periodoIU2Def}
\caption{Manual de usuario: Detalles de un periodo (administración)}
\end{figure}
\begin{figure}[H]
\centering
\includegraphics[scale=0.35]{compIUDef}
\caption{Manual de usuario: Detalles de un componente (administración)}
\end{figure}
En el formulario de añadir un periodo hay cuatro entradas de texto para nombre, detalles, curiosidades y eventos del periodo. Todos ellos deben rellenarse obligatoriamente para poder guardar el periodo. Si se pulsa el botón \textit{Cancelar}, el formulario se vaciará de nuevo. Al pulsar \textit{Guardar y continuar} con el formulario completo, se añadirá el periodo a la base de datos del sistema y nos redigirá al formulario para añadir componentes. En cambio, si el formulario no es válido se mostrará un error y no se añadirá.
\begin{figure}[H]
\centering
\includegraphics[scale=0.35]{añadirPeriodoIUDef}
\caption{Manual de usuario: Formulario para añadir un periodo}
\end{figure}
A la hora de editar un periodo, el formulario funcionará igual que al añadirlo, con la diferencia de que los valores iniciales serán los del periodo que se está editando.
\begin{figure}[H]
\centering
\includegraphics[scale=0.35]{editarPeriodoIUDef}
\caption{Manual de usuario: Formulario para editar un periodo}
\end{figure}
Los formularios para añadir y editar componentes funcionan de la misma forma que los de añadir y editar periodos, pero en este caso, hay campos que no son obligatorios, como la subida de imágenes y el sistema famoso. Además, al añadir o editar componentes se puede seleccionar su tipo: CPU o componente genérico. Al seleccionar CPU se muestran los campos de memoria ROM, memoria RAM, frecuencia de reloj, consumo energético, tamaño de palabra, nanómetros de transistores, passmark y número de transistores.
\begin{figure}[H]
\centering
\includegraphics[scale=0.35]{añadirCompIUDef}
\caption{Manual de usuario: Formulario para añadir un componente}
\end{figure}
\begin{figure}[H]
\centering
\includegraphics[scale=0.35]{editarCompIUDef}
\caption{Manual de usuario: Formulario para editar un componente}
\end{figure}


\subsection{Manual del Programador}
En este manual se explicará cómo ampliar la aplicación añadiendo nuevos tipos de componentes además de CPUs. Primero habría que crear una nueva clase para cada tipo que se desee añadir. Cada una de estas clases implementarán la interfaz \textit{MyComponent} y heredarán de \textit{GenericComp}. También habría que actualizar la enumeración \textit{CompTypes}. Estos tres elementos mencionados se encuentran en el archivo \textit{comp.ts}, que forma parte tanto del proyecto del museo (museo-eii) como de la administración (museo-eii-admin). Una vez realizado esto, común a ambos proyectos, se explicará qué debe añadirse a cada uno de ellos en específico, así como a la base de datos.

\subsubsection{Museo}
En el proyecto del museo (museo-eii) deberá generarse un componente de Angular para cada tipo añadido, se llamará \textit{`new type`DetailsComponent} y será hijo de \textit{CompDetailsComponent}, del que recibirá como input el atributo \textit{comp}. Este solo se mostrará cuando \textit{comp} sea una instancia del tipo correspondiente a `new type`. En \textit{`new type`-details.component.html} se listarán las características de \textit{comp}.\par
Además, en el método \textit{getComp} de \textit{CompDetailsComponent} habrá que añadir las comprobaciones necesarias para mostrar los nuevos tipos definidos.

\subsubsection{Administración del museo}
En el proyecto de la administración (museo-eii-admin) habrá que generar dos componentes de Angular nuevos por cada tipo añadido: 
\begin{itemize}
\item \textit{`new type`FormComponent}, hijo de \textit{AddCompComponent} y de \textit{EditCompComponent}. De ambos recibe como input el atributo \textit{model}. En \textit{`new type`-form.component.html} se incluirán los campos del formulario que se correspondan con las características del tipo creado. Se mostrará cuando \textit{model} sea una instancia del tipo correspondiente a `new type`. \par
En el método \textit{createModel} de \textit{AddCompComponent} habrá que añadir la opción de crear un objeto de este nuevo tipo, y también se añadirán las comprobaciones necesarias en los métodos \textit{isValid} y \textit{cloneComp} de \textit{AddCompComponent} y \textit{EditCompComponent}.
\item \textit{`new type`DetailsComponent}, hijo de \textit{MyComponentComponent}, del que recibe como input el atributo \textit{c}. En este caso, se hará exactamente lo mismo que lo mencionado anteriormente al añadir \textit{`new type`DetailsComponent} en el proyecto del museo, ya que ambos componentes son para mostrar las características de cada tipo.
\end{itemize}

\subsubsection{Base de datos}
En la base de datos habría que crear una tabla por cada nuevo tipo de componente, con los campos necesarios para este que no estén ya incluidos en la tabla \textit{components}. La clave primaria de esta tabla sería también una clave foránea, el identificador del componente en la tabla \textit{components}. Una vez creadas las tablas correspondientes, habría que modificar las comprobaciones y consultas realizadas en los archivos \textit{getComp.php, updateComp.php} y \textit{postComp.php} para incluir los nuevos tipos creados.


%\newpage
%\section{CSI 8: CONSTRUCCIÓN DE LOS COMPONENTES Y PROCEDIMIENTOS DE MIGRACIÓN Y CARGA INICIAL DE DATOS}



\newpage
\chapter{IMPLANTACIÓN Y ACEPTACIÓN DEL SISTEMA}
	\vspace{2cm}	
	\begin{center}
	{\Large \textbf{FASE DE DESARROLLO} \par}
	\end{center}
	\vspace{5cm}
	
	\begin{center}
	\Huge \textbf{IAS}\par
	\end{center}

\newpage

\section{IAS 1: ESTABLECIMIENTO DEL PLAN DE IMPLANTACIÓN}


\newpage
\section{IAS 4: CARGA DE DATOS AL ENTORNO DE OPERACIÓN}


\newpage
\section{IAS 5: PRUEBAS DE IMPLANTACIÓN DEL SISTEMA}


\newpage
\section{IAS 7: PREPARACIÓN DEL MANTENIMIENTO DEL SISTEMA}


\newpage
\section{IAS 8: ESTABLECIMIENTO DEL ACUERDO DE NIVEL DE SERVICIO}


\newpage
\section{IAS 9--10: PRESENTACIÓN Y APROBACIÓN DEL SISTEMA Y PASO A PRODUCCIÓN}


\newpage
\chapter{APÉNDICES}
\newpage

\section{PROBLEMAS ENCONTRADOS DURANTE EL DESARROLLO}

\newpage
\section{CONCLUSIONES}

\newpage
\section{AMPLIACIONES} 


%\newpage
%\section{REFERENCIAS BIBLIOGRÁFICAS}
\nocite{*} %El comando bibliography enseña solo las referencias que se hayan usado en el texto. Este comando permite "no citar" todas y así que aparezcan.
\bibliographystyle{ieeetr} 
\bibliography{references}

\newpage
\chapter*{ANEXOS}
\addcontentsline{toc}{chapter}{ANEXOS}
\newpage
\phantomsection
\section*{PLAN DE GESTIÓN DE RIESGOS}
\addcontentsline{toc}{section}{PLAN DE GESTIÓN DE RIESGOS}

\newpage
\section*{CONTENIDO ENTREGADO EN LOS ANEXOS} 
\addcontentsline{toc}{section}{CONTENIDO ENTREGADO EN LOS ANEXOS}

\subsection*{Contenidos} 

\textcolor[rgb]{0.65,0.16,0}{Ejemplo de como especificar los contenidos entregados}

Además de este documento, se hace entrega de una carpeta comprimida ``.zip'' en la que ahora se describirán sus contenidos. Se estructurará también la organización del código fuente.

\begin{itemize}
	\item \textbf{Planificación\_TFG.mpp} -> Archivo de Microsoft Project que contiene la planificación del proyecto entera.
	\item \textbf{Presupuesto-GuardMe\_TFG.xlsx} -> Archivo Microsoft Excel que contiene los cálculos del presupuesto del proyecto.
	\item \textbf{Diagramas} -> Carpeta que contiene todos los diagramas utilizados en este documento.
	\begin{itemize}
		\item \textit{Diagrama\_de\_paquetes.png}
		\item \textit{Diagrama\_firestore.png}
		\item \textit{Diagrama\_navegabilidad.png}
		\item \textit{Diagrama\_secuencia\_enviar.png}
		\item \textit{Diagrama\_secuencia\_visualizar.png}
		\item \textit{Diagrama\_UML-Diseño.png}
		\item \textit{Diagrama\_UML-Analisis.png}
	\end{itemize}
	\item \textbf{TFG\_codigo.zip} -> Carpeta comprimida con todo el código fuente.
\end{itemize}

Ahora se mostrará el contenido de dicha carpeta comprimida que contiene todo el código fuente de la aplicación la cual esta dividida a su vez en dos carpetas:

\paragraph*{AuthServerGuardMe}
Contiene el código que se aloja en \textit{Heroku} para darle funcionalidad al servidor. La clase principal es la llamada \texttt{mainAuthServer.js}.

\paragraph*{GuardMe}
Contiene el código fuente de la aplicación y se compone de las siguientes carpetas:
\begin{itemize}
	\item \textbf{assets} -> Carpeta que contiene los elementos gráficos usados en la aplicación. Se subdivide en una carpeta llamada \textit{images} que contiene todas las imagenes utilizadas para la construcción de la aplicación.
	\item \textbf{components} -> Carpeta que contiene el código para todos los componentes creados.
	\item \textbf{constants} -> Carpeta que contiene el código
	\item \textbf{docs} -> Carpeta que contiene los archivos html generados por JSDoc.
	\item \textbf{files} -> Carpeta en la que se encuentras los futuros archivos de Términos y Condiciones y Política de Privacidad entre otros.
	\item \textbf{modules\_LICENSES} -> Carpeta que contiene una por una todas las licencias de las librerías utilizadas en el desarrollo.
	\item \textbf{navigation} -> Carpeta que contiene las clases relativas a la navegación de la aplicación.
	\item \textbf{objects} -> Carpeta que contiene los objetos utilizados en el desarrollo que en este caso ha sido solo Fire.js.
	\item \textbf{screens} -> Carpeta que contiene todas las pantallas, agrupadas a su vez en subcarpetas que identifican la pantalla sobre la que están relacionadas.
	\item \textbf{styles} -> Carpeta que contiene todos los estilos de las pantallas, agrupadas a su vez en subcarpetas que siguen la misma estructura que \textit{screens}.
	\item \textbf{App.js} -> Clase principal y encargada de que comience la aplicación entera.
	\item \textbf{LICENSE} -> Licencia sobre el código fuente.
	\item \textbf{README.md} -> Archivo con la descripción del proyecto para la documentación y el repositorio de GitHub.
	\item \textbf{package.json} -> Archivo que contiene las librerías utilizadas en el proyecto.
	\item \textbf{app.json} -> Archivo que contiene la configuración de la aplicación.
	\item \textbf{configJSDoc.json} -> Archivo de configuración para la creación de documentación por parte de JSDoc.
	\item \textbf{Otros archivos} -> Los demás archivos no son relevantes ya que muchos se generan por defecto y los demás son configuraciones propias de expo.
\end{itemize}



\newpage
\nocite{*} %El comando bibliography enseña solo las referencias que se hayan usado en el texto. Este comando permite "no citar" todas y así que aparezcan.
\bibliographystyle{ieeetr} 
\bibliography{references}

\newpage





\end{document}
