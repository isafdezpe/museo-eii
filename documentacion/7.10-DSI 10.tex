
\subsection{Pruebas Unitarias y del Sistema} 

\begin{table}[H]
%\vspace{-4mm}
  \centering
  \caption{Diseño de pruebas: Consultar periodos (museo)}
    \begin{tabular}{p{9em}p{12em}p{15em}}
    \toprule
    \rowcolor[rgb]{ .851,  .886,  .953} \multicolumn{3}{p{36em}}{\textbf{Consultar periodos (museo)}} \\ \midrule
    \rowcolor[rgb]{ .949,  .949,  .949} \textbf{Prueba} & \textbf{Pasos} & \textbf{Resultado esperado}\\ \midrule
    \textbf{Obtener periodos existentes} & - Acceder a la vista general del museo & El sistema devolverá una lista de todos los periodos existentes. \\ \midrule
    \textbf{Obtener periodo por nombre} & - Acceder a la vista general del museo.\par - Introducir un texto en la barra de búsqueda. & El sistema devolverá una lista de los periodos cuyo nombre contenga el texto introducido. \\ \midrule
    \textbf{Obtener periodo por años} & - Acceder a la vista general del museo.\par - Cambiar los años especificados en la barra deslizadora. & El sistema devolverá una lista de los periodos cuyos años coincidan con los introducidos.\\ \bottomrule
    \end{tabular}%
\end{table}%
\begin{table}[H]
\vspace{-4mm}
  \centering
  \caption{Diseño de pruebas: Consultar componentes (museo)}
    \begin{tabular}{p{11em}p{11em}p{14em}}
    \toprule
    \rowcolor[rgb]{ .851,  .886,  .953} \multicolumn{3}{p{36em}}{\textbf{Consultar componentes (museo)}} \\ \midrule
    \rowcolor[rgb]{ .949,  .949,  .949} \textbf{Prueba} & \textbf{Pasos} & \textbf{Resultado esperado}\\ \midrule
    \textbf{Obtener componentes de un periodo} & - Acceder a un periodo. & El sistema devolverá una lista de todos los componentes pertenecientes al periodo. \\ \bottomrule
    \end{tabular}%
\end{table}%
\begin{table}[H]
\vspace{-4mm}
  \centering
  \caption{Diseño de pruebas: Iniciar sesión}
    \begin{tabular}{p{9em}p{11em}p{16em}}
    \toprule
    \rowcolor[rgb]{ .851,  .886,  .953} \multicolumn{3}{p{36em}}{\textbf{Iniciar sesión}} \\ \midrule
    \rowcolor[rgb]{ .949,  .949,  .949} \textbf{Prueba} & \textbf{Pasos} & \textbf{Resultado esperado}\\ \midrule
    \textbf{Iniciar sesión con datos válidos } & - Introducir el usuario \textit{uo257829@uniovi.es}.\par - Introducir la contraseña \textit{museoinfo2022}. & El sistema permitirá el acceso a la página de administración. \\ \midrule
    \textbf{Iniciar sesión con datos incorrectos} & - Introducir el usuario \textit{uo257829@uniovi.es}.\par - Introducir la contraseña \textit{123456}. & El sistema no permitirá el acceso y se mostrará un error. \\ \bottomrule
    \end{tabular}%
\end{table}%
\begin{table}[H]
\vspace{-4mm}
  \centering
  \caption{Diseño de pruebas: Consultar periodos (administración)}
    \begin{tabular}{p{9em}p{11em}p{16em}}
    \toprule
    \rowcolor[rgb]{ .851,  .886,  .953} \multicolumn{3}{p{36em}}{\textbf{Consultar periodos (administración)}} \\ \midrule
    \rowcolor[rgb]{ .949,  .949,  .949} \textbf{Prueba} & \textbf{Pasos} & \textbf{Resultado esperado}\\ \midrule
    \textbf{Obtener periodos existentes} & - Acceder al listado de periodos. & El sistema devolverá una lista de todos los periodos existentes. \\ \bottomrule
    \end{tabular}%
\end{table}%
\begin{table}[H]
\vspace{-4mm}
  \centering
  \caption{Diseño de pruebas: Añadir periodo}
    \begin{tabular}{p{8em}p{14em}p{14em}}
    \toprule
    \rowcolor[rgb]{ .851,  .886,  .953} \multicolumn{3}{p{36em}}{\textbf{Añadir periodo}} \\ \midrule
    \rowcolor[rgb]{ .949,  .949,  .949} \textbf{Prueba} & \textbf{Pasos} & \textbf{Resultado esperado}\\ \midrule
    \textbf{Añadir nuevo periodo (periodo 2)} & - Acceder a añadir un periodo.\par - Introducir los datos \textit{Periodo 2, detalles, eventos, curiosidades}. \par - Guardar periodo. & El sistema tendrá un periodo más. \\ \midrule
    \textbf{Añadir periodo que ya existe (periodo 1)} & - Acceder a añadir un periodo.\par - Introducir los datos \textit{Periodo 1, detalles, eventos, curiosidades}. \par - Guardar periodo. & El sistema no añadirá el periodo y responderá con un error.  \\ \midrule
    \textbf{Añadir periodo con campos vacíos} &  - Acceder a añadir un periodo.\par - Introducir los datos \textit{Periodo 2, ' ', eventos, curiosidades}. \par - Guardar periodo. & El sistema no añadirá el periodo y responderá con un error. \\ \bottomrule
    \end{tabular}%
\end{table}%
\begin{table}[H]
\vspace{-4mm}
  \centering
  \caption{Diseño de pruebas: Modificar periodo}
    \begin{tabular}{p{10em}p{13em}p{14em}}
    \toprule
    \rowcolor[rgb]{ .851,  .886,  .953} \multicolumn{3}{p{36em}}{\textbf{Modificar periodo}} \\ \midrule
    \rowcolor[rgb]{ .949,  .949,  .949} \textbf{Prueba} & \textbf{Entrada} & \textbf{Resultado esperado}\\ \midrule
    \textbf{Modificar periodo existente (periodo 1)} & - Introducir los datos \textit{Periodo 1 modificado, detalles, eventos, curiosidades, 1}.\par - Guardar periodo. & El sistema actualizará los datos del periodo. \\ \midrule
    \textbf{Modificar un periodo que no existe (periodo 25)} & - Introducir los datos \textit{Periodo 25 modificado, detalles, eventos, curiosidades, 25}.\par - Guardar periodo. & El sistemá responderá con un error. \\ \midrule
    \textbf{Modificar periodo dejando campos vacíos} & - Introducir los datos \textit{Periodo 1 modificado, detalles, eventos, ' ', 1}.\par - Guardar periodo. & El sistema no actualizará el periodo y responderá con un error. \\ \bottomrule
    \end{tabular}%
\end{table}%
\begin{table}[H]
\vspace{-4mm}
  \centering
  \caption{Diseño de pruebas: Eliminar periodo}
    \begin{tabular}{p{11em}p{11em}p{14em}}
    \toprule
    \rowcolor[rgb]{ .851,  .886,  .953} \multicolumn{3}{p{36em}}{\textbf{Eliminar periodo}} \\ \midrule
    \rowcolor[rgb]{ .949,  .949,  .949} \textbf{Prueba} & \textbf{Entrada} & \textbf{Resultado esperado}\\ \midrule
    \textbf{Eliminar un periodo existente (periodo 1)} & - Eliminar periodo 1.  & El sistema tendrá un periodo menos. \\ \midrule
    \textbf{Eliminar un periodo que no existe (periodo 25)} & - Eliminar periodo 25.  & El sistema responderá con un error. \\ \bottomrule
    \end{tabular}%
\end{table}%
\begin{table}[H]
\vspace{-4mm}
  \centering
  \caption{Diseño de pruebas: Consultar componentes (administración)}
    \begin{tabular}{p{11em}p{11em}p{14em}}
    \toprule
    \rowcolor[rgb]{ .851,  .886,  .953} \multicolumn{3}{p{36em}}{\textbf{Consultar componentes (administración)}} \\ \midrule
    \rowcolor[rgb]{ .949,  .949,  .949} \textbf{Prueba} & \textbf{Entrada} & \textbf{Resultado esperado}\\ \midrule
    \textbf{Obtener componentes de un periodo} & - Acceder a un periodo. & El sistema devolverá una lista de todos los componentes pertenecientes al periodo. \\ \bottomrule
    \end{tabular}%
\end{table}%
\begin{table}[H]
\vspace{-4mm}
  \centering
  \caption{Diseño de pruebas: Añadir componente}
    \begin{tabular}{p{13em}p{15em}p{8em}}
    \toprule
    \rowcolor[rgb]{ .851,  .886,  .953} \multicolumn{3}{p{36em}}{\textbf{Añadir componente}} \\ \midrule
    \rowcolor[rgb]{ .949,  .949,  .949} \textbf{Prueba} & \textbf{Entrada} & \textbf{Resultado esperado}\\ \midrule
    \textbf{Añadir nuevo componente (CPU 2)} & - Acceder a añadir componente. \par - Introducir los datos \textit{CPU 2, familia, descripción, años, periodo 1, etc.}\par - Guardar componente. & El sistema tendrá un componente más. \\ \midrule
    \textbf{Añadir componente que ya existe (CPU 1)} & - Acceder a añadir componente. \par - Introducir los datos \textit{CPU 1, familia, descripción, años, periodo 1, etc.}\par - Guardar componente.  & El sistema no añadirá el componente y responderá con un error.  \\ \midrule
    \textbf{Añadir componente a un periodo que no existe (CPU 3, periodo 25)} & - Acceder a añadir componente. \par - Introducir los datos \textit{CPU 3, familia, descripción, años, periodo 25, etc.}\par - Guardar componente.  & El sistema no añadirá el componente y responderá con un error.  \\ \midrule
    \textbf{Añadir componente con campos obligatorios vacíos} & - Acceder a añadir componente. \par - Introducir los datos \textit{CPU 3, familia, ' ', años, perido 1, etc.}\par - Guardar componente.  & El sistema no añadirá el componente y responderá con un error. \\ \bottomrule
    \end{tabular}%
\end{table}%
\begin{table}[H]
\vspace{-4mm}
  \centering
  \caption{Diseño de pruebas: Modificar componente}
    \begin{tabular}{p{13em}p{11em}p{12em}}
    \toprule
    \rowcolor[rgb]{ .851,  .886,  .953} \multicolumn{3}{p{36em}}{\textbf{Modificar componente}} \\ \midrule
    \rowcolor[rgb]{ .949,  .949,  .949} \textbf{Prueba} & \textbf{Entrada} & \textbf{Resultado esperado}\\ \midrule
    \textbf{Modificar componente existente (CPU 1)} & - Introducir los datos \textit{CPU 1 modificada, familia, descripción, años, periodo 1, 1, etc.}\par - Guardar componente. & El sistema actualizará los datos del componente. \\ \midrule
    \textbf{Modificar un componente que no existe (CPU 30)} & - Introducir los datos \textit{CPU 30 modificada, familia, descripción, años, periodo 1, 30, etc.}\par - Guardar componente. & El sistemá responderá con un error. \\ \midrule
    \textbf{Modificar componente dejando campos obligatorios vacíos} & - Introducir los datos \textit{CPU 1 modificada, familia, ' ', años, periodo 1, 1, etc.}\par - Guardar componente. & El sistema no actualizará el componente y responderá con un error. \\ \bottomrule
    \end{tabular}%
\end{table}%
\begin{table}[H]
\vspace{-4mm}
  \centering
  \caption{Diseño de pruebas: Eliminar componente}
    \begin{tabular}{p{13em}p{11em}p{12em}}
    \toprule
    \rowcolor[rgb]{ .851,  .886,  .953} \multicolumn{3}{p{36em}}{\textbf{Eliminar componente}} \\ \midrule
    \rowcolor[rgb]{ .949,  .949,  .949} \textbf{Prueba} & \textbf{Entrada} & \textbf{Resultado esperado}\\ \midrule
    \textbf{Eliminar un componente existente (CPU 1)} & - Eliminar CPU 1. & El sistema tendrá un componente menos. \\ \midrule
    \textbf{Eliminar un componente que no existe (CPU 30)} & - Eliminar CPU 30. & El sistema responderá con un error. \\ \bottomrule
    \end{tabular}%
\end{table}%


\subsection{Pruebas de Usabilidad y Accesibilidad} 
Las pruebas de usabilidad y accesibilidad de la aplicación web del museo serán realizadas por usuarios con distintos perfiles, mientras que las de la aplicación de administración solo las realizará el usuario administrador, ya que será el único que interactúe con este sistema. Los usuarios interactuarán con el sistema y rellenarán un cuestionario, que será detallado a continuación. También habrá un cuestionario en el que el responsable de las pruebas anotará las observaciones realizadas durante las pruebas.

\subsubsection{Cuestionario de evaluación}\label{sec:cuestionario-usabilidad}
\paragraph*{Preguntas de carácter general}
\begin{table}[H]
\centering
\caption{Pruebas de usabilidad: preguntas de carácter general}
\begin{tabular}{p{36em}}
\toprule
\rowcolor[rgb]{ .949,  .949,  .949} \textbf{¿Usa un ordenador frecuentemente?} \\ \midrule
\vspace{-4mm}
\begin{enumerate}
\item Todos los días
\item Varias veces a la semana
\item Ocasionalmente
\item Nunca
\end{enumerate} \\ \midrule
\rowcolor[rgb]{ .949,  .949,  .949} \textbf{¿Qué tipo de actividades realiza con el ordenador?} \\ \midrule
\vspace{-4mm}
\begin{enumerate}
\item Es parte de mi trabajo o profesión
\item Lo uso básicamente para ocio
\item Solo empleo aplicaciones estilo Office
\item Únicamente leo el correo y navego ocasionalmente
\end{enumerate} \\ \midrule
\rowcolor[rgb]{ .949,  .949,  .949} \textbf{¿Qué busca Vd. principalmente en una aplicación web?} \\ \midrule
\vspace{-4mm}
\begin{enumerate}
\item Que sea fácil de navegar
\item Que sea intuitiva
\item Que sea rápida
\end{enumerate} \\ \bottomrule
\end{tabular}
\end{table}

\paragraph*{Actividades guiadas}
Las actividades que realizarán los diferentes usuarios en la aplicación web del museo son las siguientes:
\begin{itemize}
\item Navegar por la linea temporal presente en la vista general del museo.
\item Realizar una búsqueda por años.
\item Realizar una búsqueda por nombre.
\item Consultar los detalles de un periodo.
\item Consultar los detalles de un componente.
\end{itemize}
El administrador realizará las actividades mencionadas y también las correspondientes a la aplicación de administración, que se muestran a continuación:
\begin{itemize}
\item Añadir un periodo.
\item Añadir un componente.
\item Editar un periodo.
\item Editar un componente.
\item Eliminar un periodo.
\item Eliminar un componente.
\end{itemize}

\paragraph*{Preguntas cortas sobre la aplicación y observaciones}
\begin{table}[H]
\centering
\caption{Pruebas de usabilidad: preguntas sobre la aplicación}
\begin{tabular}{p{15em}|p{4em}|p{7.5em}|p{7.5em}|p{3em}}
\toprule
\rowcolor[rgb]{.949,  .949,  .949} \textbf{Facilidad de uso} & \textbf{Siempre} & \textbf{Frecuentemente} & \textbf{Ocasionalmente} & \textbf{Nunca} \\ \midrule
\textit{¿Sabe dónde está dentro de la aplicación?} & & & & \\ \midrule
\textit{¿Necesita ayuda para utilizar la aplicación?} & & & & \\ \midrule
\textit{¿Le resulta sencillo el uso de la aplicación?} & & & & \\ \midrule
\textit{¿Identifica fácilmente la información que se le presenta?} & & & & \\ \midrule
\rowcolor[rgb]{.949,  .949,  .949} \textbf{Funcionalidad} & \textbf{Siempre} & \textbf{Frecuentemente} & \textbf{Ocasionalmente} & \textbf{Nunca} \\ \midrule
\textit{¿Funciona cada tarea como Vd. espera?} & & & & \\ \midrule
\textit{¿El tiempo de respuesta de la aplicación es muy grande?} & & & & \\ \midrule
\rowcolor[rgb]{ .851,  .886,  .953} \multicolumn{5}{p{36em}}{\textbf{Calidad del interfaz}} \\ \midrule
\rowcolor[rgb]{.949,  .949,  .949} \textbf{Aspectos gráficos} & \textbf{Muy adecuado} & \textbf{Adecuado} & \textbf{Poco adecuado} & \textbf{Nada adecuado} \\ \midrule
\textit{El tipo y el tamaño de letra es} & & & & \\ \midrule
\textit{Los iconos e imágenes usados son} & & & & \\ \midrule
\textit{Los colores empleados son} & & & & \\ \midrule
\rowcolor[rgb]{.949,  .949,  .949}\multicolumn{2}{p{19em}|}{\textbf{Diseño de la interfaz}} & \textbf{Sí} & \textbf{A veces} & \textbf{No} \\ \midrule
\multicolumn{2}{p{19em}|}{\textit{¿Le resulta fácil de usar?}} & & & \\ \midrule
\multicolumn{2}{p{19em}|}{\textit{¿El diseño de las pantallas es claro y atractivo?}} & & & \\ \midrule
\multicolumn{2}{p{19em}|}{\textit{¿Es coherente el diseño general del sitio web?}} & & & \\ \midrule
\multicolumn{2}{p{19em}|}{\textit{¿Cree que el programa está bien estructurado?}} & & & \\ \midrule
\rowcolor[rgb]{ .851,  .886,  .953}\multicolumn{5}{p{36em}}{\textbf{Observaciones}} \\ \midrule
\multicolumn{5}{p{36em}}{} \\ \bottomrule
\end{tabular}
\end{table}

\subsubsection{Cuestionario para el responsable de las pruebas} 
\begin{table}[H]
\centering
\caption{Pruebas de usabilidad: cuestionario para el responsable de las pruebas}
\begin{tabular}{p{12em}p{24em}}
\toprule
\rowcolor[rgb]{ .949,  .949,  .949}\multicolumn{2}{p{36em}}{\textbf{\textit{Nombre de la actividad}}} \\ \midrule
\textbf{Tiempo empleado:} &  \\ \midrule
\textbf{Problemas encontrados:} &  \\ \midrule
\textbf{Observaciones:} &  \\ \bottomrule
\end{tabular}
\end{table}

%\subsection{Pruebas de Accesibilidad} 

%\subsection{Pruebas de Rendimiento} 
