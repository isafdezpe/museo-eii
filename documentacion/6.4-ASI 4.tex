\subsection{Caso de Uso 1} 

\begin{table}[H]
  \centering
  \vspace{-5mm}
  \caption{Análisis del Caso de Uso 1}
    \begin{tabular}{p{7.5em}p{24.145em}}
    \toprule
    \rowcolor[rgb]{ .871,  .918,  .965} \multicolumn{2}{p{31.645em}}{\textbf{Consultar periodos}} \\
    \midrule
    \rowcolor[rgb]{ .906,  .902,  .902} \textbf{Precondiciones} & \cellcolor[rgb]{ 1,  1,  1}- \\
    \midrule
    \rowcolor[rgb]{ .906,  .902,  .902} \textbf{Postcondiciones} & \cellcolor[rgb]{ 1,  1,  1}- \\
    \midrule
    \rowcolor[rgb]{ .906,  .902,  .902} \textbf{Actores} & \cellcolor[rgb]{ 1,  1,  1}Usuario estándar \\
    \midrule
    \rowcolor[rgb]{ .906,  .902,  .902} \textbf{Descripción} & \cellcolor[rgb]{ 1,  1,  1}El usuario accederá a la vista principal del museo y podrá visualizar los periodos existentes. Podrá acceder a los periodos. Podrá acceder a los componentes de los periodos. Podrá realizar una búsqueda.\\
    \midrule
    \rowcolor[rgb]{ .906,  .902,  .902} \textbf{Escenarios          Secundarios} & \cellcolor[rgb]{ 1,  1,  1}  \\
    \bottomrule
    \end{tabular}%
\end{table}%
 
\subsection{Caso de Uso 2}
\begin{table}[H]
  \centering
  \vspace{-5mm}
  \caption{Análisis del Caso de Uso 2}
    \begin{tabular}{p{7.5em}p{24.145em}}
    \toprule
    \rowcolor[rgb]{ .871,  .918,  .965} \multicolumn{2}{p{31.645em}}{\textbf{Consultar componentes}} \\
    \midrule
    \rowcolor[rgb]{ .906,  .902,  .902} \textbf{Precondiciones} & \cellcolor[rgb]{ 1,  1,  1}- \\
    \midrule
    \rowcolor[rgb]{ .906,  .902,  .902} \textbf{Postcondiciones} & \cellcolor[rgb]{ 1,  1,  1}- \\
    \midrule
    \rowcolor[rgb]{ .906,  .902,  .902} \textbf{Actores} & \cellcolor[rgb]{ 1,  1,  1}Usuario estándar \\
    \midrule
    \rowcolor[rgb]{ .906,  .902,  .902} \textbf{Descripción} & \cellcolor[rgb]{ 1,  1,  1}El usuario accederá a la vista de un periodo y podrá visualizar los componentes pertenecientes al mismo. Podrá acceder a otros periodos. Podrá acceder a los otros componentes de ese periodo. \\
    \midrule
    \rowcolor[rgb]{ .906,  .902,  .902} \textbf{Escenarios          Secundarios} & \cellcolor[rgb]{ 1,  1,  1}  \\
    \bottomrule
    \end{tabular}%
\end{table}%
 
\subsection{Caso de Uso 3}
\begin{table}[H]
  \centering
  \vspace{-5mm}
  \caption{Análisis del Caso de Uso 3}
    \begin{tabular}{p{7.5em}p{24.145em}}
    \toprule
    \rowcolor[rgb]{ .871,  .918,  .965} \multicolumn{2}{p{31.645em}}{\textbf{Iniciar sesión}} \\
    \midrule
    \rowcolor[rgb]{ .906,  .902,  .902} \textbf{Precondiciones} & \cellcolor[rgb]{ 1,  1,  1}El usuario no debe haber iniciado sesión. \\
    \midrule
    \rowcolor[rgb]{ .906,  .902,  .902} \textbf{Postcondiciones} & \cellcolor[rgb]{ 1,  1,  1}- \\
    \midrule
    \rowcolor[rgb]{ .906,  .902,  .902} \textbf{Actores} & \cellcolor[rgb]{ 1,  1,  1}Usuario \\
    \midrule
    \rowcolor[rgb]{ .906,  .902,  .902} \textbf{Descripción} & \cellcolor[rgb]{ 1,  1,  1}El usuario accederá a la página principal de la aplicación de administración e introducirá su email y contraseña para iniciar sesión en el sistema. \\
    \midrule
    \rowcolor[rgb]{ .906,  .902,  .902} \textbf{Escenarios          Secundarios} & \cellcolor[rgb]{ 1,  1,  1} Los datos introducidos no se corresponden con los datos de un usuario con permiso de administrador. Se muestra un error y de nuevo se solicita iniciar sesión. \\
    \bottomrule
    \end{tabular}%
\end{table}%
 
\subsection{Caso de Uso 4}
\begin{table}[H]
  \centering
  \vspace{-5mm}
  \caption{Análisis del Caso de Uso 4}
    \begin{tabular}{p{7.5em}p{24.145em}}
    \toprule
    \rowcolor[rgb]{ .871,  .918,  .965} \multicolumn{2}{p{31.645em}}{\textbf{Consultar periodos}} \\
    \midrule
    \rowcolor[rgb]{ .906,  .902,  .902} \textbf{Precondiciones} & \cellcolor[rgb]{ 1,  1,  1}El usuario debe haber iniciado sesión. \\
    \midrule
    \rowcolor[rgb]{ .906,  .902,  .902} \textbf{Postcondiciones} & \cellcolor[rgb]{ 1,  1,  1}- \\
    \midrule
    \rowcolor[rgb]{ .906,  .902,  .902} \textbf{Actores} & \cellcolor[rgb]{ 1,  1,  1}Usuario administrador \\
    \midrule
    \rowcolor[rgb]{ .906,  .902,  .902} \textbf{Descripción} & \cellcolor[rgb]{ 1,  1,  1}El usuario accederá al listado de periodos. Podrá acceder a cada uno de ellos. \\
    \midrule
    \rowcolor[rgb]{ .906,  .902,  .902} \textbf{Escenarios          Secundarios} & \cellcolor[rgb]{ 1,  1,  1}Aún no existe ningún periodo en el sistema. Se muestra una tabla vacía. \\
    \bottomrule
    \end{tabular}%
\end{table}
 
\subsection{Caso de Uso 5}
\begin{table}[H]
  \centering
  \vspace{-5mm}
  \caption{Análisis del Caso de Uso 5}
    \begin{tabular}{p{7.5em}p{24.145em}}
    \toprule
    \rowcolor[rgb]{ .871,  .918,  .965} \multicolumn{2}{p{31.645em}}{\textbf{Añadir periodo}} \\
    \midrule
    \rowcolor[rgb]{ .906,  .902,  .902} \textbf{Precondiciones} & \cellcolor[rgb]{ 1,  1,  1}El usuario debe haber iniciado sesión. \\
    \midrule
    \rowcolor[rgb]{ .906,  .902,  .902} \textbf{Postcondiciones} & \cellcolor[rgb]{ 1,  1,  1}El periodo añadido se guardará en la base de datos. \\
    \midrule
    \rowcolor[rgb]{ .906,  .902,  .902} \textbf{Actores} & \cellcolor[rgb]{ 1,  1,  1}Usuario administrador \\
    \midrule
    \rowcolor[rgb]{ .906,  .902,  .902} \textbf{Descripción} & \cellcolor[rgb]{ 1,  1,  1}El usuario accede al formulario para añadir un periodo. Rellena los campos necesarios. Pulsa el botón de guardar. \\
    \midrule
    \rowcolor[rgb]{ .906,  .902,  .902} \textbf{Escenarios          Secundarios} & \cellcolor[rgb]{ 1,  1,  1}- Se pulsa el botón cancelar. El formulario se restablece.\par - Se intenta acceder a otra página de la aplicación sin haber guardado los cambios. Se avisa de la situación y se pide una confirmación para continuar.\par - No se puede añadir el periodo. Se mostrará un error avisando de la situación. \\
    \bottomrule
    \end{tabular}%
\end{table}%
 
\subsection{Caso de Uso 6}
\begin{table}[H]
  \centering
  \vspace{-5mm}
  \caption{Análisis del Caso de Uso 6}
    \begin{tabular}{p{7.5em}p{24.145em}}
    \toprule
    \rowcolor[rgb]{ .871,  .918,  .965} \multicolumn{2}{p{31.645em}}{\textbf{Modificar periodo}} \\
    \midrule
    \rowcolor[rgb]{ .906,  .902,  .902} \textbf{Precondiciones} & \cellcolor[rgb]{ 1,  1,  1}El usuario debe haber iniciado sesión. Debe existir al menos un periodo. \\
    \midrule
    \rowcolor[rgb]{ .906,  .902,  .902} \textbf{Postcondiciones} & \cellcolor[rgb]{ 1,  1,  1}Los cambios realizados al periodo se guardarán en la base de datos. \\
    \midrule
    \rowcolor[rgb]{ .906,  .902,  .902} \textbf{Actores} & \cellcolor[rgb]{ 1,  1,  1}Usuario administrador \\
    \midrule
    \rowcolor[rgb]{ .906,  .902,  .902} \textbf{Descripción} & \cellcolor[rgb]{ 1,  1,  1}El usuario accede al periodo deseado y selecciona la opción de editar. Se mostrará el formulario correspondiente. Se realizan los cambios en el formulario. Pulsa el botón de guardar. \\
    \midrule
    \rowcolor[rgb]{ .906,  .902,  .902} \textbf{Escenarios          Secundarios} & \cellcolor[rgb]{ 1,  1,  1}- Se pulsa el botón cancelar. El formulario se restablece.\par - Se intenta acceder a otra página de la aplicación sin haber guardado los cambios. Se avisa de la situación y se pide una confirmación para continuar.\par - No se puede modificar el periodo. Se mostrará un error avisando de la situación. \\
    \bottomrule
    \end{tabular}%
\end{table}%
 
\subsection{Caso de Uso 7}
\begin{table}[H]
  \centering
  \vspace{-5mm}
  \caption{Análisis del Caso de Uso 7}
    \begin{tabular}{p{7.5em}p{24.145em}}
    \toprule
    \rowcolor[rgb]{ .871,  .918,  .965} \multicolumn{2}{p{31.645em}}{\textbf{Eliminar periodo}} \\
    \midrule
    \rowcolor[rgb]{ .906,  .902,  .902} \textbf{Precondiciones} & \cellcolor[rgb]{ 1,  1,  1}El usuario debe haber iniciado sesión. Debe existir al menos un periodo. \\
    \midrule
    \rowcolor[rgb]{ .906,  .902,  .902} \textbf{Postcondiciones} & \cellcolor[rgb]{ 1,  1,  1}El periodo eliminado y los componentes que pertenecen al mismo se borrarán de la base de datos y dejarán de mostrarse en la aplicación. \\
    \midrule
    \rowcolor[rgb]{ .906,  .902,  .902} \textbf{Actores} & \cellcolor[rgb]{ 1,  1,  1}Usuario administrador \\
    \midrule
    \rowcolor[rgb]{ .906,  .902,  .902} \textbf{Descripción} & \cellcolor[rgb]{ 1,  1,  1}El usuario accede al periodo deseado y selecciona la opción de eliminar. Se pide confirmación para eliminarlo. Se acepta esta confirmación. \\
    \midrule
    \rowcolor[rgb]{ .906,  .902,  .902} \textbf{Escenarios          Secundarios} & \cellcolor[rgb]{ 1,  1,  1}- No se acepta la confirmación para eliminarlo. El periodo y sus componentes permanecen en la base de datos.\par - No se puede eliminar el periodo. Se mostrará un error avisando de la situación. \\
    \bottomrule
    \end{tabular}%
\end{table}%
 
\subsection{Caso de Uso 8}
\begin{table}[H]
  \centering
  \vspace{-5mm}
  \caption{Análisis del Caso de Uso 8}
    \begin{tabular}{p{7.5em}p{24.145em}}
    \toprule
    \rowcolor[rgb]{ .871,  .918,  .965} \multicolumn{2}{p{31.645em}}{\textbf{Consultar componentes}} \\
    \midrule
    \rowcolor[rgb]{ .906,  .902,  .902} \textbf{Precondiciones} & \cellcolor[rgb]{ 1,  1,  1}El usuario debe haber iniciado sesión. Debe existir al menos un periodo. \\
    \midrule
    \rowcolor[rgb]{ .906,  .902,  .902} \textbf{Postcondiciones} & \cellcolor[rgb]{ 1,  1,  1}- \\
    \midrule
    \rowcolor[rgb]{ .906,  .902,  .902} \textbf{Actores} & \cellcolor[rgb]{ 1,  1,  1}Usuario administrador \\
    \midrule
    \rowcolor[rgb]{ .906,  .902,  .902} \textbf{Descripción} & \cellcolor[rgb]{ 1,  1,  1}El usuario accederá a un periodo existente y visualizará los componentes pertenecientes a este. Podrá acceder a cada uno de ellos. \\
    \midrule
    \rowcolor[rgb]{ .906,  .902,  .902} \textbf{Escenarios          Secundarios} & \cellcolor[rgb]{ 1,  1,  1}Aún no existen componentes para el periodo que se consulta. Se mostrará una tabla vacía. \\
    \bottomrule
    \end{tabular}%
\end{table}%
 
\subsection{Caso de Uso 9}
\begin{table}[H]
  \centering
  \vspace{-5mm}
  \caption{Análisis del Caso de Uso 9}
    \begin{tabular}{p{7.5em}p{24.145em}}
    \toprule
    \rowcolor[rgb]{ .871,  .918,  .965} \multicolumn{2}{p{31.645em}}{\textbf{Añadir componente}} \\
    \midrule
    \rowcolor[rgb]{ .906,  .902,  .902} \textbf{Precondiciones} & \cellcolor[rgb]{ 1,  1,  1}El usuario debe haber iniciado sesión. Debe existir al menos un periodo. \\
    \midrule
    \rowcolor[rgb]{ .906,  .902,  .902} \textbf{Postcondiciones} & \cellcolor[rgb]{ 1,  1,  1}El componente añadido se guardará en la base de datos y se asociará al periodo correspondiente. \\
    \midrule
    \rowcolor[rgb]{ .906,  .902,  .902} \textbf{Actores} & \cellcolor[rgb]{ 1,  1,  1}Usuario administrador \\
    \midrule
    \rowcolor[rgb]{ .906,  .902,  .902} \textbf{Descripción} & \cellcolor[rgb]{ 1,  1,  1}El usuario accede al formulario para añadir un componente. Rellena los campos necesarios. Pulsa el botón de guardar. \\
    \midrule
    \rowcolor[rgb]{ .906,  .902,  .902} \textbf{Escenarios          Secundarios} & \cellcolor[rgb]{ 1,  1,  1}- Se pulsa el botón cancelar. El formulario se restablece.\par - Se intenta acceder a otra página de la aplicación sin haber guardado los cambios. Se avisa de la situación y se pide una confirmación para continuar.\par - No se puede añadir el componente. Se mostrará un error avisando de la situación.  \\
    \bottomrule
    \end{tabular}%
\end{table}%
 
\subsection{Caso de Uso 10}
\begin{table}[H]
  \centering
  \vspace{-5mm}
  \caption{Análisis del Caso de Uso 10}
    \begin{tabular}{p{7.5em}p{24.145em}}
    \toprule
    \rowcolor[rgb]{ .871,  .918,  .965} \multicolumn{2}{p{31.645em}}{\textbf{Modificar componente}} \\
    \midrule
    \rowcolor[rgb]{ .906,  .902,  .902} \textbf{Precondiciones} & \cellcolor[rgb]{ 1,  1,  1}El usuario debe haber iniciado sesión. Debe existir al menos un componente. \\
    \midrule
    \rowcolor[rgb]{ .906,  .902,  .902} \textbf{Postcondiciones} & \cellcolor[rgb]{ 1,  1,  1}Los cambios realizados en el componente se guardarán en la base de datos. \\
    \midrule
    \rowcolor[rgb]{ .906,  .902,  .902} \textbf{Actores} & \cellcolor[rgb]{ 1,  1,  1}Usuario administrador \\
    \midrule
    \rowcolor[rgb]{ .906,  .902,  .902} \textbf{Descripción} & \cellcolor[rgb]{ 1,  1,  1}El usuario accede al componente deseado y selecciona la opción de editar. Se mostrará el formulario correspondiente. Se realizan los cambios en el formulario. Pulsa el botón de guardar. \\
    \midrule
    \rowcolor[rgb]{ .906,  .902,  .902} \textbf{Escenarios          Secundarios} & \cellcolor[rgb]{ 1,  1,  1}- Se pulsa el botón cancelar. El formulario se restablece.\par - Se intenta acceder a otra página de la aplicación sin haber guardado los cambios. Se avisa de la situación y se pide una confirmación para continuar.\par - No se puede modificar el componente. Se mostrará un error avisando de la situación.  \\
    \bottomrule
    \end{tabular}%
\end{table}
 
\subsection{Caso de Uso 11}
\begin{table}[H]
  \centering
  \vspace{-5mm}
  \caption{Análisis del Caso de Uso 11}
    \begin{tabular}{p{7.5em}p{24.145em}}
    \toprule
    \rowcolor[rgb]{ .871,  .918,  .965} \multicolumn{2}{p{31.645em}}{\textbf{Eliminar componente}} \\
    \midrule
    \rowcolor[rgb]{ .906,  .902,  .902} \textbf{Precondiciones} & \cellcolor[rgb]{ 1,  1,  1}El usuario debe haber iniciado sesión. Debe existir al menos un componente. \\
    \midrule
    \rowcolor[rgb]{ .906,  .902,  .902} \textbf{Postcondiciones} & \cellcolor[rgb]{ 1,  1,  1}El componente eliminado  se borrará de la base de datos y dejará de mostrarse en la aplicación. \\
    \midrule
    \rowcolor[rgb]{ .906,  .902,  .902} \textbf{Actores} & \cellcolor[rgb]{ 1,  1,  1}Usuario administrador \\
    \midrule
    \rowcolor[rgb]{ .906,  .902,  .902} \textbf{Descripción} & \cellcolor[rgb]{ 1,  1,  1}El usuario accede al componente deseado y selecciona la opción de eliminar. Se pide confirmación para eliminarlo. Se acepta esta confirmación.  \\
    \midrule
    \rowcolor[rgb]{ .906,  .902,  .902} \textbf{Escenarios          Secundarios} & \cellcolor[rgb]{ 1,  1,  1}- No se acepta la confirmación para eliminarlo. El componente permanece en la base de datos.\par - No se puede eliminar el componente. Se mostrará un error avisando de la situación.  \\
    \bottomrule
    \end{tabular}%
\end{table}%